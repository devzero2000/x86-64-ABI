\chapter{Introduction}

The \xARCH architecture\footnote{The architecture specification is
  available on the web at \url{http://www.x86-64.org/documentation}.} is an
extension of the x86 architecture.  Any processor implementing the
x86-64 architecture specification will also provide compatiblity modes
for previous descendants of the Intel 8086 architecture, including
32-bit processors such as the Intel 386, Intel Pentium, and AMD K6-2
processor.  Operating systems conforming to the x86-64 ABI may provide
support for executing programs that are designed to execute in these
compatiblity modes.  The x86-64 ABI does not apply to such prorams;
this document applies only programs running in the ``long'' mode
provided by the x86-64 architecture.

Except where otherwise noted, the \xARCH architecture ABI follows the
conventions described in the \intelabi.  Rather than replicate the
entire contents of the \intelabi, the \xARCH ABI indicates only those
places where changes have been made to the \intelabi.

No attempt has been made to specify an ABI for languages other than C.
However, it is assumed that many programming languages will wish to
link with code written in C, so
that the ABI specifications documented here are relevant.%
\footnote{See section \ref{section-cpp} for details on C++ ABI.}

\section{Differences from the \intelabi}

The most fundamental differences from the \intelabi document
are as follows:
\begin{itemize}
\item Sizes of fundamental data types.
\item Parameter-passing conventions.
\item Floating-point computations.
\item Removal of the GOT register.
\item Use of RELA relocations.
\end{itemize}

%%% Local Variables:
%%% mode: latex
%%% TeX-master: "abi"
%%% End:
