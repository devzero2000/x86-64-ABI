\chapter{Conventions}

\editornote{This chapter is used to document some features special to
  the \xARCH ABI.  The different sections might be moved to another
  place or removed completely.}


\section{GOT pointer and IP relative addressing}

\index{global offset table}
A basic difference between the i386 ABI and the \xARCH ABI is the
way the GOT table is found.  The i386 ABI, like (most) other processor
specific ABIs, uses a dedicated register (\reg{ebx}) to address the
base of the GOT table.  The function prologue of every function needs
to set up this register to the correct value.  The \xARCH processor
family introduces a new IP-relative addressing mode which is used in
this ABI instead of using a dedicated register.

On \xARCH the GOT table contains 64 bit entries.

\section{Execution of 32bit programs}

% Let's follow Sparc64 and MIPS64

The \xARCH is able to execute 64 bit \xARCH and also 32 bit ia32
programs.  Libraries conforming to the \intelabi will live in the
normal places like \path{/lib}, \path{/usr/lib} and \path{/usr/bin}.
Libraries following the \xARCH, will use \path{lib64} subdirectories
for the libraries, e.g \path{/lib64} and \path{/usr/lib64}.  Programs
conforming to \intelabi and to the \xARCH ABI will share directories
like \path{/usr/bin}.  In particular, there will be no \path{bin64}
directory.

\section{C++\label{section-cpp}}

For the \textindex{C++} ABI we will use the ia64 C++ ABI and instantiate it
appropriately.  The current draft of that ABI is available at:\\
\url{http://www.codesourcery.com/cxx-abi/}


%%% Local Variables:
%%% mode: latex
%%% TeX-master: "abi"
%%% End:
