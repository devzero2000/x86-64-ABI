\documentclass[12pt]{report}

% All macros

%  First find out if we're running pdftex
\newif\ifpdf

\ifx\pdfoutput\undefined
  \pdffalse
\else
  \pdftrue
\fi

\ifpdf
\pdfcompresslevel 5
\fi

% Times is nicer with pdf
\usepackage{times}
% American style
\usepackage[american]{babel}

\usepackage[T1]{fontenc}

% To make typsetting easier
\usepackage{xspace}

% Generate an index
\usepackage{makeidx}
\makeindex

% Some convenient macros:
%  Add text to index and print it also
\newcommand{\textindex}[1]{#1\index{#1}\xspace}
% Add text to index and print it also with \code{}
\newcommand{\codeindex}[1]{\code{#1}\index{#1@\texttt{#1}}\xspace}
% A version without xspace - I had some strange problems with commas
% afterwards.
\newcommand{\codeindexwo}[1]{\texttt{#1}\index{#1@\texttt{#1}}}

% Control placement of floats
\usepackage{here}

% Version number of document - increment occasionally ;-)
\newcommand{\version}{0.91}

% Print a footer everywhere with current date
% prelim2e needs \thistime
\newcount\hours
\newcount\minutes
\def\SetTime{\hours=\time
        \global\divide\hours by 60
        \minutes=\hours
        \multiply\minutes by 60
        \advance\minutes by-\time
        \global\multiply\minutes by-1 }
\SetTime
\def\thistime{\number\hours:\ifnum\minutes<10 0\fi\number\minutes}
\usepackage[time]{prelim2e}
\renewcommand{\PrelimWords}{AMD64 ABI Draft \version}

% Some commands:
\newcommand{\editornote}[1]{\footnote{#1}}

%Typesetting of registers
\newcommand{\reg}[1]{{\texttt{\%#1}}\xspace}
\newcommand{\RAX}{\reg{rax}}
\newcommand{\RBX}{\reg{rbx}}
\newcommand{\RCX}{\reg{rcx}}
\newcommand{\RDX}{\reg{rdx}}
\newcommand{\RSI}{\reg{rsi}}
\newcommand{\RDI}{\reg{rdi}}
\newcommand{\RBP}{\reg{rbp}}
\newcommand{\RSP}{\reg{rsp}}
\newcommand{\RIP}{\reg{rip}}

%Typesetting of opcodes
\newcommand{\op}[1]{\texttt{#1}}

%Typesetting common names
\newcommand{\MMX}{\emph{MMX}\xspace}
\newcommand{\xARCH}{AMD64\xspace}
\newcommand{\threednow}{3DNow!\xspace}

% Typesetting paths and files
\newcommand{\path}[1]{\texttt{#1}\xspace}

% Typesetting program code
\newcommand{\code}[1]{\texttt{#1}\xspace}

% Long Hrule
\newcommand{\Hrule}{\noindent\rule{\linewidth}{0.3mm}}

% Use Hyperref for PDF support - this should be last
\ifpdf
\usepackage[pdftex]{hyperref}
\else
\usepackage[dvips]{hyperref}
\fi

% Make `_' an ordinary character.
\catcode`_=12

% The Intel386 psABI document.
\newcommand{\intelabi}{Intel386 ABI\xspace}

\newcommand{\byte}{byte\xspace}
\newcommand{\twobyte}{twobyte\xspace}
\newcommand{\fourbyte}{fourbyte\xspace}
\newcommand{\eightbyte}{eightbyte\xspace}
\newcommand{\eightbytes}{eightbytes\xspace}
\newcommand{\sixteenbyte}{sixteenbyte\xspace}


%%% Local Variables:
%%% mode: latex
%%% TeX-master: "abi"
%%% End:


\begin{document}

\author{Edited by\\
  Michael Matz\thanks{matz@suse.de},
  Jan Hubi\v{c}ka\thanks{jh@suse.cz}, Andreas Jaeger\thanks{aj@suse.de},
  Mark Mitchell\thanks{mark@codesourcery.com}}

\title{System V Application Binary Interface\\
{\Large AMD64 Architecture Processor Supplement\\
Draft Version \version}}
\maketitle
\tableofcontents
\listoftables
\listoffigures

\section*{Revision History}

\begin{description}


\item[0.98] Various clarifications and fixes according to feedback
  from Sun, thanks to Terrence Miller.  DWARF register numbers for some
  system registers, thanks to Jan Beulich.  Add \code{R_X86_64_SIZE32} and
  \code{R_X86_64_SIZE64} relocations; extend meaning of \code{e_phnum}
  to handle more than 0xffff program headers, thanks to Rod Evans.
  Add footnote about passing of \code{decimal} datatypes. 
\item[0.97] Integrate Fortran ABI.
\item[0.96] Use \code{SHF_X86_64_LARGE} instead \code{SHF_AMD64_LARGE}
  (thanks to Evandro Menezes).  Correct various grammatical errors
  noted by Mark F. Haigh, who also noted that there are no global VLAs
  in C99.  Thanks also to Robert R. Henry.
\item[0.95] Include description of the medium PIC memory model (thanks
  to Jan Hubi\v{c}ka) and large model (thanks to Evandro Menezes).
\item[0.94] Add sections in Development Environment, Program Loading,
  a description of EH_FRAME sections and general cleanups to make
  text in this ABI self-contained.  Thanks to Michael Walker and Terrence
  Miller.
\item[0.93] Add sections about program headers, new section types and
  special sections for unwinding information.  Thanks to Michael
  Walker.
\item[0.92] Fix some typos (thanks to Bryan Ford), add section about
  stack layout in the Linux kernel.  Fix example in figure
  \ref{fig_passing_example} (thanks to Tom Horsley).  Add section on
  unwinding through assembler (written by Michal Ludvig).  Remove
  \code{mmxext} feature (thanks to Evandro Menezes).  Add section on
  Fortran (by Steven Bosscher) and stack unwinding (by Jan
  Hubi\v{c}ka).
\item[0.91] Clarify that x87 is default mode, not MMX (by Hans Peter
  Anvin).
\item[0.90]
  Change DWARF register
  numbers again; mention that \code{__m128} needs alignment; fix typo
  in figure \ref{fig-stack-frame}; add some comments on kernel
  expectations; mention TLS extensions; add example for passing of
  variable-argument lists; change semantics of \RAX in
  variable-argument lists; improve formatting; mention that X87 class
  is not used for passing; make \code{/lib64} a Linux specific
  section; rename x86-64 to AMD64; describe passing of complex types.
  Special thanks to Andi Kleen, Michal Ludvig, Michael Matz, David
  O'Brien and Eric Young for their comments.
\item[0.21] Define \code{__int128} as class INTEGER in register
  passing.  Mention that \reg{al} is used for variadic argument lists.  Fix
  some textual problems.  Thanks to H. Peter Anvin, Bo Thorsen, and
  Michael Matz.
\item[0.20 --- 2002-07-11] Change DWARF register number values of
  \RBX, \RSI, \RSI (thanks to Michal Ludvig).  Fix footnotes for
  fundamental types (thanks to H. Peter Anvin). Specify \code{size\_t}
  (thanks to Bo Thorsen and Andreas Schwab).  Add new section on
  floating point environment functions.
\item[0.19 --- 2002-03-27] Set name of Linux dynamic linker, mention
  \reg{fs}.
  Incorporate changes from H. Peter Anvin <hpa@zytor.com>
  for booleans and define handling of sub-64-bit integer types in
  registers.
\end{description}


\chapter{Introduction}

Except where otherwise noted, the \xARCH architecture follows the
conventions described in the Intel386 psABI.  Rather than replicate
the entire contents of the Intel386 psABI document, this document
simply indicates the differences from that document.

\editornote{The following paragraphs might disappear again - for now
  they should avoid confusion}.
The \xARCH architecture defines a new mode, referred to as 32/64-bit
mode. All information in this document is relevant only to 32/64-bit
mode, unless stated otherwise.

This ABI specifies the language bindings for the C language except
where noted explicitly.
\editornote{For C++ we will use ia64 C++ ABI.}

\section{Major differences to i386 ABI}

The major differences of the \xARCH ABI to the i386 ABI are as
follows:
\begin{itemize}
\item 64 bit code and data model
\item No explicit GOT register
\item Different function calling conventions due to more registers
\item Usage of SSE/SSE2 for floating point
\item 128 bit long double floating point type. \editornote{This needs
    further discussion.}
\end{itemize}

%%% Local Variables: 
%%% mode: latex
%%% TeX-master: "abi"
%%% End: 


\chapter{Software Installation}

Not yet done.

%%% Local Variables: 
%%% mode: latex
%%% TeX-master: "abi"
%%% End: 

\chapter{Low Level System Information}

\section{Machine Interface}

\subsection{Processor Architecture}

\subsection{Data Representation}

Within this specification, the term \emph{halfword} refers to a 16-bit
object, the term \emph{word} refers to a 32-bit object, the term
\emph{doubleword} refers to a 64-bit object, and the term
\emph{quadword} refers to a 128-bit object.

\subsubsection{Fundamental Types}

Figure~\ref{basic-types} shows the correspondence between ISO C's
scalar types and the processor's.

\begin{figure}
  \caption{Scalar Types}\label{basic-types}
{ % Use small here - the table is still too large
  % Has anybody an idea how to shrink the table so that it fits the page?
  \small
  \begin{tabular}{l|l|c|c|l}
    \hline\noalign{\smallskip}
     & &  & \multicolumn{1}{c|}{Alignment} & \multicolumn{1}{c|}{x86-64} \\
    \multicolumn{1}{c|}{Type} & \multicolumn{1}{c|}{C}
     &  \texttt{sizeof} & (bytes)   
     & \multicolumn{1}{c|}{Architecture}  \\
    \hline
    & \texttt{char}        & 1 & 1 & signed byte \\
    & \texttt{signed char} & & \\
    \cline{2-5}
    & \texttt{unsigned char} & 1 & 1 & unsigned byte \\
    \cline{2-5}
    & \texttt{short} & 2 & 2 & signed halfword \\
    & \texttt{signed short} & & \\
    \cline{2-5}
    & \texttt{unsigned short} & 2 & 2 & unsigned halfword \\
    \cline{2-5}
    & \texttt{int} & 4 & 4 & signed word \\
    Integral & \texttt{signed int} & & \\
    & \texttt{enum} & & \\
    \cline{2-5}
    & \texttt{unsigned int} & 4 & 4 & unsigned word \\
    \cline{2-5}
    & \texttt{long} & 8 & 8 & signed doubleword \\
    & \texttt{signed long} & & \\
    & \texttt{long long} & & \\
    & \texttt{signed long long} & & \\
    \cline{2-5}
    & \texttt{unsigned long} & 8 & 8 & unsigned doubleword \\
    & \texttt{unsigned long long} & 8 & 8 & unsigned doubleword \\
    \hline
    & \texttt{__int128} & 16 & 8 & signed quadword \\
    & \texttt{signed __int128} & 16 & 8 & signed quadword \\
    \hline
    & \texttt{unsigned __int128} & 16 & 8 & unsigned quadword \\
    \hline
    Pointer & \texttt{\textit{any-type} *} & 8 & 8 & unsigned doubleword \\
    & \texttt{\textit{any-type} (*)()} & & \\
    \hline
    Floating-& \texttt{float} & 4 & 4 & single (IEEE) \\
    point & \texttt{double} & 8 & 8 & double (IEEE) \\
    & \texttt{long double} & 16 & 16 & 80-bit extended (IEEE) \\
    & \texttt{__float128} & 16 & 16 & 128-bit extended (IEEE) \\
    \hline
    Packed & \texttt{__m64} & 8 & 8 & \MMX{} and \threednow \\
    & \texttt{__m128} & 16 & 16 & SSE and SSE-2 \\
  \end{tabular}
}
\end{figure}

The \codeindex{__float128} type uses a 15-bit exponent, a 113-bit
mantissa (the high order significant bit is implicit) and an exponent
bias of 16383.\footnote{Initial implementations of the \xARCH
  architecture are expected to support operations on the
  \texttt{__float128} type only via software emulation.}

The \code{long double} type uses a 15 bit exponent, a 64-bit mantissa
with an explicit high order significand bit and an exponent bias of
16383.\footnote{This type is the x87 double extented precision data
  type.}  Although a \code{long double} requires 16 bytes of storage,
only the first 10 bytes are significant.  The remaining six bytes are
tail padding, and the contents of these bytes are undefined.

The \code{__int128} type is stored in little-endian order in memory,
i.e., the 64 low-order bits are stored at a a lower address than the
64 high-order bits.

A null pointer (for all types) has value zero.

Like the Intel386 architecture, the \xARCH architecture does not
require all data access to be properly aligned.  Accessing misaligned
data will be slower than accessing properly aligned data, but
otherwise there is no difference.

\subsubsection{Aggregates and Unions}

An array uses the same alignment as its elements, except that a local
or global array variable that requires at least 16 bytes, or a C99
local or global variable-length array variable, always has alignment
of at least 16 bytes.\footnote{The alignment requirement allows the
  use of SSE instructions when operating on the array.  The compiler
  cannot in general calculate the size of a variable-length array, but
  it is expected that most VLAs will require at least 16 bytes, so it
  is logical to mandate that VLAs have at least a 16-byte alignment.}

No other changes required.

\subsubsection{Bit-Fields}

Amend the description of bit-field ranges as follows:

\begin{figure}[h]
\Hrule
  \caption{Bit-Field Ranges}
  \begin{center}
    \leavevmode
    \begin{tabular}{l|l|l}
      \multicolumn{1}{c}{Bit-field Type} 
         & \multicolumn{1}{c}{Width $w$} 
         & \multicolumn{1}{c}{Range} \\
      \hline
      \texttt{signed long} & & $-2^{w - 1}$ to $2^{w-1}-1$ \\
      \texttt{long} & 1 to 64 & 0 to $2^{w}-1$ \\
      \texttt{unsigned long} & & 0 to $2^{w}-1$ \\
    \end{tabular}
  \end{center}
\Hrule
\end{figure}

The ABI does not permit bitfields having the type \texttt{__m64} or
\texttt{__m128}.  Programs using bitfields of these types are not
portable.

No other changes required.

\section{Function Calling Sequence}

This section describes the standard function calling sequence,
including stack frame layout, register usage, parameter passing and so
on.

The standard calling sequence requirements apply only to global
functions.  Local functions that are not reachable from other
compilation units may use different conventions.  Nevertheless, it is
recommended that all functions use the standard calling sequence when
possible.

\subsection{Registers and the Stack Frame}
\label{subsec-registers}

The \xARCH architecture provides 16 general purpose 64-bit registers.
In addition the architecture provides 16 SSE registers, each 128 bits
wide and 8 x87 floating point registers, each 80 bits wide.  Each of
the x87 floating point registers may be referred to in \MMX/\threednow
mode as a 64-bit register.  All of these registesr are global to all
procedures in a running program.

This subsection discusses usage of each register.  Registers \RBP, \RBX and
\reg{r12} through \reg{r15} ``belong'' to the calling function and the
called function is required to preserve their values.  In other words,
a called function must preserve these registers' values for its
caller.  Remaining registers ``belong'' to the called
function.\footnote{Note that in contrast to the \intelabi, \RDI,
  and \RSI belong to the called function, not the caller.}  If a
calling function wants to preserve such a register value across a
function call, it must save the value in its local stack frame.

The CPU shall be in MMX mode upon entry to a function.  Therefore,
every function that uses the x87 register stack is required to issue
an \op{emms} or \op{femms} instruction before accessing the x87
register stack.\footnote{All \MMX{} registers are caller-saved, so
  callees that make use of the x87 register stack may use the faster
  \op{femms} instruction.}  The direction flag in the \reg{eflags}
register must be clear on function entry, and on function return.

\subsection{The Stack Frame}
In addition to registers, each function has a frame on the run-time
stack.  This stack grows downwards from high addresses.  Figure
\ref{fig-stack-frame} shows the stack organization.

\begin{figure}
\Hrule
  \caption{Stack Frame}
  \label{fig-stack-frame}
  \begin{center}
    \begin{tabular}{r|c|l}
      \noalign{\smallskip}
      \multicolumn{1}{l}{\bf Position} &
      \multicolumn{1}{c}{\bf Contents} &
      \multicolumn{1}{l}{\bf Frame} \\
      \noalign{\smallskip}  \cline{2-3}
      \code{8n+16(\RBP)} & argument doubleword $n$ \\
      & \dots & Previous \\
      \code{16(\RBP)} & argument doubleword $0$ \\
      \cline{1-3} 
      \code{8(\RBP)} & return address \\ \cline{2-2}
      \code{0(\RBP)} & previous \RBP value \\ 
      \cline{2-2}
      \code{-8(\RBP)} & unspecified & Current \\ 
      & \dots & \\ 
      \code{0(\RSP)} & variable size \\
      \cline{2-2}
      \code{128(\RSP)} & red zone\\
    \end{tabular}
  \end{center}
\Hrule
\end{figure}

The end of the input argument area shall be aligned on a 16 byte
boundary.  In other words, the value $(\RSP - 8)$ is always a multiple
of $16$ when control is transferred to the function entry point.  The
stack pointer, \RSP, always points to the end of the latest allocated
stack frame.  \footnote{The conventional use of \RBP{} as a frame
  pointer for the stack frame may be avoided by using \RSP (the stack
  pointer) to index into the stack frame.  This technique saves two
  instructions in the prologue and epilogue and makes one additional
  general-purpose register (\RBP) available.}

The 128-byte area beyond the location pointed to by \RSP is considered
to be reserved and shall not be modified by signal or interrupt
handlers.\footnote{Locations within 128 bytes can be addressed using
  one-byte displacements.}  Therefore, functions may use this area for
temporary data that is not needed across function calls.  In
particular, leaf functions may use this area for their entire stack
frame, rather than adjusting the stack pointer in the prologue and
epilogue.

\subsection{Parameter Passing}

After the argument values have been computed, they are placed in
registers, or pushed on the stack.  The arguments are processed in
right-to-left order.  Since the stack grows downwards, the rightmost
argument will have the highest address.\footnote{Right-to-left order
  makes the handling of functions that take a variable number of
  arguments simpler.  The location of the first argument can always be
  computed statically, based on the type of that argument.  It would
  be difficult to compute the address of the first argument if the
  arguments were pushed in left-to-right order.}

For each argument, the following method is used to determine the
location in which the argument is passed.  In all cases where a value
is pushed on the stack, the stack shall be left doubleword-aligned.
In all cases, the value pushed shall be located on a boundary whose
alignment is the maximum of 8 (doubleword alignment) and the alignment
of the type being pushed.  If the value does not consume the entire
doubleword, the contents of the unused storage located at higher
numbered addresses within the doubleword is undefined.

If the argument is a scalar type, other than \code{__m64} or
\code{__int128}, then it is placed in the next available register
suitable for its type.  In particular, arguments of integral or
pointer type are placed in the next available general purpose
register, taken in the order \RDI, \RSI, \RDX, \RCX, \reg{r8}, and
\reg{r9}.  \footnote{Note that \reg{r11} is neither required to
  be preserved, nor is it used to pass arguments.  Making this
  register available as scratch register means that code in the PLT
  need not spill any registers when computing the address to which
  control needs to be transferred.  \RAX is used to indicate the
  number of SSE arguments passed to a function requiring a variable
  number of arguments. \reg{r10} is used for passing a function's
  static chain pointer.}  Arguments of floating point type and of type
\code{__m128}, are placed in the next available SSE register, taken in
order from \reg{xmm0} to \reg{xmm15}.  If no registers are available,
the values are placed on the stack.

Scalar values of type \code{__int128} are handled almost as if they
consisted of two separate 64-bit arguments of type \code{long}.  The
64 low-order bits are considered to be the first argument, and
therefore processed second.\footnote{Thus, if the value is placed on
  the stack, the low-order bits will be at a lower address, which
  allows the caller to access the value normally.}  If there are not
enough registers available to allow passing both arguments in
registers, then the value is passed on the stack.

Scalar values of type \code{__m64} are always placed on the stack.

Structure or union objects with more than 16 bytes, those that contain
scalar components of type \code{__m64}, or, in C++,
non-POD\footnote{The term POD is from the ANSI/ISO C++ Standard, and
  stands for Plain Ol' Data.  Although the exact definition is
  technical, a POD is essentially a structure or union that could
  could have been written in C; there cannot be any member functions,
  or base classes, or similar C++ extensions.}  structure or union
types,\footnote{A non-POD object cannot be returned in registers
  because such objects must have well defined addresses; the address
  at which an object is constructed (by the caller) and the address at
  which the object is destroyed (by the callee) must be the same.
  Similar issues apply when returning a non-POD object from a
  function.} are always passed on the stack.  The stack is aligned as
required by the alignment of the structure or union type being passed.
Then, the structure value is copied onto the stack.

Structure or union objects that contain no more than 16 bytes and, in
C++, are PODs, are passed in registers, if registers are available.
If the entire structure cannot be placed in registers as described
below, then the structure is passed on the stack, as above.

If the structure contains a single scalar component, then it is passed
as if it were a single argument of that scalar type.

Otherwise, the first (or only, if the structure has no more than 8
bytes), doubleword of the structure is passed first.  If there is only
one component, the argument is passed as if it were an argument of
that type.  Otherwise, if all scalar components in the first component
are of floating point type, the first doubleword is passed as if it
were an argument of floating point type.  If all the components begin
at the same address, then the component is passed as if it were an
argument of the longest floating point type used by these components.
Otherwise, the argument is passed as if it were an argument of type
\code{__float128}.\footnote{In this case, there will be no padding
  bits, because this case can only occur if there is a value of type
  \code{float} in the second word of the doubleword.}

Otherwise, the doubleword is passed as if it were a single integer
argument, using the smallest type sufficient to contain the argument.
Any bytes not used by the value passed have undefined contents.  

If the structure has more than 8 bytes, then the process is repeated
for the second doubleword.

\subsection{Functions Returning Scalars or No Value}

If a function returns a scalar of type \code{__m64}, the value is
returned in \reg{mm0}.  If the function returns a scalar of some other
type, the value is returned in a register or registers.  The register
or registers chosen are the same as those that would be selected for
a value of the same type to be passed to a function taking only one
argument of that same type, except that \RAX is added at the beginning
of the list of available registers.\footnote{This implies that only
  \RAX, \RBX, \reg{mm0} and \reg{xmm0} will be used as return
  registers.}

\subsection{Functions Returning Structures or Unions}

If a function returns a structure or union whose size is greater than
16 bytes, or, for C++, if the structure or union is a non-POD, then
the caller provides space for the return value and passes the address
of this storage in \RAX as if it were the first argument to the
function.  In effect, this address becomes a ``hidden'' first
argument.

In this case, the function also sets \RAX to the value of the original
address in the callee's area before it returns.  Thus when the caller
receives control again, the address of the returned object resides in
register \RAX and can be used to access the object.
  
If a function returns a structure or union whose size is less than or
equal to 16 bytes, and, for C++, the structure or union is a POD, then
the structure is returned in registers.  The registers chosen for a
given structure or union type are the same as those that would be
selected for a value of the same type to be passed to a function
taking only one argument of that same structure type, except that \RAX
is added at the beginning of the list of available registers.

\section{Operating System Interface}

\subsection{Virtual Address Space}

Although the \xARCH architecture uses 64-bit pointers, implementations
are only required to handle 48-bit addresses.  Therefore, conforming
processes may only use addresses from \texttt{0x0000000000000000} to
\texttt{0x00007fffffffffff}\footnote{0x0000ffffffffffff is not a
  canoncial address and cannot be used.}.

No other changes required.

\subsection{Page Size}

Systems are permitted to use any page size between 4KB and 64KB,
inclusive.

No other changes required.

\subsection{Virtual Address Assignments}

Conceptually processes have the full address space available.
In practice, however, several factors limit the size of a process.
\begin{itemize}
  \item The system reserves a configuration dependent amount of virtual space.
  \item The system reserves a configuration dependent amount of space per
    process.
  \item
    A process whose size exceeds the system's available combined physical
    memory and secondary storage cannot run. Although some physical memory
    must be present to run any process, the system can execute processes that
    are bigger than physical memory, paging them to and from secondary storage.
    Nonetheless, both physical memory and secondary storage are
    shared resources. System load, which can vary from one program execution
    to the next, affects the available amount.
\end{itemize}

\begin{figure}[H]
\Hrule
  \caption{Virtual Address Configuration}
  \label{fig-address}
  \begin{center}
    \begin{tabular}{r|c|l}
      \noalign{\smallskip}  \cline{2-2}
      \verb|0xffffffffffffffff| & Reserved system area & End of memory\\ 
      & \dots & \\ \cline{2-2}
      & \dots & \\
      \verb|0x80000000000| & Dynamic segments & \\ \cline{2-2}
      & \dots & \\
      \verb|0| & Process segments & Beginning of memory\\ \cline{2-2}
    \end{tabular}
  \end{center}
\Hrule
\end{figure}

Although applications may control their memory assignments, the typical
arrangement appears in figure \ref{fig-cfg}.

\begin{figure}[H]
\Hrule
  \caption{Conventional Segment Arrangements}
  \label{fig-cfg}
  \begin{center}
    \begin{tabular}{r|c|l}
      \cline{2-2}
      & \dots & \\
      \verb|0x80000000000| & Dynamic segments & \\ \cline{2-2}
      & Stack segment & \\ 
      & \dots & \\ \cline{2-2}
      & \dots & \\
      & Data segments & \\ \cline{2-2}
      & \dots & \\
      \verb|0x10000| & Text segments & \\ \cline{2-2}
      \verb|0| & Unmapped & \\ \cline{2-2}
    \end{tabular}
  \end{center}
\Hrule
\end{figure}

\section{Coding Examples}

The following sections show only the difference to the i386 ABI.

\subsection{Position-Independend Function Prologue}

\xARCH does not need any function prologue for calculating the global
offset table address since it does not have an explicit GOT pointer.

\subsection{Data Objects}

Not done yet.

\subsection{Function Calls}

\begin{figure}[H]
\Hrule
\caption{Position-Independent Direct Function Call}
\begin{center}
\begin{tabular}{|l|c|l|}
\cline{1-1}\cline{3-3}
extern void function ();  &&.globl function\\
function ();              &&call function@PLT\\
\cline{1-1}\cline{3-3}
\end{tabular}
\end{center}
\Hrule
\end{figure}

\begin{figure}[H]
\Hrule
\caption{Position-Independent Indirect Function Call}
\begin{center}
\begin{tabular}{|l|c|l|}
\cline{1-1}\cline{3-3}
extern void (*ptr) ();    &&.globl ptr, name\\
extern void name ();      && \\
ptr = name;               &&movl ptr@GOTPCREL(\%rip), \%rax  \\
                          &&movl name@GOTPCREL(\%rip), \%rdx  \\
                          &&movl \%rdx, (\%rax)  \\
                          &&  \\
(*ptr)();                 &&movl ptr@GOTPCREL(\%rip), \%rax  \\
                          &&call *(\%rax)  \\
\cline{1-1}\cline{3-3}
\end{tabular}
\end{center}
\Hrule
\end{figure}

\subsection{Variable Argument Lists}

Some otherwise portable C programs depend on the argument passing
scheme, implicitly assuming that 1) all arguments are passed on the
stack, and 2) arguments appear in increasing order on the stack.
Programs that make these assumptions never have been portable, but
they have worked on many implementations. However, they do not work on
the \xARCH architecture because some arguments are passed in
registers.  Portable C programs must use the header files
\code{<stdarg.h>} or \code{<varargs.h>} in order to handle variable
argument lists.

When a function taking variable-arguments is called, \reg{r9} must be
set to the eight times the number of floating point parameters passed
to the function in SSE registers.  The callee can use the contents of
\reg{r9} to allocate stack space in which to save the values.

%%% Local Variables: 
%%% mode: latex
%%% TeX-master: "abi"
%%% End: 


\chapter{Object Files}

\section{ELF Header}

\subsection{Machine Information}

For file identification in \texttt{e_ident}, the x86-64 architecture
requires the following values.

\begin{table}[h]
  \begin{center}
    \begin{tabular}[t]{l|l}
      \multicolumn{1}{c}{Position} & \multicolumn{1}{c}{Value} \\
      \hline
      \texttt{e_ident[EI_CLASS]} & \texttt{ELFCLASS64} \\
      \texttt{e_ident[EI_DATA]} & \texttt{ELFDATA2LSB}
    \end{tabular}
  \end{center}
  \caption{x86-64 Identification}
\end{table}

\section{Sections}

No changes required.

\section{Symbol Table}

No changes required.

\section{Relocation}

\subsection{Relocation Types}

The x86-64 ABI adds one additional field:

\begin{figure}[h]
  \begin{picture}(300,100)
    \put(0,66){\framebox(150, 33){31\hfill\textit{word32}\hfill 0}}
    \put(0,0){\framebox(300, 33){63\hfill\textit{word64}\hfill 0}}
  \end{picture}
  \caption{Relocatable Fields}
\end{figure}

\noindent
\begin{tabular*}{\textwidth}{l@{\extracolsep{\fill}}p{4in}}
\textit{word32} & This specifies a 32-bit field occupying 4 bytes
                  with arbitrary byte alignment.  These values use
                  the smae byte order as other word values in the
                  x86-64 architecture. \\
\textit{word64} & This specifies a 64-bit field occupying 8 bytes
                  with arbitrary byte alignment.  These values use
                  the smae byte order as other word values in the
                  x86-64 architecture. \\
\end{tabular*}

The x86-64 ABI architectures uses only \texttt{Elf64_Rel} relocation
entries.

\begin{figure}[h]
  \begin{center}
    \begin{tabular}[t]{l|r|l|l}
      \multicolumn{1}{c}{Name} & 
      \multicolumn{1}{c}{Value} & 
      \multicolumn{1}{c}{Field} & 
      \multicolumn{1}{c}{Calculation} \\
      \hline
      \texttt{R_X8664_NONE}  & 0 & none & none \\
      \texttt{R_X8664_64}    & 1 & \textit{word64} & \texttt{S + A} \\
      \texttt{R_X8664_PC32}  & 2 & \textit{word32} & \texttt{S + A - P} \\
      \texttt{R_X8664_GOT32} & 3 & \textit{word32} & \texttt{G + A - P} \\
      \texttt{R_X8664_PLT64} & 4 & \textit{word64} & \texttt{L + A - P} \\
      \texttt{R_X8664_COPY}  & 5 & none            & none \\
      \texttt{R_X8664_GLOB_DATA} & 6 & \textit{word64} & \texttt{S} \\
      \texttt{R_X8664_JMP_SLOT} & 7 & \textit{word64} & \texttt{S} \\
      \texttt{R_X8664_RELATIVE} & 8 & \textit{word64} & \texttt{B + A} \\
      \texttt{R_X8664_GOTOFF} & 9 & \textit{word64} & \texttt{S + A - GOT} \\
      \texttt{R_X8664_GOTPC} & 10 & \textit{word64} & \texttt{GOT + A - P} \\
    \end{tabular}
  \end{center}
  \caption{Relocation Types}
\end{figure}

The special semantics for these relocation types are identical to
those used for the Intel386 architecture.
\footnote{Even though the x86-64 architecture supports IP-relative
  addressing modes, a GOT is still required since the address from the
  a particular instruction to a particular data item cannot be
  known by the static linker.}
\footnote{Note that the
  x86-64 architecture assumes that GOT offsets remain 32-bit values.
  This choice means that a maximum of $2^{32}/8 = 2^{29}$ entries can
  be placed in the GOT.  However, that should be more than enough for
  most programs.  In the event that it is not enough, the linker could
  create multiple GOTs.  Because 32-bit offsets are used, loads of
  global data do not require loading the offset into a displacement
  register; the base plus immediate displacement addressing form can
  be used.}

%%% Local Variables: 
%%% mode: latex
%%% TeX-master: "abi"
%%% End: 

\chapter{Program Loading and Dynamic Linking}


\section{Program Loading}

No changes required.


\section{Dynamic Linking}

We need three parameters for the dynamic linker to fixup the
PLT:
\begin{enumerate}
\item The address of the entry point of the dynamic linker.
\item The relocation entry that needs to be relocated (either GOT or
  PLT, depending on the proposal)
\item A special parameter to identify the shared object.
\end{enumerate}

\editornote{A location for these parameters has to be determined.}

\editornote{So far only the conventions for shared objects have been
  specified.  The conventions for the executable need to be written
  down.}
% executable: PIC or not PIC?

\subsubsection{Dynamic Section}

No changes required.

\subsubsection{Global Offset Table}

\index{global offset table}

\editornote{Some minor changes might be required, depending on the
  proposal.  This will be clarified later.}

\begin{figure}[H]
\caption{Global Offset Table}
\begin{center}
\fbox{\code{extern Elf64_Addr _GLOBAL_OFFSET_TABLE_ [];}}
\end{center}
\end{figure}

\subsubsection{Function Addresses}

No changes required.

\subsubsection{Procedure Linkage Table}

\editornote{Either of the three proposals will be implemented.  Some
  details need to filled in, like the number of reserved entries and
  the exact algorithm used.}

\paragraph{Proposal 1: RW-PLT}

% This has been copied from the i386 ABI and the SPARC ABI.

Much as the global offset table redirects position-independent address
calculations to absolute locations, the procedure linkage table
redirects position-independent function calls to absolute locations.
The link editor cannot resolve execution transfers (such as function
calls) from one executable or shared object to another.  Consequently,
the link editor arranges to have the program transfer control to
entries in the procedure linkage table.  On the \xARCH architecture,
procedure linkage tables reside in private data.  The dynamic linker
determines the destinations' absolute addresses and modifies the the
procedure linkage table's memory image accordingly.  The dynamic
linker thus can redirect the entries without compromising the
position-independence and sharability of the program's text.
Executable files and shared object files have separate procedure
linkage tables.

\begin{figure}[H]
\caption{Position-Independent Direct Function Call}
\begin{verbatim}
extern void function ();  .globl function
function ();              call function@PLT
\end{verbatim}
\end{figure}

\begin{figure}[H]
\caption{Position-Independent Indirect Function Call}
\begin{verbatim}
extern void (*ptr) ();    .globl ptr, name
extern void name ();
ptr = name;               movl ptr@GOTPC(%rip), %rax
                          movl name@GOTPC(%rip), %rdx
                          movl %rdx, (%rax)

(*ptr)();                 movl ptr@GOTPC(%rip), %rax
                          call *(%rax)
\end{verbatim}
\end{figure}

\begin{figure}[H]
\caption{Absolute Procedure Linkage Table}
\begin{tabular}{lll}
.PLT0: & pushl & got\_plus\_8; GOT[1]\\
& jmp &got\_plus\_16 ; GOT[2] \\
& nop & \\
& nop & \\
& nop & \\
& nop & \\
.PLT1: & jmp & .PLT1a \\
.PLT1a& pushl & \$offset \\
&jmp &.PLT0@PC \\
.PLT2: & jmp& name2\\
&pushl & \$offset \\
& jmp & .PLT0@PC \\
&\dots\\
\end{tabular}
\end{figure}


\begin{figure}[H]
\caption{Position-Independent Procedure Linkage Table}
\begin{tabular}{lll}
.PLT0: & pushl & GOT+8(\%rip); GOT[1]\\
& jmp &GOT+16(\%rip) ; GOT[2] \\
& nop & \\
& nop & \\
& nop & \\
& nop & \\
.PLT1: & jmp & PLT1a\\
.PLT1a& pushl & \$offset \\
&jmp &.PLT0@PC \\
.PLT2: & jmp& name2\\
&pushl & \$offset \\
& jmp & .PLT0@PC \\
&\dots\\
\end{tabular}
\end{figure}

\editornote{The costs for a call are (after fixing the PLT): one call
  and one direct jmp, giving an overhead of one jump.  The problem is
  that we have self modifying code.}

\paragraph{Proposal 2: RO-PLT}

\editornote{This is ia32 enhanced to 64 bits without an explicit GOT
  register.}

% This has been copied from the i386 ABI.
Much as the global offset table redirects position-independent address
calculations to absolute locations, the procedure linkage table
redirects position-independent function calls to absolute locations.
The link editor cannot resolve execution transfers (such as function
calls) from one executable or shared object to another.  Consequently,
the link editor arranges to have the program transfer control to
entries in the procedure linkage table.  On the \xARCH architecture,
procedure linkage tables reside in shared text, but they use address
in the private global offset table.  The dynamic linker determines the
destinations' absolute addresses and modifies the global offset
table's memory image accordingly.  The dynamic linker thus can
redirect the entries without compromising the position-independence
and sharability of the program's text.  Executable files and shared
object files have separate procedure linkage tables.  



\begin{figure}[H]
\caption{Position-Independent Direct Function Call}
\begin{verbatim}
extern void function ();  .globl function
function ();              call function@PLT
\end{verbatim}
\end{figure}

\begin{figure}[H]
\caption{Absolute Procedure Linkage Table}
\begin{tabular}{lll}
.PLT0: & pushl & got\_plus\_8; GOT[1]\\
& jmp &got\_plus\_16 ; GOT[2] \\
& nop & \\
& nop & \\
& nop & \\
& nop & \\
.PLT1: & jmp & *name1\_in\_GOT\\
.PLT1a& pushl & \$offset \\
&jmp &.PLT0@PC \\
.PLT2: & jmp& *name2\_in\_GOT\\
&pushl & \$offset \\
& jmp & .PLT0@PC \\
&\dots\\
\end{tabular}
\end{figure}

\begin{figure}[H]
\caption{Position-Independent Indirect Function Call}
\begin{verbatim}
extern void (*ptr) ();    .globl ptr, name
extern void name ();
ptr = name;               movl ptr@GOTPC(%rip), %rax
                          movl name@GOTPC(%rip), %rdx
                          movl %rdx, (%rax)

(*ptr)();                 movl ptr@GOTPC(%rip), %rax
                          call *(%rax)
\end{verbatim}
\end{figure}

\begin{figure}[H]
\caption{Position-Independent Procedure Linkage Table}
\begin{tabular}{lll}
.PLT0: & pushl & GOT+8(\%rip); GOT[1]\\
& jmp &GOT+16(\%rip) ; GOT[2] \\
& nop & \\
& nop & \\
& nop & \\
& nop & \\
.PLT1: & jmp &*name1@GOTPC(\%rip)\\
& pushl & \$offset \\
&jmp &.PLT0@PC \\
.PLT2: & jmp&*name2@GOTPC(\%rip)\\
&pushl & \$offset \\
& jmp & .PLT0@PC \\
&\dots\\
\end{tabular}
\end{figure}

\editornote{The costs for a call are (after fixing the PLT): one call
  and one memory indirect jmp, giving an overhead of one jump and one
  memory reference.}

\paragraph{Proposol 3: Indirect Calls}

%\editornote{This was suggested to me by Wayne Meretsky}

\editornote{Since we only relocate the GOT entries, we have a
  read-only PLT.  The disadvantage of this proposal is that
  executables need to be compiled as PIC.}

% This has been copied from the i386 ABI.
Much as the global offset table redirects position-independent address
calculations to absolute locations, the procedure linkage table
redirects position-independent function calls to absolute locations.
The link editor cannot resolve execution transfers (such as function
calls) from one executable or shared object to another.  Consequently,
the link editor arranges to have the program transfer control to
entries in the procedure linkage table.  On the \xARCH architecture,
procedure linkage tables reside in shared text, but they use address
in the private global offset table.  The dynamic linker determines the
destinations' absolute addresses and modifies the global offset
table's memory image accordingly.  The dynamic linker thus can
redirect the entries without compromising the position-independence
and sharability of the program's text.  Executable files and shared
object files have separate procedure linkage tables.  

\begin{figure}[H]
\caption{Position-Independent Direct Function Call}
\begin{verbatim}
extern void function ();  .globl function
function ();              call *name@GOTPC(%rip)
\end{verbatim}
\end{figure}


\begin{figure}[H]
\caption{Position-Independent Indirect Function Call}
\begin{verbatim}
extern void (*ptr) ();    .globl ptr, name
extern void name ();
ptr = name;               movl ptr@GOTPC(%rip), %rax
                          movl name@GOTPC(%rip), %rdx
                          movl %rdx, (%rax)

(*ptr)();                 movl ptr@GOTPC(%rip), %rax
                          call *(%rax)
\end{verbatim}
\end{figure}


\begin{figure}[H]
\caption{Procedure Linkage Table}
\begin{tabular}{lll}
.PLT0: & pushl & GOT+8(\%rip); GOT[1]\\
& jmp &GOT+16(\%rip) ; GOT[2] \\
& nop & \\
& nop & \\
& nop & \\
& nop & \\
.PLT1: & pushl & \$offset \\
&jmp &.PLT0@PC \\
.PLT2: &pushl & \$offset \\
& jmp & .PLT0@PC \\
&\dots\\
\end{tabular}
\end{figure}


\editornote{The costs for a call are (after fixing the PLT): one
  memory indirect call, giving an overhead of one memory reference.}

\paragraph{Comparison of Proposals}

\begin{figure}
\caption{Comparison of Different PLT Proposals}

\begin{center}
\begin{tabular}{llll}
    \hline\noalign{\smallskip}
&RW-PLT & RO-PLT & Indirect Calls \\
    \noalign{\smallskip}\hline\noalign{\smallskip}
PLT location & private data & shared text & shared text \\
Address location & PLT & GOT & GOT \\
Overhead after relocation & 1 jump & 1 memory ref. & 1 memory ref.\\
&&and 1 jump &  \\
Self modifying code? & Yes &  No & No\\
Executable is PIC? & No & No & Yes \\
Other ABIs & Sparc & i386 & None? \\
\end{tabular}
\end{center}
\end{figure}
                                

% common stuff
\subsection{Programm Interpreter}

The name and location of the program interpreter \index{program
  interpreter} is%
\footnote{Every Operating System will change this.
  For example Linux will use \path{/lib/ld-linux-x86-64.so.1}.}:

\bigskip
\path{/usr/lib/ld.so.1}

%%% Local Variables:
%%% mode: latex
%%% TeX-master: "abi"
%%% End:

\chapter{Libraries}

A further review of the \intelabi is needed.

\section{C Library}

\subsection{Global Data Symbols}

The symbols \code{_fp_hw}, \code{__flt_rounds} and \code{__huge_val}
are not provided by the \xARCH ABI.

\section{Unwind Library Interface}

This section defines the \textindex{Unwind Library interface}%
\footnote{The overall structure and the external interface is derived
from the IA-64 UNIX System V ABI},
expected to be provided by any \xARCH psABI-compliant system.
This is the interface on which the C++ ABI exception-handling
facilities are built. We assume as a basis the
\textindex{Call Frame Information tables} described in the
\textindex{DWARF Debugging Information Format} document.

This section is meant to specify a language-independent interface that
can be used to provide higher level exception-handling facilities such
as those defined by C++.

The unwind library interface consists of at least the following routines:\\
\codeindex{\_Unwind\_RaiseException},\\
\codeindex{\_Unwind\_Resume},\\
\codeindex{\_Unwind\_DeleteException},\\
\codeindex{\_Unwind\_GetGR},\\
\codeindex{\_Unwind\_SetGR},\\
\codeindex{\_Unwind\_GetIP},\\
\codeindex{\_Unwind\_SetIP},\\
\codeindex{\_Unwind\_GetRegionStart},\\
\codeindex{\_Unwind\_GetLanguageSpecificData},\\
\codeindex{\_Unwind\_ForcedUnwind}

In addition, two datatypes are defined (\codeindex{\_Unwind\_Context}
and \codeindex{\_Unwind\_Exception}) to interface a calling runtime
(such as the C++ runtime) and the above routine. All routines and
interfaces behave as if defined \texttt{extern "C"}. In particular,
the names are not mangled. All names defined as part of this interface
have a \texttt{"\_Unwind\_"} prefix.

Lastly, a language and vendor specific personality routine will be
stored by the compiler in the unwind descriptor for the stack frames
requiring exception processing. The personality routine is called by
the unwinder to handle language-specific tasks such as identifying the
frame handling a particular exception.

\subsection{Exception Handler Framework}

\subsubsection{Reasons for Unwinding}
There are two major reasons for unwinding the stack:
\begin{itemize}
  \item exceptions, as defined by languages that support them (such as C++)
  \item ``forced'' unwinding (such as caused by \codeindex{longjmp} or thread
    termination).
\end{itemize}

The interface described here tries to keep both similar. There is a
major difference, however.
\begin{itemize}

\item In the case where an exception is thrown, the stack is unwound
while the exception propagates, but it is expected that the personality
routine for each stack frame knows whether it wants to catch the exception
or pass it through. This choice is thus delegated to the personality
routine, which is expected to act properly for any type of exception,
whether ``native'' or ``foreign''.  Some guidelines for ``acting properly''
are given below.

\item During ``forced unwinding'', on the other hand, an external agent is
driving the unwinding. For instance, this can be the \texttt{longjmp}
routine. This external agent, not each personality routine,
knows when to stop unwinding. The fact that a personality routine is
not given a choice about whether unwinding will proceed is indicated by the
\codeindex{\_UA\_FORCE\_UNWIND} flag.
\end{itemize}

To accommodate these differences, two different routines are proposed.
\codeindex{\_Unwind\_RaiseException} performs exception-style unwinding,
under control of the personality routines. \codeindex{\_Unwind\_ForcedUnwind},
on the other hand, performs unwinding, but gives an external agent the
opportunity to intercept calls to the personality routine. This is done using
a proxy personality routine, that intercepts calls to the personality routine,
letting the external agent override the defaults of the stack frame's
personality routine.

As a consequence, it is not necessary for each personality routine to know
about any of the possible external agents that may cause an unwind. For
instance, the C++ personality routine need deal only with C++ exceptions
(and possibly disguising foreign exceptions), but it does not need to know
anything specific about unwinding done on behalf of \texttt{longjmp} or
pthreads cancellation.

\subsubsection{The Unwind Process}

The standard ABI exception handling/unwind process begins with the raising
of an exception, in one of the forms mentioned above. This call specifies an
exception object and an exception class.

The runtime framework then starts a two-phase process:
\begin{itemize}
\item In the \emph{search} phase, the framework repeatedly calls the
personality routine, with the \codeindex{\_UA\_SEARCH\_PHASE} flag as
described below, first for the current \RIP and register state, and then
unwinding a frame to a new \RIP at each step, until the personality
routine reports either success (a handler found in the queried frame)
or failure (no handler) in all frames. It does not actually restore the
unwound state, and the personality routine must access the state through
the API.

\item If the search phase reports a failure, e.g. because no handler was
found, it will call \codeindex{terminate()} rather than commence phase 2.

If the search phase reports success, the framework restarts in the
\emph{cleanup} phase. Again, it repeatedly calls the personality
routine, with the \codeindex{\_UA\_CLEANUP\_PHASE} flag as described
below, first for the current \RIP and register state, and then unwinding
a frame to a new \RIP at each step, until it gets to the frame with
an identified handler. At that point, it restores the register state,
and control is transferred to the user landing pad code.
\end {itemize}

Each of these two phases uses both the unwind library and the personality
routines, since the validity of a given handler and the mechanism for
transferring control to it are language-dependent, but the method of
locating and restoring previous stack frames is language-independent.

A two-phase exception-handling model is not strictly necessary to
implement C++ language semantics, but it does provide some benefits. For
example, the first phase allows an exception-handling mechanism to
\emph{dismiss} an exception before stack unwinding begins, which allows
\emph{resumptive} exception handling (correcting the exceptional condition
and resuming execution at the point where it was raised). While C++ does
not support resumptive exception handling, other languages do, and the
two-phase model allows C++ to coexist with those languages on the stack.

Note that even with a two-phase model, we may execute each of the two
phases more than once for a single exception, as if the exception was
being thrown more than once. For instance, since it is not possible to
determine if a given catch clause will rethrow or not without executing
it, the exception propagation effectively stops at each catch clause,
and if it needs to restart, restarts at phase 1. This process is not
needed for destructors (cleanup code), so the phase 1 can safely process
all destructor-only frames at once and stop at the next enclosing
catch clause.

For example, if the first two frames unwound contain only cleanup code,
and the third frame contains a C++ catch clause, the personality routine
in phase 1, does not indicate that it found a handler for the first two
frames. It must do so for the third frame, because it is unknown how the
exception will propagate out of this third frame, e.g. by rethrowing the
exception or throwing a new one in C++.

The API specified by the \xARCH psABI for implementing this framework
is described in the following sections.

\subsection{Data Structures}

\subsubsection{Reason Codes}

The unwind interface uses reason codes in several contexts to identify the
reasons for failures or other actions, defined as follows:

\code{
\begin{tabular}{l}
    typedef enum \{\\
    \ \ \ \_URC\_NO\_REASON = 0,\\
    \ \ \ \_URC\_FOREIGN\_EXCEPTION\_CAUGHT = 1,\\
    \ \ \ \_URC\_FATAL\_PHASE2\_ERROR = 2,\\
    \ \ \ \_URC\_FATAL\_PHASE1\_ERROR = 3,\\
    \ \ \ \_URC\_NORMAL\_STOP = 4,\\
    \ \ \ \_URC\_END\_OF\_STACK = 5,\\
    \ \ \ \_URC\_HANDLER\_FOUND = 6,\\
    \ \ \ \_URC\_INSTALL\_CONTEXT = 7,\\
    \ \ \ \_URC\_CONTINUE\_UNWIND = 8\\
    \} \_Unwind\_Reason\_Code;\\
\end{tabular}
}

The interpretations of these codes are described below.

\subsubsection{Exception Header}

The unwind interface uses a pointer to an exception header object
as its representation of an exception being thrown. In general,
the full representation of an exception object is language- and
implementation-specific, but it will be prefixed by a header understood
by the unwind interface, defined as follows:

\code{
\begin{tabular}{l}
    typedef void (*\_Unwind\_Exception\_Cleanup\_Fn)\\
    \ \ (\_Unwind\_Reason\_Code reason,\\
    \ \ \ struct \_Unwind\_Exception *exc);\\
\end{tabular}
}

\code{
\begin{tabular}{ll}
    struct \_Unwind\_Exception \{\\
    \ \ \ uint64 & exception\_class;\\
    \ \ \ \_Unwind\_Exception\_Cleanup\_Fn & exception\_cleanup;\\
    \ \ \ uint64 & private\_1;\\
    \ \ \ uint64 & private\_2;\\
    \};\\
\end{tabular}
}

An \code{\_Unwind\_Exception} object must be \eightbyte aligned.  The first
two fields are set by user code prior to raising the exception, and the
latter two should never be touched except by the runtime.

The \code{exception\_class} field is a language- and implementation-specific
identifier of the kind of exception. It allows a personality routine
to distinguish between native and foreign exceptions, for example.
By convention, the high 4 bytes indicate the vendor (for instance
AMD$\backslash$0), and the low 4 bytes indicate the language.  For the C++
ABI described in this document, the low four bytes are C++$\backslash$0.

The \code{exception\_cleanup} routine is called whenever an exception object
needs to be destroyed by a different runtime than the runtime
which created the exception object, for instance if a Java exception
is caught by a C++ catch handler. In such a case, a reason code (see
above) indicates why the exception object needs to be deleted:

\begin{description}
\item[\code{\_URC\_FOREIGN\_EXCEPTION\_CAUGHT = 1}] This indicates that a
     different runtime caught this exception. Nested foreign exceptions,
     or rethrowing a foreign exception, result in undefined behavior.

\item[\code{\_URC\_FATAL\_PHASE1\_ERROR = 3}] The personality routine encountered
     an error during phase 1, other than the specific error codes defined.

\item[\code{\_URC\_FATAL\_PHASE2\_ERROR = 2}] The personality routine
  encountered an error during phase 2, for instance a stack corruption.
\end{description}

Normally, all errors should be reported during phase 1 by returning
from \code{\_Unwind\_RaiseException}. However, landing pad code could cause
stack corruption between phase 1 and phase 2. For a C++ exception,
the runtime should call \code{terminate()} in that case.

The private unwinder state (\code{private\_1} and \code{private\_2}) in an exception
object should be neither read by nor written to by personality routines or
other parts of the language-specific runtime.  It is used by the specific
implementation of the unwinder on the host to store internal information,
for instance to remember the final handler frame between unwinding phases.

In addition to the above information, a typical runtime such as the
C++ runtime will add language-specific information used to process the
exception.  This is expected to be a contiguous area of memory after
the \code{\_Unwind\_Exception} object, but this is not required as
long as the matching personality routines know how to deal with it,
and the \code{exception\_cleanup} routine de-allocates it properly.

\subsubsection{Unwind Context}

The \code{\_Unwind\_Context} type is an opaque type used to refer to a
system-specific data structure used by the system unwinder.  This context
is created and destroyed by the system, and passed to the personality
routine during unwinding.

\code{
struct \_Unwind\_Context\\
}

\subsection{Throwing an Exception}

\subsubsection{\code{\_Unwind\_RaiseException}}

\code{
\begin{tabular}{l}
   \_Unwind\_Reason\_Code \_Unwind\_RaiseException\\
   \ \ ( struct \_Unwind\_Exception *exception\_object );\\
\end{tabular}
}

Raise an exception, passing along the given exception object,
which should have its \code{exception\_class} and \code{exception\_cleanup} fields
set. The exception object has been allocated by the language-specific
runtime, and has a language-specific format, except that it must
contain an \code{\_Unwind\_Exception} struct (see Exception Header above).
\code{\_Unwind\_RaiseException} does not return, unless an error condition is
found (such as no handler for the exception, bad stack format, etc.).
In such a case, an \code{\_Unwind\_Reason\_Code} value is returned.

Possibilities are:
\begin{description}
\item[\code{\_URC\_END\_OF\_STACK}] The unwinder encountered the end of the
   stack during phase 1, without finding a handler. The unwind runtime
   will not have modified the stack. The C++ runtime will normally call
   \code{uncaught\_exception()} in this case.
\item[\code{\_URC\_FATAL\_PHASE1\_ERROR}] The unwinder encountered an unexpected
   error during phase 1, e.g. stack corruption. The unwind runtime will
   not have modified the stack. The C++ runtime will normally call
   \code{terminate()} in this case.
\end{description}

If the unwinder encounters an unexpected error during phase 2, it
should return \code{\_URC\_FATAL\_PHASE2\_ERROR} to its caller.  In
C++, this will usually be \code{\_\_cxa\_throw}, which will call
\code{terminate()}.

The unwind runtime will likely have modified the stack (e.g. popped
frames from it) or register context, or landing pad code may have
corrupted them. As a result, the the caller of \code{\_Unwind\_RaiseException}
can make no assumptions about the state of its stack or registers.

\subsubsection{\code{\_Unwind\_ForcedUnwind}}

\code{
\begin{tabular}{l}
   typedef \_Unwind\_Reason\_Code (*\_Unwind\_Stop\_Fn)\\
   \ \ (int version,\\
   \ \ \ \_Unwind\_Action actions,\\
   \ \ \ uint64 exceptionClass,\\
   \ \ \ struct \_Unwind\_Exception *exceptionObject,\\
   \ \ \ struct \_Unwind\_Context *context,\\
   \ \ \ void *stop\_parameter );\\
\end{tabular}
}

\code{
\begin{tabular}{l}
   \_Unwind\_Reason\_Code\_Unwind\_ForcedUnwind\\
   \ \ ( struct \_Unwind\_Exception *exception\_object,\\
   \ \ \ \_Unwind\_Stop\_Fn stop,\\
   \ \ \ void *stop\_parameter );\\
\end{tabular}
}

Raise an exception for forced unwinding, passing along the given
exception object, which should have its \code{exception\_class} and
\code{exception\_cleanup} fields set. The exception object has been allocated
by the language-specific runtime, and has a language-specific format,
except that it must contain an \code{\_Unwind\_Exception} struct (see Exception
Header above).

Forced unwinding is a single-phase process (phase 2 of the normal
exception-handling process). The \code{stop} and
\code{stop\_parameter} parameters control the termination of the
unwind process, instead of the usual personality routine query. The
\code{stop} function parameter is called for each unwind frame, with
the parameters described for the usual personality routine below, plus
an additional \code{stop\_parameter}.

When the \code{stop} function identifies the destination frame, it
transfers control (according to its own, unspecified, conventions)
to the user code as appropriate without returning, normally after
calling \code{\_Unwind\_DeleteException}. If not, it should return an
\code{\_Unwind\_Reason\_Code} value as follows:

\begin{description}
\item[\code{\_URC\_NO\_REASON}]
     This is not the destination frame. The unwind
     runtime will call the frame's personality routine with the
     \code{\_UA\_FORCE\_UNWIND} and \code{\_UA\_CLEANUP\_PHASE} flags set in actions,
     and then unwind to the next frame and call the stop function again.

\item[\code{\_URC\_END\_OF\_STACK}] In order to allow \code{\_Unwind\_ForcedUnwind}
     to perform special processing when it reaches the end of the stack,
     the unwind runtime will call it after the last frame is rejected,
     with a \code{NULL} stack pointer in the context, and the stop function must
     catch this condition (i.e. by noticing the \code{NULL} stack pointer).
     It may return this reason code if it cannot handle end-of-stack.

\item[\code{\_URC\_FATAL\_PHASE2\_ERROR}] The stop function may return this code
     for other fatal conditions, e.g. stack corruption.
\end{description}

If the stop function returns any reason code other than \code{\_URC\_NO\_REASON},
the stack state is indeterminate from the point of view of the caller of
\code{\_Unwind\_ForcedUnwind}. Rather than attempt to return, therefore,
the unwind library should return \code{\_URC\_FATAL\_PHASE2\_ERROR} to its caller.

\paragraph{Example: \code{longjmp\_unwind()}\\}

The expected implementation of \code{longjmp\_unwind()} is as follows.
The \code{setjmp()} routine will have saved the state to be restored in its
customary place, including the frame pointer. The \code{longjmp\_unwind()}
routine will call \code{\_Unwind\_ForcedUnwind} with a stop function that
compares the frame pointer in the context record with the saved frame
pointer. If equal, it will restore the \code{setjmp()} state as customary,
and otherwise it will return \code{\_URC\_NO\_REASON} or \code{\_URC\_END\_OF\_STACK}.

If a future requirement for two-phase forced unwinding were identified,
an alternate routine could be defined to request it, and an actions
parameter flag defined to support it.

\subsubsection{\code{\_Unwind\_Resume}}

\code{
\begin{tabular}{l}
    void \_Unwind\_Resume\\
    \ \ (struct \_Unwind\_Exception *exception\_object);\\
\end{tabular}
}

Resume propagation of an existing exception e.g. after executing cleanup
code in a partially unwound stack. A call to this routine is inserted
at the end of a landing pad that performed cleanup, but did not resume
normal execution. It causes unwinding to proceed further.

\code{\_Unwind\_Resume} should not be used to implement rethrowing.
To the unwinding runtime, the catch code that rethrows was a handler,
and the previous unwinding session was terminated before entering it.
Rethrowing is implemented by calling \code{\_Unwind\_RaiseException}
again with the same exception object.

This is the only routine in the unwind library which is expected
to be called directly by generated code: it will be called at the
end of a landing pad in a "landing-pad" model.

\subsection{Exception Object Management}

\subsubsection{\code{\_Unwind\_DeleteException}}

\code{
\begin{tabular}{l}
    void \_Unwind\_DeleteException\\
    \ \ (struct \_Unwind\_Exception *exception\_object);\\
\end{tabular}
}

Deletes the given exception object. If a given runtime resumes normal
execution after catching a foreign exception, it will not know how to
delete that exception. Such an exception will be deleted by calling
\code{\_Unwind\_DeleteException}. This is a convenience function that calls
the function pointed to by the \code{exception\_cleanup} field of the exception
header.

\subsection{Context Management}

These functions are used for communicating information about the unwind
context (i.e. the unwind descriptors and the user register state) between
the unwind library and the personality routine and landing pad. They
include routines to read or set the context record images of registers in
the stack frame corresponding to a given unwind context, and to identify
the location of the current unwind descriptors and unwind frame.

\subsubsection{\code{\_Unwind\_GetGR}}

\code{
\begin{tabular}{l}
    uint64 \_Unwind\_GetGR\\
    \ \ (struct \_Unwind\_Context *context, int index);\\
\end{tabular}
}

This function returns the 64-bit value of the given general register.
The register is identified by its index as given in \ref{tbl-reg-num-map}.

During the two phases of unwinding, no registers have a guaranteed value.

\subsubsection{\code{\_Unwind\_SetGR}}

\code{
\begin{tabular}{l}
    void \_Unwind\_SetGR\\
    \ \ (struct \_Unwind\_Context *context,\\
    \ \ \ int index,\\
    \ \ \ uint64 new\_value);\\
\end{tabular}
}

This function sets the 64-bit value of the given register, identified by
its index as for \code{\_Unwind\_GetGR}.

The behavior is guaranteed only if the function is called during phase 2
of unwinding, and applied to an unwind context representing a handler frame,
for which the personality routine will return \code{\_URC\_INSTALL\_CONTEXT}.
In that case, only registers \RDI, \RSI, \RDX, \RCX should be used.
These scratch registers are reserved for passing arguments between the
personality routine and the landing pads.

\subsubsection{\code{\_Unwind\_GetIP}}

\code{
\begin{tabular}{l}
    uint64 \_Unwind\_GetIP\\
    \ \ (struct \_Unwind\_Context *context);\\
\end{tabular}
}

This function returns the 64-bit value of the instruction pointer (IP).

During unwinding, the value is guaranteed to be the address of the
instruction immediately following the call site in the function
identified by the unwind context. This value may be outside of the
procedure fragment for a function call that is known to not return
(such as \code{\_Unwind\_Resume}).

\subsubsection{\code{\_Unwind\_SetIP}}
\code{
\begin{tabular}{l}
    void \_Unwind\_SetIP\\
    \ \ (struct \_Unwind\_Context *context,\\
    \ \ \ uint64 new\_value);\\
\end{tabular}
}

This function sets the value of the instruction pointer (IP) for the
routine identified by the unwind context.

The behavior is guaranteed only when this function is called for an
unwind context representing a handler frame, for which the personality
routine will return \code{\_URC\_INSTALL\_CONTEXT}. In this case, control will
be transferred to the given address, which should be the address of a
landing pad.

\subsubsection{\code{\_Unwind\_GetLanguageSpecificData}}

\code{
\begin{tabular}{l}
    uint64 \_Unwind\_GetLanguageSpecificData\\
            (struct \_Unwind\_Context *context);\\
\end{tabular}
}

This routine returns the address of the language-specific data area for
the current stack frame.

This routine is not strictly required: it could be accessed through
\code{\_Unwind\_GetIP} using the documented format of the DWARF Call Frame
Information Tables, but since this work has been done for finding the
personality routine in the first place, it makes sense to cache the
result in the context.
We could also pass it as an argument to the personality routine.

\subsubsection{\code{\_Unwind\_GetRegionStart}}

\code{
\begin{tabular}{l}
    uint64 \_Unwind\_GetRegionStart\\
    \ \ (struct \_Unwind\_Context *context);\\
\end{tabular}
}

This routine returns the address of the beginning of the procedure or
code fragment described by the current unwind descriptor block.

This information is required to access any data stored relative to the
beginning of the procedure fragment. For instance, a call site table
might be stored relative to the beginning of the procedure fragment
that contains the calls. During unwinding, the function returns the
start of the procedure fragment containing the call site in the current
stack frame.

\subsection{Personality Routine}

\code{
\begin{tabular}{l}
    \_Unwind\_Reason\_Code (*\_\_personality\_routine)\\
    \ \ (int version,\\
    \ \ \ \_Unwind\_Action actions,\\
    \ \ \ uint64 exceptionClass,\\
    \ \ \ struct \_Unwind\_Exception *exceptionObject,\\
    \ \ \ struct \_Unwind\_Context *context);\\
\end{tabular}
}

The personality routine is the function in the C++ (or other language)
runtime library which serves as an interface between the system
unwind library and language-specific exception handling semantics.
It is specific to the code fragment described by an unwind info block,
and it is always referenced via the pointer in the unwind info block,
and hence it has no psABI-specified name.

\subsubsection{Parameters}

The personality routine parameters are as follows:

\begin{description}
\item[\code{version}] Version number of the unwinding runtime,
   used to detect a mis-match between the unwinder conventions and the
   personality routine, or to provide backward compatibility. For the
   conventions described in this document, version will be 1.
\item[\code{actions}] Indicates what processing the personality routine
   is expected to perform, as a bit mask. The possible actions are
   described below.
\item[\code{exceptionClass}] An 8-byte identifier specifying the type of the
   thrown exception. By convention, the high 4 bytes indicate the vendor
   (for instance AMD$\backslash$0), and the low 4 bytes indicate
   the language.  For the C++ ABI described in this document, the low
   four bytes are C++$\backslash$0.  This is not a null-terminated string.
   Some implementations may use no null bytes.
\item[\code{exceptionObject}] The pointer to a memory location recording the
   necessary information for processing the exception according to the
   semantics of a given language (see the Exception Header section above).
\item[\code{context}] Unwinder state information for use by the personality routine.
   This is an opaque handle used by the personality routine in particular
   to access the frame's registers (see the Unwind Context section above).
\item[return value] The return value from the personality routine indicates
   how further unwind should happen, as well as possible error conditions.
   See the following section.
\end{description}

\subsubsection{Personality Routine Actions}

The actions argument to the personality routine is a bitwise OR of one or
more of the following constants: \\
\code{
    typedef int \_Unwind\_Action;\\
    const \_Unwind\_Action \_UA\_SEARCH\_PHASE = 1;\\
    const \_Unwind\_Action \_UA\_CLEANUP\_PHASE = 2;\\
    const \_Unwind\_Action \_UA\_HANDLER\_FRAME = 4;\\
    const \_Unwind\_Action \_UA\_FORCE\_UNWIND = 8;\\
}

\begin{description}
\item[\code{\_UA\_SEARCH\_PHASE}] Indicates that the personality routine should
   check if the current frame contains a handler, and if so return
   \code{\_URC\_HANDLER\_FOUND}, or otherwise return \code{\_URC\_CONTINUE\_UNWIND}.
   \code{\_UA\_SEARCH\_PHASE} cannot be set at the same time as \code{\_UA\_CLEANUP\_PHASE}.

\item[\code{\_UA\_CLEANUP\_PHASE}] Indicates that the personality routine should
   perform cleanup for the current frame. The personality routine can perform
   this cleanup itself, by calling nested procedures, and return
   \code{\_URC\_CONTINUE\_UNWIND}. Alternatively, it can setup the registers
   (including the IP) for transferring control to a "landing pad", and
   return \code{\_URC\_INSTALL\_CONTEXT}.

\item[\code{\_UA\_HANDLER\_FRAME}]
   During phase 2, indicates to the personality routine that the current
   frame is the one which was flagged as the handler frame during phase 1.
   The personality routine is not allowed to change its mind between phase 1
   and phase 2, i.e. it must handle the exception in this frame in phase 2.

\item[\code{\_UA\_FORCE\_UNWIND}] During phase 2, indicates that no language is
   allowed to "catch" the exception. This flag is set while unwinding the
   stack for \code{longjmp} or during thread cancellation. User-defined code in a
   catch clause may still be executed, but the catch clause must resume
   unwinding with a call to \code{\_Unwind\_Resume} when finished.
\end{description}

\subsubsection{Transferring Control to a Landing Pad}

If the personality routine determines that it should transfer control to a
landing pad (in phase 2), it may set up registers (including IP) with
suitable values for entering the landing pad (e.g. with landing pad
parameters), by calling the context management routines above. It then
returns \code{\_URC\_INSTALL\_CONTEXT}.

Prior to executing code in the landing pad, the unwind library restores
registers not altered by the personality routine, using the context
record, to their state in that frame before the call that threw the exception,
as follows. All registers specified as callee-saved by the base ABI are
restored, as well as scratch registers \RDI, \RSI, \RDX, \RCX (see below).
Except for those exceptions, scratch (or caller-saved) registers are not
preserved, and their contents are undefined on transfer.

The landing pad can either resume normal execution (as, for instance, at
the end of a C++ catch), or resume unwinding by calling \code{\_Unwind\_Resume} and
passing it the \code{exceptionObject} argument received by the personality routine.
\code{\_Unwind\_Resume} will never return.

\code{\_Unwind\_Resume} should be called if and only if the personality routine
did not return \code{\_Unwind\_HANDLER\_FOUND} during phase 1.  As a result,
the unwinder can allocate resources (for instance memory) and keep track
of them in the exception object reserved words. It should then free these
resources before transferring control to the last (handler) landing pad.
It does not need to free the resources before entering non-handler
landing-pads, since \code{\_Unwind\_Resume} will ultimately be called.

The landing pad may receive arguments from the runtime, typically passed
in registers set using \code{\_Unwind\_SetGR} by the personality routine.
For a landing pad that can call to \code{\_Unwind\_Resume}, one argument must
be the \code{exceptionObject} pointer, which must be preserved to be passed to
\code{\_Unwind\_Resume}.

The landing pad may receive other arguments, for instance a switch value
indicating the type of the exception. Four scratch registers are reserved
for this use (\RDI, \RSI, \RDX, \RCX).

\subsubsection{Rules for Correct Inter-Language Operation}

The following rules must be observed for correct operation between
languages and/or runtimes from different vendors:

An exception which has an unknown class must not be altered by the
personality routine. The semantics of foreign exception processing
depend on the language of the stack frame being unwound. This covers
in particular how exceptions from a foreign language are mapped to
the native language in that frame.

If a runtime resumes normal execution, and the caught exception was
created by another runtime, it should call \code{\_Unwind\_DeleteException}.
This is true even if it understands the exception object format
(such as would be the case between different C++ runtimes).

A runtime is not allowed to catch an exception if the
\code{\_UA\_FORCE\_UNWIND} flag was passed to the personality routine.

\paragraph{Example: Foreign Exceptions in C++.} In C++, foreign exceptions can be
caught by a \code{catch(\dots)} statement. They can also be caught as if they
were of a \code{\_\_foreign\_exception} class, defined in \code{<exception>}. The
\code{\_\_foreign\_exception} may have subclasses, such as
\code{\_\_java\_exception} and \code{\_\_ada\_exception}, if the runtime is capable
of identifying some of the foreign languages.

The behavior is undefined in the following cases:
\begin{itemize}
\item A \code{\_\_foreign\_exception} catch argument is accessed in any way
     (including taking its address).

\item A \code{\_\_foreign\_exception} is active at the same time as another
     exception (either there is a nested exception while catching the
     foreign exception, or the foreign exception was itself nested).

\item \code{uncaught\_exception()}, \code{set\_terminate()},
     \code{set\_unexpected()}, \code{terminate()}, or
     \code{unexpected()} is called at a time a foreign exception
     exists (for example, calling \code{set\_terminate()} during
     unwinding of a foreign exception).
\end{itemize}

All these cases might involve accessing C++ specific content of the
thrown exception, for instance to chain active exceptions.

Otherwise, a catch block catching a foreign exception is allowed:
\begin{itemize}
\item to resume normal execution, thereby stopping propagation of
      the foreign exception and deleting it, or
\item to rethrow the foreign exception. In that case, the original
      exception object must be unaltered by the C++ runtime.
\end{itemize}

A catch-all block may be executed during forced unwinding.  For
instance, a longjmp may execute code in a \code{catch(\dots)} during
stack unwinding. However, if this happens, unwinding will proceed at
the end of the catch-all block, whether or not there is an explicit
rethrow.

Setting the low 4 bytes of exception class to C++$\backslash$0 is reserved
for use by C++ runtimes compatible with the common C++ ABI.

%%% Local Variables:
%%% mode: latex
%%% TeX-master: "abi"
%%% End:


\chapter{Development Environment}

During compilation of C or C++ code at least the symbols in
table \ref{prepro_defines} are defined by the pre-processor.

\begin{table}[H]
\Hrule
\caption{Predefined pre-processor symbols}
\label{prepro_defines}
  \begin{center}\code{
    \begin{tabular}[t]{l}
      __amd64\\
      __amd64__\\
      __x86_64\\
      __x86_64__\\
    \end{tabular}
  }\end{center}
\Hrule
\end{table}

%%% Local Variables: 
%%% mode: latex
%%% TeX-master: "abi"
%%% End: 


\chapter{Execution Environment}


%%% Local Variables: 
%%% mode: latex
%%% TeX-master: "abi"
%%% End: 


%%% Bah, we can't nest includes, hence begin the chapter outside
%%% of the included file, so that conventions and fortran come to lie
%%% in the same chapter
\chapter{Conventions}\editornote{This chapter is used to document some features special to
  the \xARCH ABI.  The different sections might be moved to another
  place or removed completely.}
\section{C++\label{section-cpp}}

For the \textindex{C++} ABI we will use the IA-64 C++ ABI and instantiate it
appropriately.  The current draft of that ABI is available at:\\
\url{http://mentorembedded.github.io/cxx-abi/}

%%% We would like to include the fortran text here, but we can't nest
%%% includes.  Hence, do it in the main file.
%%% \section{Fortran}

A formal Fortran ABI does not exist.  Most Fortran compilers are
designed for very specific high performance computing applications, so
Fortran compilers use different passing conventions and memory layouts
optimized for their specific purpose.  For example, Fortran
applications that must run on distributed memory machines need a
different data representation for array descriptors (also known as
dope vectors, or fat pointers) than applications running on symmetric
multiprocessor shared memory machines.  A normative ABI for Fortran is
therefore not desirable.  However, for interoperability of different
Fortran compilers, as well as for interoperability with other
languages, this section provides some some guidelines for data types
representation, and argument passing.  The guidelines in this section
are derived from the GNU Fortran 77 (G77) compiler, and are also
followed by the GNU Fortran 95 (\code{gfortran}) compiler (restricted to
Fortran 77 features).  Other Fortran
compilers already available for AMD64 at the time of this writing may
use different conventions, so compatibility is not guaranteed.

When this text uses the term {\em Fortran procedure}, the text applies
to both Fortran \code{FUNCTION} and \code{SUBROUTINE} subprograms as
well as for alternate \code{ENTRY} points,
unless specifically stated otherwise.

Everything not explicitely defined in this ABI is left to the implementation.

\subsection{Names}
\label{sub_fortran_names}

External names in Fortran are names of entities visible to all
subprograms at link time.  This includes names of \code{COMMON} blocks
and Fortran procedures.
To avoid name space conflicts with linked-in libraries, all external
names have to be mangled.  And to avoid name space conflicts
of mangled external names with local names, all local names must also
be mangled.
The mangling scheme is straightforward as follows:
\begin{itemize}
\item all names that do not have any underscores in it should have
{\em one} underscore appended
\item all external names containing one or
   more underscores in it (whereever) should have {\em two} underscores
   appended \footnote{Historically, this is to be compatible with f2c.}.
\item all external names should be mapped to lower case,
   following the traditional UNIX model for Fortran compilers
\end{itemize}

For examples see figure~\ref{fortran-external-names}:

\begin{figure}[H]
\Hrule
\caption{Example mapping of names} \label{fortran-external-names}
\begin{center}
\begin{footnotesize}
\begin{tabular}{l|l}
\multicolumn{1}{c}{Fortran external name}&\multicolumn{1}{c}{Linker name}\\
\hline
\code{FOO}    & \code{foo_} \\
\code{foo}    & \code{foo_} \\
\code{Foo}    & \code{foo_} \\
\code{foo_}   & \code{foo___} \\
\code{f_oo}   & \code{f_oo__} \\
\end{tabular}
\end{footnotesize}
\end{center}
\Hrule
\end{figure}

The entry point of the main program unit is called \code{MAIN__}.
The symbol name for the blank common block is \code{__BLNK__}.
the external name of the unnamed \code{BLOCK DATA} routine is
\code{__BLOCK_DATA__}.

\subsection{Representation of Fortran Types}
\label{sub_fortran_types}

For historical reasons, GNU Fortran 77 maps Fortran programs to the C
ABI, so the data representation can be explained best by providing the
mapping of Fortran types to C types used by G77 on AMD64\footnote{G77
  provides a header \code{g2c.h} with the equivalent C type
  definitions for all supported Fortran scalar types.} as in figure~
\ref{fortran-c-types}.  The ``\code{TYPE*N}'' notation specifies that
variables or aggregate members of type \code{TYPE} shall occupy
\code{N} bytes of storage.


\begin{figure}[H]
\Hrule
\caption{Mapping of Fortran to C types} \label{fortran-c-types}
\begin{center}
\begin{footnotesize}
\begin{tabular}{l|l|l}
\multicolumn{1}{c}{Fortran}&\multicolumn{1}{c}{Data
  kind}&\multicolumn{1}{c}{Equivalent C type}\\
\hline
\code{INTEGER*4}      &  Default integer                       &  \code{signed int} \\
\code{INTEGER*8}      &  Double precision integer              &  \code{signed long}\\
\code{REAL*4}         &  Single precision FP number            &  \code{float}\\
\code{REAL*8}         &  Double precision FP number            &  \code{double}\\
\code{COMPLEX*4}      &  Single precision complex FP number    &  \code{complex float}\\
\code{COMPLEX*8}      &  Double precision complex FP number    &  \code{complex double}\\
\code{LOGICAL}        &  Boolean logical type                  &  \code{signed int}\\
\code{CHARACTER}      &  Text string                           &  \code{char[] + length}\\
\end{tabular}
\end{footnotesize}
\end{center}
\Hrule
\end{figure}

The values for type \code{LOGICAL} are \code{.TRUE.} implemented as 1 and
\code{.FALSE.} implemented as 0.

Data objects with a \code{CHARACTER} type\footnote{This includes sub-strings.}
are represented as an array
of characters of the C char type (not guaranteed to be
``\code{$\backslash$0}'' terminated) with a separate length counter to
distinguish between \code{CHARACTER} data objects with a
length parameter, and aggregate types of \code{CHARACTER} data
objects, possibly also with a length parameter.

Layout of other aggregate types is implementation defined.  GNU
Fortran puts all arrays in contiguous memory in column-major order.
GNU Fortran 95 builds an equivalent C struct for derived types without
reordering the type fields.  Other compilers may use other
representations as needed.  The representation and use of Fortran
90/95 array descriptors is implementation defined.  Note that array indices
start at 1 by default.

Fortran 90/95 allow different kinds of each basic type using the \code{kind}
type parameter of a type.  Kind type parameter values are
implementation defined.

Layout of he commonly used Cray pointers is implementation defined.

\subsection{Argument Passing}

For each given Fortran 77 procedure, an equivalent C prototype can
be derived. Once this equivalent C prototype is known, the C ABI
conventions should be applied to determine how arguments are passed
to the Fortran procedure.

G77 passes all (user defined) formal arguments of a procedure by
reference.  Specifically, pointers to the location in memory of a
variable, array, array element, a temporary location that holds the
result of evaluating an expression or a temporary or permanent
location that holds the value of a constant (xf. g77 manual) are
passed as actual arguments. Artificial compiler generated arguments
may be passed by value or by reference as they are inherently
compiler and hence implementation specific.

Data objects with a \code{CHARACTER} type are passed as a pointer to
the character string and its length, so that each \code{CHARACTER}
formal argument in a Fortran procedure results in two actual arguments
in the equivalent C prototype.  The first argument occupies the
position in the formal argument list of the Fortran procedure.  This
argument is a pointer to the array of characters that make up the
string, passed by the caller.  The second argument is appended to the
end of the user-specified formal argument list.  This argument is of
the default integer type and its value is the length of the array of
characters, that is the length, passed as the first argument.  This
length is passed by value.
When more than one \code{CHARACTER} argument is present in an argument
list, the length arguments are appended in the order the original
arguments appear.  The above discussion also applies to sub-strings.

This ABI does not define the passing of optional arguments.  They are
allowed only in Fortran 90/95 and their passing is implementation defined.

This ABI does not define array functions (function returning arrays).
They are allowed only in Fortran 90/95 and requires the definition of
array descriptors.

Note that Fortran 90/95 procedure arguments with the \code{INTENT(IN)}
attribute should also passed by reference if the procedure is to be
linked with code written in Fortran 77.  Fortran 77 does not and can
not support the \code{INTENT} attribute because it has no concept of
explicit interfaces.  It is therefore not possible to declare the
callee's arguments as \code{INTENT(IN)}.  A Fortran 77 compiler must
assume that all procedure arguments are \code{INTENT(INOUT)} in the
Fortran 90/95 sense.


\subsection{Functions}

The calling of statement functions is implementation defined (as they
are defined only locally, the compiler has the freedom to apply any
calling convention it likes).

Subroutines with alternate returns (e.g. "SUBROUTINE X(*,*)" called
as "CALL X(*10,*20)") are implemented as functions returning
an \code{INTEGER} of the default kind.  The value of this returned integer
is whatever integer is specified in the "RETURN" statement for
the subroutine \footnote{
This integer indicates the position of an alternate
return from the subroutine in the formal argument list}, or 0 for
a \code{RETURN} statement without an argument.  It is up to the caller
to jump to the corresponding alternate return label.  The actual
alternate-return arguments are omitted from the calling sequence.

An example:

\begin{footnotesize}
\begin{verbatim}
       SUBROUTINE SHOW_ALTERNATE_RETURN (N)
          INTEGER N
          CALL ALTERNATE_RETURN_EXAMPLE (N, *10, *20, *30)
          WRITE (*,*) 'OK - Normal Return'
          RETURN
10        WRITE (*,*) '1st alternate return'
          RETURN
20        WRITE (*,*) '2nd alternate return'
          RETURN
30        WRITE (*,*) '2nd alternate return'
          RETURN
       END

       SUBROUTINE ALTERNATE_RETURN_EXAMPLE (N, *, *, *)
          INTEGER N
          IF (N .EQ. 0 ) RETURN     ! Implicit "RETURN 0"
          IF ( N .EQ. 1 ) RETURN 1
          IF ( N .EQ. 2 ) RETURN 2
          RETURN 3
       END
\end{verbatim}
\end{footnotesize}

Here the \code{SUBROUTINE ALTERNATE_RETURN_EXAMPLE} is implemented as
a function returning an \code{INTEGER*4} with value 0 if N is 0, 1 if N is 1,
2 if N is 2 and 3 for all other values of N.  This return value is used by
the caller as if the actual call were replaced by this sequence:
\begin{footnotesize}
\begin{verbatim}
          INTEGER X
          X = CALL ALTERNATE_RETURN_EXAMPLE (N)
          GOTO (10, 20, 30), X
\end{verbatim}
\end{footnotesize}

All in all the effect is that the index of the returned to label (starting
from 1) will be contained in \RAX after the call.

Alternate \code{ENTRY} points of a \code{SUBROUTINE} or \code{FUNCTION}
should be treated as separate subprograms, as mandated by the Fortran
standard.  I.e. arguments passed to an alternate \code{ENTRY} should be passed
as if the alternate \code{ENTRY} is a separate \code{SUBROUTINE} or
\code{FUNCTION}.  If a \code{FUNCTION} has alternate \code{ENTRY} points,
the result of each of the alternate \code{ENTRY} points must be returned
as if the alternate \code{ENTRY} is a separate \code{FUNCTION} with the
result type of the alternate \code{ENTRY}.  The external naming of alternate
\code{ENTRY} points follows section~\ref{sub_fortran_names}.

\subsection{COMMON blocks}
In absence of any \code{EQUIVALENCE} declaration involving variables
in \code{COMMON} blocks the layout of a 
\code{COMMON} block is exactly the same as the layout of the equivalent C
structure (with types of variables substituted according to section~
\ref{sub_fortran_types}), including the alignment requirements.

This ABI defines the layout under presence of \code{EQUIVALENCE} statements only
in some cases:
\begin{itemize}
\item the layout of the \code{COMMON} block must not change if one ignores
      the \code{EQUIVALENCE}, which amongst other things means:
\item If two arrays are equivalenced, the larger array must be named in
      the \code{COMMON} block, and there must be complete inclusion,
      in particular the other array may not extend the size of the 
      equivalenced segment.  It may also not change the alignment
      requirement.
\item If an array element and a scalar are equivalenced, the array must be
      named in the \code{COMMON} block and it must not be smaller than
      the scalar.  The type of the scalar must not require bigger alignment
      than the array.
\item if two scalars are equivalenced they must have the same size and
      alignment requirements.
\end{itemize}

Other cases are implementation defined.

Because the Fortran standard allows the blank \code{COMMON} block to have
different sizes in different subprograms, it may be impossible to determine
if it is small enough to fit in the \texttt{.bss}
section.  When compiling for the medium or large code models the blank
\code{COMMON} block should therefore always be put in the \texttt{.lbss}
section.

\subsection{Intrinsics}
This sections lists the set of intrinsics which has to be supported
at minimum by a conforming compiler.  They are separated by origin.
They follow regular calling and naming conventions.

The signature of intrinsics uses the syntax
$return-type(argtype1, argtype2, ...)$, where the individual types can
be the following characters: {\bf V} (as in void) designates a
\code{SUBROUTINE}, {\bf L} a \code{LOGICAL}, {\bf I} an \code{INTEGER},
{\bf R} a \code{REAL}, and {\bf C} a \code{CHARACTER}.
Hence \code{I(R,L)} designates a \code{FUNCTION} returning an
\code{INTEGER} and taking a \code{REAL} and a \code{LOGICAL}.
If an argument is an array, this is indicated using a trailing number,
e.g. {\bf I13} is an \code{INTEGER} array with 13 elements.
If a \code{CHARACTER} argument or return value has a fixed length, this
is indicated using an asterisk and a trailing number, for example
{\bf C*16} is a \code{CHARACTER(len=16)}.  If a \code{CHARACTER}
argument of arbitrary length must be passed, the trailing number is
replaced with \code{N}, for example {\bf C*N}.


\begin{table}[H]
\Hrule
  \caption{Mil intrinsics}
  \label{intrinsics-mil}
  \begin{center}
  \begin{tabular}[t]{l|l|l}
    \multicolumn{1}{c}{Name} & \multicolumn{1}{c}{Signature} & \multicolumn{1}{c}{Meaning} \\
    \hline
    \code{BTest}  & L(I,I)          & Test bit \\
    \code{IAnd}   & I(I,I)          & Boolean AND \\
    \code{IOr}    & I(I,I)          & Boolean OR \\
    \code{IEOr}   & I(I,I)          & Boolean XOR \\
    \code{Not}    & I(I)            & Boolean NOT \\
    \code{IBClr}  & I(I,I)          & Clear a bit \\
    \code{IBits}  & I(I,I,I)        & Extract a bit subfield of a variable \\
    \code{IBSet}  & I(I,I)          & Set a bit \\
    \code{IShft}  & I(I,I)          & Logical bit shift \\
    \code{IShftC} & I(I,I,I)        & Circular bit shift \\
    \code{MvBits} & V(I,I,I,I,I)    & Move a bit field \\
  \end{tabular}
  \end{center}
\Hrule
\end{table}

\begin{itemize}
  \item[\tt BTest]{\tt (I, Pos)}
    Returns \code{.TRUE.} if bit \code{Pos} in \code{I} is set,
    returns \code{.FALSE.} otherwise.
  \item[\tt IAnd]{\tt (I, J)}
    Returns value resulting from a boolean AND on each pair of bits
    in \code{I} and \code{J}.
  \item[\tt IOr]{\tt (I, J)}
    Returns value resulting from a boolean OR on each pair of bits
    in \code{I} and \code{J}.
  \item[\tt IEOr]{\tt (I, J)}
    Returns value resulting from a boolean XOR on each pair of bits
    in \code{I} and \code{J}.
  \item[\tt Not]{\tt (I)}
    Returns value resulting from a boolean NOT on each bit in \code{I}.
  \item[\tt IBClr]{\tt (I, Pos)}
    Returns the value of \code{I} with bit \code{Pos} cleared (set to zero).
  \item[\tt IBits]{\tt (I, Pos, Len)}
    Extracts a subfield starting from bit position \code{Pos} and with a
    length (towards the most significant bit) of \code{Len} bits from \code{I}.
    The result is right-justified and the remaining bits are zeroed.
  \item[\tt IBSet]{\tt (I, Pos)}
    Returns the value of \code{I} with the bit in position \code{Pos} set to one.
  \item[\tt IShft]{\tt (I, Shift)}
    All bits of \code{I} are shifted \code{Shift} places.
    \code{Shift.GT.0} indicates a left shift,
    \code{Shift.EQ.0} indicates no shift, and
    \code{Shift.LT.0} indicates a right shift.
    Bits shifted out from the least (when shifting right) or
    most (when shifting left) significant position are lost.
    Bits shifted in at the opposite end are not set (i.e. zero).
  \item[\tt IShftC]{\tt (I, Shift, Size)}
    The rightmost \code{Size} bits of the argument \code{I} are shifted
    circularly \code{Shift} places.
    The unshifted bits of the result are the same as the unshifted bits of I.
  \item[\tt MvBits]{\tt (From, FromPos, Len, To, ToPos)}
    Move \code{Len} bits of \code{From} from bit positions \code{FromPos}
    through \code{FromPos+Len-1} to bit positions \code{ToPos} through
    \code{ToPos+Len-1} of \code{To}. The bit portions of \code{To} that
    are not affected by the movement of bits are unchanged.
\end{itemize}

\begin{table}[H]
\Hrule
  \caption{F77 intrinsics}
  \label{intrinsics-f77}
  \begin{center}
  \begin{tabular}[t]{l|l}
    \multicolumn{1}{c}{Name} & \multicolumn{1}{c}{Meaning} \\
    \hline
    Abs & Absolute value \\
    ACos & Arc cosine \\
    AInt & Truncate to whole number \\
    ANInt & Round to nearest whole number \\
    ASin & Arc sine \\
    ATan & Arc Tangent \\
    ATan2 & Arc Tangent \\
    Char & Character from code \\
    Cmplx & Construct \code{COMPLEX(KIND=1)} value \\
    Conjg & Complex conjugate \\
    Cos & Cosine \\
    CosH & Hyperbolic cosine \\
    Dble & Convert to double precision \\
    DiM & Difference magnitude (non-negative subtract) \\
    DProd & Double-precision product \\
    Exp & Exponential \\
    IChar & Code for character \\
    Index & Locate a \code{CHARACTER} substring \\
    Int & Convert to \code{INTEGER} value truncated to whole number \\
    Len & Length of character entity \\
    LGe & Lexically greater than or equal \\
    LGt & Lexically greater than \\
    LLe & Lexically less than or equal \\
    LLt & Lexically less than \\
    Log & Natural logarithm \\
    Log10 & Common logarithm \\
    Max & Maximum value \\
    Min & Minimum value \\
    Mod & Remainder \\
    NInt & Convert to \code{INTEGER} value rounded to nearest whole number \\
    Real & Convert value to type \code{REAL(KIND=1)} \\
    Sin & Sine \\
    SinH & Hyperbolic sine \\
    SqRt & Square root \\
    Tan & Tangent \\
    TanH & Hyperbolic tangent \\
  \end{tabular}
  \end{center}
\Hrule
\end{table}

Refer to the Fortran 77 language standard for signature and definition
of the F77 intrinsics listed in table~\ref{intrinsics-f77}.  These intrinsics
can have a prefix as per the standard hence the table is not exhaustive.

\begin{table}[H]
\Hrule
  \caption{F90 intrinsics}
  \label{intrinsics-f90}
  \begin{center}
  \begin{tabular}[t]{l|l}
    \multicolumn{1}{c}{Name} & \multicolumn{1}{c}{Meaning} \\
    \hline
    AChar & ASCII character from code \\
    Bit_Size & Number of bits in arguments type \\
    CPU_Time & Get current CPU time \\
    IAChar & ASCII code for character \\
    Len_Trim & Get last non-blank character in string \\
    System_Clock & Get current system clock value \\
  \end{tabular}
  \end{center}
\Hrule
\end{table}

Refer to the Fortran 90 language standard for signature and definition
of the F90 intrinsics listed in table~\ref{intrinsics-f90}.

\begin{table}[H]
\Hrule
  \caption{Math intrinsics}
  \label{intrinsics-math}
  \begin{center}
  \begin{tabular}[t]{l|l|l}
    \multicolumn{1}{c}{Name} & \multicolumn{1}{c}{Signature} & \multicolumn{1}{c}{Meaning} \\
    \hline
    \code{BesJ0} & R(R)   & Bessel function \\
    \code{BesJ1} & R(R)   & Bessel function \\
    \code{BesJN} & R(I,R) & Bessel function \\
    \code{BesY0} & R(R)   & Bessel function \\
    \code{BesY1} & R(R)   & Bessel function \\
    \code{BesYN} & R(I,R) & Bessel function \\
    \code{ErF}   & R(R)   & Error function \\
    \code{ErFC}  & R(R)   & Complementary error function \\
    \code{IRand} & I(I)   & Random number \\
    \code{Rand}  & R(I)   & Random number \\
    \code{SRand} & V(I)   & Random seed \\
  \end{tabular}
  \end{center}
\Hrule
\end{table}

\begin{itemize}
  \item[\tt BesJ0]{\tt (X)}
    Calculates the Bessel function of the first kind of order 0 of X.
    Returns a \code{REAL} of the same kind as \code{X}.
  \item[\tt BesJ1]{\tt (X)}
    Calculates the Bessel function of the first kind of order 1 of X.
    Returns a \code{REAL} of the same kind as \code{X}.
  \item[\tt BesJN]{\tt (N, X)}
    Calculates the Bessel function of the first kind of order N of X.
    Returns a \code{REAL} of the same kind as \code{X}.
  \item[\tt BesY0]{\tt (X)}
    Calculates the Bessel function of the second kind of order 0 of X.
    Returns a \code{REAL} of the same kind as \code{X}.
  \item[\tt BesY1]{\tt (X)}
    Calculates the Bessel function of the second kind of order 1 of X.
    Returns a \code{REAL} of the same kind as \code{X}.
  \item[\tt BesYN]{\tt (N, X)}
    Calculates the Bessel function of the second kind of order N of X.
    Returns a \code{REAL} of the same kind as \code{X}.
  \item[\tt ErF]{\tt (X)}
    Calculates the error function of X.
    Returns a \code{REAL} of the same kind as \code{X}.
  \item[\tt ErFC]{\tt (X)}
    Calculates the complementary error function of X, i.e. \code{1 - ERF(X)}.
    Returns a \code{REAL} of the same kind as \code{X}.
  \item[\tt IRand]{\tt (Flag)}
    Flag is optional.
    Returns a uniform quasi-random number up to a system-dependent limit.
    If \code{Flag .EQ. 0} or \code{Flag} is not passed, the next number
    in sequence is returned.
    If \code{Flag .EQ. 1}, the generator is restarted.
    If \code{Flag} has any other value, the generator is restarted with
    the value of \code{Flag} as the new seed.
  \item[\tt Rand]{\tt (Flag)}
    Flag is optional.
    Returns a uniform quasi-random number between 0 and 1.
    If \code{Flag .EQ. 0} or \code{Flag} is not passed, the next number
    in sequence is returned.
    If \code{Flag .EQ. 1}, the generator is restarted.
    If \code{Flag} has any other value, the generator is restarted with
    the value of \code{Flag} as the new seed.
  \item[\tt SRand]{\tt (Seed)}
    Reinitializes the random number generator for \code{IRand} and \code{Rand}
    with the seed in \code{Seed}.
\end{itemize}

\begin{table}[H]
\Hrule
  \caption{Unix intrinsics}
  \label{intrinsics-unix}
  \begin{center}
  \begin{tabular}[t]{l|l|l}
    \multicolumn{1}{c}{Name} & \multicolumn{1}{c}{Signature} & \multicolumn{1}{c}{Meaning} \\
    \hline
    \code{Abort}  & V()               & Abort the program \\
    \code{Access} & I(C,C)            & Check file accessibility \\
    \code{DTime}  & V(R2,R)           & Get elapsed time since last call \\
    \code{ETime}  & V(R2,R)           & Get elapsed time for process \\
    \code{Flush}  & V(I)              & Flush buffered output \\
    \code{FNum}   & I(I)              & Get file descriptor from Fortran unit number \\
    \code{FStat}  & V(I,I13,I)        & Get file information \\
    \code{GError} & V(C*N)            & Get error message for last error \\
    \code{GetArg} & V(I,C*N)          & Obtain command-line argument \\
    \code{GetCWD} & V(C*N,I)          & Get current working directory \\
    \code{GetEnv} & V(C*N,C*N)        & Get environment variable \\
    \code{GetGId} & I()               & Get process group ID \\
    \code{GetPId} & I()               & Get process ID \\
    \code{GetUId} & I()               & Get process user ID \\
    \code{GetLog} & V(C*N)            & Get login name \\
    \code{HostNm} & V(C*N,I)          & Get host name \\
    \code{IArgC}  & I()               & Obtain count of command-line arguments \\
    \code{IDate}  & V(I3)             & Get local date info \\
    \code{IErrNo} & I()               & Get error number for last error \\
    \code{ITime}  & V(I3)             & Get local time of day \\
    \code{LStat}  & V(C*N,I13,I)      & Get file information \\
    \code{PError} & V(C*N)            & Print error message for last error \\
    \code{Rename} & V(C*N,C*N,I)      & Rename file \\
    \code{Sleep}  & V(I)              & Sleep for a specified time \\
    \code{System} & V(C*N,I)          & Invoke shell (system) command \\
  \end{tabular}
  \end{center}
\Hrule
\end{table}


\begin{itemize}
  \item[\tt Abort]{\tt ()}
    Prints a message and potentially causes a core dump.
  \item[\tt Access]{\tt (Name, Mode)}
    Checks file \code{Name} for accessibility in the mode specified by
    \code{Mode}.  Returns $0$ if the file is accessible in that mode,
    otherwise an error code.  \code{Name} must be a \code{NULL}-terminated
    string of \code{CHARACTER} (i.e. a C-style string).  Trailing blanks
    in \code{Name} are ignored.
    \code{Mode} must be a concatenation of any of the following characters:
    {\bf r} meaning test for read permission, {\bf w} meaning test for write
    permission, {\bf x} meaning test for execute/search permission, or
    a space meaning test for existence of the file.
  \item[\tt DTime]{\tt (TArray, Result)}
    When called for the first time, returns the number of seconds of runtime
    since the start of the program in \code{Result}, the user component of
    this runtime in \code{TArray(1)}, and the system time in \code{TArray(2)}.
    Subsequent invocations values based on accumulations since the previous
    invocation.
  \item[\tt ETime]{\tt (TArray, Result)}
    Returns the number of seconds of runtime since the start of the program
    in \code{Result}, the user component of
    this runtime in \code{TArray(1)}, and the system time in \code{TArray(2)}.
    Subsequent invocations values based on accumulations since the previous
    invocation.
  \item[\tt Flush]{\tt (Unit)}
    Flushes the Fortran I/O unit with ID \code{Unit}.  The unit must be open
    for output. If the optional \code{Unit} argument is omitted, all open
    units are flushed.
  \item[\tt FNum]{\tt (Unit)}
    Returns the UNIX(tm) file descriptor number corresponding to the Fortran
    I/O unit \code{Unit}.  The unit must be open.
  \item[\tt FStat]{\tt (Unit, SArray, Status)}
    Obtains data about the file open on Fortran I/O unit \code{Unit} and
    places it in the array \code{SArray}. The values in this array are
    as follows:
    \begin{enumerate}
      \item Device ID 
      \item Inode number 
      \item File mode 
      \item Number of links 
      \item Owner's UID 
      \item Owner's GID 
      \item ID of device containing directory entry for file
      \item File size (bytes) 
      \item Last access time 
      \item Last modification time 
      \item Last file status change time 
      \item Preferred I/O block size (-1 if not available) 
      \item Number of blocks allocated (-1 if not available)
    \end{enumerate}
    If an element is not available, or not relevant on the host system,
    it is returned as 0 except when indicated otherwise in the above list.
    If the optional \code{Status} argument is supplied, it contains 0 on
    success or a nonzero error code upon return.
  \item[\tt Gerror]{\tt (Message)}
    Returns the system error message corresponding to the last system error
    (errno in C). The message is returned in \code{Message}.
    If \code{Message} is longer than the error message, it is padded with
    blanks after the message.  If \code{Message} is not long enough to hold
    the error message, the error message is truncated to the length of
    \code{Message}.
  \item[\tt GetArg]{\tt (Pos, Value)}
    Returns in \code{Value} the command-line argument in position \code{Pos}.
    If there are fever than \code{Pos} command-line arguments, \code{Value}
    is filled with blanks.  If \code{Pos} is 0, the name of the program is
    returned.
    If \code{Value} is longer than the command-line argument, it is padded
    with blanks after the argument.  If \code{Value} is not long enough to
    hold the command-line argument, the argument is truncated to the length
    of \code{Value}.
  \item[\tt GetCWD]{\tt (Name, Status)}
    Returns in \code{Name} the current working directory. If the optional
    \code{Status} argument is supplied, it contains 0 on success or a
    nonzero error code upon return.
  \item[\tt GetEnv]{\tt (Name, Value)}
    Returns in \code{Value} the environment variable identified with
    \code{Name}.  If \code{Name} has not been set, \code{Value} is filled
    with blanks. A \code{null} character marks the end of the name in
    \code{Name}. Trailing blanks in \code{Name} are ignored.
    If \code{Value} is longer than the environment variable, it is padded
    with blanks after the variable.  If \code{Value} is not long enough to
    hold the environment variable, the variable is truncated to the length
    of \code{Value}.
  \item[\tt GetGId]{\tt ()}
    Returns the group ID for the current process.
  \item[\tt GetPId]{\tt ()}
    Returns the process ID for the current process.
  \item[\tt GetUId]{\tt ()}
    Returns the user ID for the current process.
  \item[\tt GetLog]{\tt (Login)}
    Returns the login name for the process in \code{Login}, or a blank
    string if the host system does not support \code{getlogin(3)}.
    If \code{Login} is longer than the login name, it is padded
    with blanks after the login name.  If \code{Login} is not long enough
    to hold the login name, the login name is truncated to the length of
    of \code{Login}.
  \item[\tt HotNm]{\tt (Name, Status)}
    Returns in \code{Name} system's host name. If the optional \code{Status}
    argument is supplied, it contains 0 on success or a nonzero error code
    upon return.
    If \code{Name} is longer than the host name, it is padded
    with blanks after the host name.  If \code{Name} is not long enough
    to hold the host name, the host name is truncated to the length of
    of \code{Name}.
  \item[\tt IArgC]{\tt ()}
    Returns the number of command-line arguments.  The program name
    itself is not included in this number.
  \item[\tt IDate]{\tt (TArray)}
    Returns the current local date day, month, year in
    elements 1, 2, and 3 of \code{Tarray}, respectively.
    The year has four significant digits. 
  \item[\tt IErrno]{\tt ()}
    Returns the last system error number (\code{errno} in C).
  \item[\tt ITime]{\tt (TArray)}
    Returns the current local time hour, minutes, and seconds in
    elements 1, 2, and 3 of \code{TArray}, respectively.
  \item[\tt LStat]{\tt (File, SArray, Status)}
    Obtains data about a file named \code{File} and places places it in
    the array \code{SArray}. The values in this array are as follows:
    \begin{enumerate}
      \item Device ID 
      \item Inode number 
      \item File mode 
      \item Number of links 
      \item Owner's UID 
      \item Owner's GID 
      \item ID of device containing directory entry for file
      \item File size (bytes) 
      \item Last access time 
      \item Last modification time 
      \item Last file status change time 
      \item Preferred I/O block size (-1 if not available) 
      \item Number of blocks allocated (-1 if not available)
    \end{enumerate}
    If an element is not available, or not relevant on the host system,
    it is returned as 0 except when indicated otherwise in the above list.
    If the optional \code{Status} argument is supplied, it contains 0 on
    success or a nonzero error code upon return.
  \item[\tt PError]{\tt (MsgPrefix)}
    Prints a newline-terminated error message corresponding to the last
    system error. This is prefixed by the string \code{MsgPrefix},
    a colon and a space.  The error message is printed on the C
    {\tt stderr} stream.
  \item[\tt Rename]{\tt (Path1, Path2, Status)}
    Renames the file named \code{Path1} to \code{Path2}. A \code{null}
    character marks the end of the names.  Trailing blanks are ignored.
    If the optional \code{Status} argument is supplied, it contains 0
    on success or a nonzero error code upon return.
  \item[\tt Sleep]{\tt (Seconds)}
    Causes the program to pause for \code{Seconds} seconds.
  \item[\tt System]{\tt (Command, Status)}
    Passes the string in \code{Command} to a shell though {\tt system(3)}.
    If the optional argument \code{Status} is present, it contains the
    value returned by {\tt system(3)}.
\end{itemize}

 


%%% Local Variables:
%%% mode: latex
%%% TeX-master: "abi"
%%% End:

\section{Fortran}

A formal Fortran ABI does not exist.  Most Fortran compilers are
designed for very specific high performance computing applications, so
Fortran compilers use different passing conventions and memory layouts
optimized for their specific purpose.  For example, Fortran
applications that must run on distributed memory machines need a
different data representation for array descriptors (also known as
dope vectors, or fat pointers) than applications running on symmetric
multiprocessor shared memory machines.  A normative ABI for Fortran is
therefore not desirable.  However, for interoperability of different
Fortran compilers, as well as for interoperability with other
languages, this section provides some some guidelines for data types
representation, and argument passing.  The guidelines in this section
are derived from the GNU Fortran 77 (G77) compiler, and are also
followed by the GNU Fortran 95 (\code{gfortran}) compiler (restricted to
Fortran 77 features).  Other Fortran
compilers already available for AMD64 at the time of this writing may
use different conventions, so compatibility is not guaranteed.

When this text uses the term {\em Fortran procedure}, the text applies
to both Fortran \code{FUNCTION} and \code{SUBROUTINE} subprograms as
well as for alternate \code{ENTRY} points,
unless specifically stated otherwise.

Everything not explicitely defined in this ABI is left to the implementation.

\subsection{Names}
\label{sub_fortran_names}

External names in Fortran are names of entities visible to all
subprograms at link time.  This includes names of \code{COMMON} blocks
and Fortran procedures.
To avoid name space conflicts with linked-in libraries, all external
names have to be mangled.  And to avoid name space conflicts
of mangled external names with local names, all local names must also
be mangled.
The mangling scheme is straightforward as follows:
\begin{itemize}
\item all names that do not have any underscores in it should have
{\em one} underscore appended
\item all external names containing one or
   more underscores in it (whereever) should have {\em two} underscores
   appended \footnote{Historically, this is to be compatible with f2c.}.
\item all external names should be mapped to lower case,
   following the traditional UNIX model for Fortran compilers
\end{itemize}

For examples see figure~\ref{fortran-external-names}:

\begin{figure}[H]
\Hrule
\caption{Example mapping of names} \label{fortran-external-names}
\begin{center}
\begin{footnotesize}
\begin{tabular}{l|l}
\multicolumn{1}{c}{Fortran external name}&\multicolumn{1}{c}{Linker name}\\
\hline
\code{FOO}    & \code{foo_} \\
\code{foo}    & \code{foo_} \\
\code{Foo}    & \code{foo_} \\
\code{foo_}   & \code{foo___} \\
\code{f_oo}   & \code{f_oo__} \\
\end{tabular}
\end{footnotesize}
\end{center}
\Hrule
\end{figure}

The entry point of the main program unit is called \code{MAIN__}.
The symbol name for the blank common block is \code{__BLNK__}.
the external name of the unnamed \code{BLOCK DATA} routine is
\code{__BLOCK_DATA__}.

\subsection{Representation of Fortran Types}
\label{sub_fortran_types}

For historical reasons, GNU Fortran 77 maps Fortran programs to the C
ABI, so the data representation can be explained best by providing the
mapping of Fortran types to C types used by G77 on AMD64\footnote{G77
  provides a header \code{g2c.h} with the equivalent C type
  definitions for all supported Fortran scalar types.} as in figure~
\ref{fortran-c-types}.  The ``\code{TYPE*N}'' notation specifies that
variables or aggregate members of type \code{TYPE} shall occupy
\code{N} bytes of storage.


\begin{figure}[H]
\Hrule
\caption{Mapping of Fortran to C types} \label{fortran-c-types}
\begin{center}
\begin{footnotesize}
\begin{tabular}{l|l|l}
\multicolumn{1}{c}{Fortran}&\multicolumn{1}{c}{Data
  kind}&\multicolumn{1}{c}{Equivalent C type}\\
\hline
\code{INTEGER*4}      &  Default integer                       &  \code{signed int} \\
\code{INTEGER*8}      &  Double precision integer              &  \code{signed long}\\
\code{REAL*4}         &  Single precision FP number            &  \code{float}\\
\code{REAL*8}         &  Double precision FP number            &  \code{double}\\
\code{COMPLEX*4}      &  Single precision complex FP number    &  \code{complex float}\\
\code{COMPLEX*8}      &  Double precision complex FP number    &  \code{complex double}\\
\code{LOGICAL}        &  Boolean logical type                  &  \code{signed int}\\
\code{CHARACTER}      &  Text string                           &  \code{char[] + length}\\
\end{tabular}
\end{footnotesize}
\end{center}
\Hrule
\end{figure}

The values for type \code{LOGICAL} are \code{.TRUE.} implemented as 1 and
\code{.FALSE.} implemented as 0.

Data objects with a \code{CHARACTER} type\footnote{This includes sub-strings.}
are represented as an array
of characters of the C char type (not guaranteed to be
``\code{$\backslash$0}'' terminated) with a separate length counter to
distinguish between \code{CHARACTER} data objects with a
length parameter, and aggregate types of \code{CHARACTER} data
objects, possibly also with a length parameter.

Layout of other aggregate types is implementation defined.  GNU
Fortran puts all arrays in contiguous memory in column-major order.
GNU Fortran 95 builds an equivalent C struct for derived types without
reordering the type fields.  Other compilers may use other
representations as needed.  The representation and use of Fortran
90/95 array descriptors is implementation defined.  Note that array indices
start at 1 by default.

Fortran 90/95 allow different kinds of each basic type using the \code{kind}
type parameter of a type.  Kind type parameter values are
implementation defined.

Layout of he commonly used Cray pointers is implementation defined.

\subsection{Argument Passing}

For each given Fortran 77 procedure, an equivalent C prototype can
be derived. Once this equivalent C prototype is known, the C ABI
conventions should be applied to determine how arguments are passed
to the Fortran procedure.

G77 passes all (user defined) formal arguments of a procedure by
reference.  Specifically, pointers to the location in memory of a
variable, array, array element, a temporary location that holds the
result of evaluating an expression or a temporary or permanent
location that holds the value of a constant (xf. g77 manual) are
passed as actual arguments. Artificial compiler generated arguments
may be passed by value or by reference as they are inherently
compiler and hence implementation specific.

Data objects with a \code{CHARACTER} type are passed as a pointer to
the character string and its length, so that each \code{CHARACTER}
formal argument in a Fortran procedure results in two actual arguments
in the equivalent C prototype.  The first argument occupies the
position in the formal argument list of the Fortran procedure.  This
argument is a pointer to the array of characters that make up the
string, passed by the caller.  The second argument is appended to the
end of the user-specified formal argument list.  This argument is of
the default integer type and its value is the length of the array of
characters, that is the length, passed as the first argument.  This
length is passed by value.
When more than one \code{CHARACTER} argument is present in an argument
list, the length arguments are appended in the order the original
arguments appear.  The above discussion also applies to sub-strings.

This ABI does not define the passing of optional arguments.  They are
allowed only in Fortran 90/95 and their passing is implementation defined.

This ABI does not define array functions (function returning arrays).
They are allowed only in Fortran 90/95 and requires the definition of
array descriptors.

Note that Fortran 90/95 procedure arguments with the \code{INTENT(IN)}
attribute should also passed by reference if the procedure is to be
linked with code written in Fortran 77.  Fortran 77 does not and can
not support the \code{INTENT} attribute because it has no concept of
explicit interfaces.  It is therefore not possible to declare the
callee's arguments as \code{INTENT(IN)}.  A Fortran 77 compiler must
assume that all procedure arguments are \code{INTENT(INOUT)} in the
Fortran 90/95 sense.


\subsection{Functions}

The calling of statement functions is implementation defined (as they
are defined only locally, the compiler has the freedom to apply any
calling convention it likes).

Subroutines with alternate returns (e.g. "SUBROUTINE X(*,*)" called
as "CALL X(*10,*20)") are implemented as functions returning
an \code{INTEGER} of the default kind.  The value of this returned integer
is whatever integer is specified in the "RETURN" statement for
the subroutine \footnote{
This integer indicates the position of an alternate
return from the subroutine in the formal argument list}, or 0 for
a \code{RETURN} statement without an argument.  It is up to the caller
to jump to the corresponding alternate return label.  The actual
alternate-return arguments are omitted from the calling sequence.

An example:

\begin{footnotesize}
\begin{verbatim}
       SUBROUTINE SHOW_ALTERNATE_RETURN (N)
          INTEGER N
          CALL ALTERNATE_RETURN_EXAMPLE (N, *10, *20, *30)
          WRITE (*,*) 'OK - Normal Return'
          RETURN
10        WRITE (*,*) '1st alternate return'
          RETURN
20        WRITE (*,*) '2nd alternate return'
          RETURN
30        WRITE (*,*) '2nd alternate return'
          RETURN
       END

       SUBROUTINE ALTERNATE_RETURN_EXAMPLE (N, *, *, *)
          INTEGER N
          IF (N .EQ. 0 ) RETURN     ! Implicit "RETURN 0"
          IF ( N .EQ. 1 ) RETURN 1
          IF ( N .EQ. 2 ) RETURN 2
          RETURN 3
       END
\end{verbatim}
\end{footnotesize}

Here the \code{SUBROUTINE ALTERNATE_RETURN_EXAMPLE} is implemented as
a function returning an \code{INTEGER*4} with value 0 if N is 0, 1 if N is 1,
2 if N is 2 and 3 for all other values of N.  This return value is used by
the caller as if the actual call were replaced by this sequence:
\begin{footnotesize}
\begin{verbatim}
          INTEGER X
          X = CALL ALTERNATE_RETURN_EXAMPLE (N)
          GOTO (10, 20, 30), X
\end{verbatim}
\end{footnotesize}

All in all the effect is that the index of the returned to label (starting
from 1) will be contained in \RAX after the call.

Alternate \code{ENTRY} points of a \code{SUBROUTINE} or \code{FUNCTION}
should be treated as separate subprograms, as mandated by the Fortran
standard.  I.e. arguments passed to an alternate \code{ENTRY} should be passed
as if the alternate \code{ENTRY} is a separate \code{SUBROUTINE} or
\code{FUNCTION}.  If a \code{FUNCTION} has alternate \code{ENTRY} points,
the result of each of the alternate \code{ENTRY} points must be returned
as if the alternate \code{ENTRY} is a separate \code{FUNCTION} with the
result type of the alternate \code{ENTRY}.  The external naming of alternate
\code{ENTRY} points follows section~\ref{sub_fortran_names}.

\subsection{COMMON blocks}
In absence of any \code{EQUIVALENCE} declaration involving variables
in \code{COMMON} blocks the layout of a 
\code{COMMON} block is exactly the same as the layout of the equivalent C
structure (with types of variables substituted according to section~
\ref{sub_fortran_types}), including the alignment requirements.

This ABI defines the layout under presence of \code{EQUIVALENCE} statements only
in some cases:
\begin{itemize}
\item the layout of the \code{COMMON} block must not change if one ignores
      the \code{EQUIVALENCE}, which amongst other things means:
\item If two arrays are equivalenced, the larger array must be named in
      the \code{COMMON} block, and there must be complete inclusion,
      in particular the other array may not extend the size of the 
      equivalenced segment.  It may also not change the alignment
      requirement.
\item If an array element and a scalar are equivalenced, the array must be
      named in the \code{COMMON} block and it must not be smaller than
      the scalar.  The type of the scalar must not require bigger alignment
      than the array.
\item if two scalars are equivalenced they must have the same size and
      alignment requirements.
\end{itemize}

Other cases are implementation defined.

Because the Fortran standard allows the blank \code{COMMON} block to have
different sizes in different subprograms, it may be impossible to determine
if it is small enough to fit in the \texttt{.bss}
section.  When compiling for the medium or large code models the blank
\code{COMMON} block should therefore always be put in the \texttt{.lbss}
section.

\subsection{Intrinsics}
This sections lists the set of intrinsics which has to be supported
at minimum by a conforming compiler.  They are separated by origin.
They follow regular calling and naming conventions.

The signature of intrinsics uses the syntax
$return-type(argtype1, argtype2, ...)$, where the individual types can
be the following characters: {\bf V} (as in void) designates a
\code{SUBROUTINE}, {\bf L} a \code{LOGICAL}, {\bf I} an \code{INTEGER},
{\bf R} a \code{REAL}, and {\bf C} a \code{CHARACTER}.
Hence \code{I(R,L)} designates a \code{FUNCTION} returning an
\code{INTEGER} and taking a \code{REAL} and a \code{LOGICAL}.
If an argument is an array, this is indicated using a trailing number,
e.g. {\bf I13} is an \code{INTEGER} array with 13 elements.
If a \code{CHARACTER} argument or return value has a fixed length, this
is indicated using an asterisk and a trailing number, for example
{\bf C*16} is a \code{CHARACTER(len=16)}.  If a \code{CHARACTER}
argument of arbitrary length must be passed, the trailing number is
replaced with \code{N}, for example {\bf C*N}.


\begin{table}[H]
\Hrule
  \caption{Mil intrinsics}
  \label{intrinsics-mil}
  \begin{center}
  \begin{tabular}[t]{l|l|l}
    \multicolumn{1}{c}{Name} & \multicolumn{1}{c}{Signature} & \multicolumn{1}{c}{Meaning} \\
    \hline
    \code{BTest}  & L(I,I)          & Test bit \\
    \code{IAnd}   & I(I,I)          & Boolean AND \\
    \code{IOr}    & I(I,I)          & Boolean OR \\
    \code{IEOr}   & I(I,I)          & Boolean XOR \\
    \code{Not}    & I(I)            & Boolean NOT \\
    \code{IBClr}  & I(I,I)          & Clear a bit \\
    \code{IBits}  & I(I,I,I)        & Extract a bit subfield of a variable \\
    \code{IBSet}  & I(I,I)          & Set a bit \\
    \code{IShft}  & I(I,I)          & Logical bit shift \\
    \code{IShftC} & I(I,I,I)        & Circular bit shift \\
    \code{MvBits} & V(I,I,I,I,I)    & Move a bit field \\
  \end{tabular}
  \end{center}
\Hrule
\end{table}

\begin{itemize}
  \item[\tt BTest]{\tt (I, Pos)}
    Returns \code{.TRUE.} if bit \code{Pos} in \code{I} is set,
    returns \code{.FALSE.} otherwise.
  \item[\tt IAnd]{\tt (I, J)}
    Returns value resulting from a boolean AND on each pair of bits
    in \code{I} and \code{J}.
  \item[\tt IOr]{\tt (I, J)}
    Returns value resulting from a boolean OR on each pair of bits
    in \code{I} and \code{J}.
  \item[\tt IEOr]{\tt (I, J)}
    Returns value resulting from a boolean XOR on each pair of bits
    in \code{I} and \code{J}.
  \item[\tt Not]{\tt (I)}
    Returns value resulting from a boolean NOT on each bit in \code{I}.
  \item[\tt IBClr]{\tt (I, Pos)}
    Returns the value of \code{I} with bit \code{Pos} cleared (set to zero).
  \item[\tt IBits]{\tt (I, Pos, Len)}
    Extracts a subfield starting from bit position \code{Pos} and with a
    length (towards the most significant bit) of \code{Len} bits from \code{I}.
    The result is right-justified and the remaining bits are zeroed.
  \item[\tt IBSet]{\tt (I, Pos)}
    Returns the value of \code{I} with the bit in position \code{Pos} set to one.
  \item[\tt IShft]{\tt (I, Shift)}
    All bits of \code{I} are shifted \code{Shift} places.
    \code{Shift.GT.0} indicates a left shift,
    \code{Shift.EQ.0} indicates no shift, and
    \code{Shift.LT.0} indicates a right shift.
    Bits shifted out from the least (when shifting right) or
    most (when shifting left) significant position are lost.
    Bits shifted in at the opposite end are not set (i.e. zero).
  \item[\tt IShftC]{\tt (I, Shift, Size)}
    The rightmost \code{Size} bits of the argument \code{I} are shifted
    circularly \code{Shift} places.
    The unshifted bits of the result are the same as the unshifted bits of I.
  \item[\tt MvBits]{\tt (From, FromPos, Len, To, ToPos)}
    Move \code{Len} bits of \code{From} from bit positions \code{FromPos}
    through \code{FromPos+Len-1} to bit positions \code{ToPos} through
    \code{ToPos+Len-1} of \code{To}. The bit portions of \code{To} that
    are not affected by the movement of bits are unchanged.
\end{itemize}

\begin{table}[H]
\Hrule
  \caption{F77 intrinsics}
  \label{intrinsics-f77}
  \begin{center}
  \begin{tabular}[t]{l|l}
    \multicolumn{1}{c}{Name} & \multicolumn{1}{c}{Meaning} \\
    \hline
    Abs & Absolute value \\
    ACos & Arc cosine \\
    AInt & Truncate to whole number \\
    ANInt & Round to nearest whole number \\
    ASin & Arc sine \\
    ATan & Arc Tangent \\
    ATan2 & Arc Tangent \\
    Char & Character from code \\
    Cmplx & Construct \code{COMPLEX(KIND=1)} value \\
    Conjg & Complex conjugate \\
    Cos & Cosine \\
    CosH & Hyperbolic cosine \\
    Dble & Convert to double precision \\
    DiM & Difference magnitude (non-negative subtract) \\
    DProd & Double-precision product \\
    Exp & Exponential \\
    IChar & Code for character \\
    Index & Locate a \code{CHARACTER} substring \\
    Int & Convert to \code{INTEGER} value truncated to whole number \\
    Len & Length of character entity \\
    LGe & Lexically greater than or equal \\
    LGt & Lexically greater than \\
    LLe & Lexically less than or equal \\
    LLt & Lexically less than \\
    Log & Natural logarithm \\
    Log10 & Common logarithm \\
    Max & Maximum value \\
    Min & Minimum value \\
    Mod & Remainder \\
    NInt & Convert to \code{INTEGER} value rounded to nearest whole number \\
    Real & Convert value to type \code{REAL(KIND=1)} \\
    Sin & Sine \\
    SinH & Hyperbolic sine \\
    SqRt & Square root \\
    Tan & Tangent \\
    TanH & Hyperbolic tangent \\
  \end{tabular}
  \end{center}
\Hrule
\end{table}

Refer to the Fortran 77 language standard for signature and definition
of the F77 intrinsics listed in table~\ref{intrinsics-f77}.  These intrinsics
can have a prefix as per the standard hence the table is not exhaustive.

\begin{table}[H]
\Hrule
  \caption{F90 intrinsics}
  \label{intrinsics-f90}
  \begin{center}
  \begin{tabular}[t]{l|l}
    \multicolumn{1}{c}{Name} & \multicolumn{1}{c}{Meaning} \\
    \hline
    AChar & ASCII character from code \\
    Bit_Size & Number of bits in arguments type \\
    CPU_Time & Get current CPU time \\
    IAChar & ASCII code for character \\
    Len_Trim & Get last non-blank character in string \\
    System_Clock & Get current system clock value \\
  \end{tabular}
  \end{center}
\Hrule
\end{table}

Refer to the Fortran 90 language standard for signature and definition
of the F90 intrinsics listed in table~\ref{intrinsics-f90}.

\begin{table}[H]
\Hrule
  \caption{Math intrinsics}
  \label{intrinsics-math}
  \begin{center}
  \begin{tabular}[t]{l|l|l}
    \multicolumn{1}{c}{Name} & \multicolumn{1}{c}{Signature} & \multicolumn{1}{c}{Meaning} \\
    \hline
    \code{BesJ0} & R(R)   & Bessel function \\
    \code{BesJ1} & R(R)   & Bessel function \\
    \code{BesJN} & R(I,R) & Bessel function \\
    \code{BesY0} & R(R)   & Bessel function \\
    \code{BesY1} & R(R)   & Bessel function \\
    \code{BesYN} & R(I,R) & Bessel function \\
    \code{ErF}   & R(R)   & Error function \\
    \code{ErFC}  & R(R)   & Complementary error function \\
    \code{IRand} & I(I)   & Random number \\
    \code{Rand}  & R(I)   & Random number \\
    \code{SRand} & V(I)   & Random seed \\
  \end{tabular}
  \end{center}
\Hrule
\end{table}

\begin{itemize}
  \item[\tt BesJ0]{\tt (X)}
    Calculates the Bessel function of the first kind of order 0 of X.
    Returns a \code{REAL} of the same kind as \code{X}.
  \item[\tt BesJ1]{\tt (X)}
    Calculates the Bessel function of the first kind of order 1 of X.
    Returns a \code{REAL} of the same kind as \code{X}.
  \item[\tt BesJN]{\tt (N, X)}
    Calculates the Bessel function of the first kind of order N of X.
    Returns a \code{REAL} of the same kind as \code{X}.
  \item[\tt BesY0]{\tt (X)}
    Calculates the Bessel function of the second kind of order 0 of X.
    Returns a \code{REAL} of the same kind as \code{X}.
  \item[\tt BesY1]{\tt (X)}
    Calculates the Bessel function of the second kind of order 1 of X.
    Returns a \code{REAL} of the same kind as \code{X}.
  \item[\tt BesYN]{\tt (N, X)}
    Calculates the Bessel function of the second kind of order N of X.
    Returns a \code{REAL} of the same kind as \code{X}.
  \item[\tt ErF]{\tt (X)}
    Calculates the error function of X.
    Returns a \code{REAL} of the same kind as \code{X}.
  \item[\tt ErFC]{\tt (X)}
    Calculates the complementary error function of X, i.e. \code{1 - ERF(X)}.
    Returns a \code{REAL} of the same kind as \code{X}.
  \item[\tt IRand]{\tt (Flag)}
    Flag is optional.
    Returns a uniform quasi-random number up to a system-dependent limit.
    If \code{Flag .EQ. 0} or \code{Flag} is not passed, the next number
    in sequence is returned.
    If \code{Flag .EQ. 1}, the generator is restarted.
    If \code{Flag} has any other value, the generator is restarted with
    the value of \code{Flag} as the new seed.
  \item[\tt Rand]{\tt (Flag)}
    Flag is optional.
    Returns a uniform quasi-random number between 0 and 1.
    If \code{Flag .EQ. 0} or \code{Flag} is not passed, the next number
    in sequence is returned.
    If \code{Flag .EQ. 1}, the generator is restarted.
    If \code{Flag} has any other value, the generator is restarted with
    the value of \code{Flag} as the new seed.
  \item[\tt SRand]{\tt (Seed)}
    Reinitializes the random number generator for \code{IRand} and \code{Rand}
    with the seed in \code{Seed}.
\end{itemize}

\begin{table}[H]
\Hrule
  \caption{Unix intrinsics}
  \label{intrinsics-unix}
  \begin{center}
  \begin{tabular}[t]{l|l|l}
    \multicolumn{1}{c}{Name} & \multicolumn{1}{c}{Signature} & \multicolumn{1}{c}{Meaning} \\
    \hline
    \code{Abort}  & V()               & Abort the program \\
    \code{Access} & I(C,C)            & Check file accessibility \\
    \code{DTime}  & V(R2,R)           & Get elapsed time since last call \\
    \code{ETime}  & V(R2,R)           & Get elapsed time for process \\
    \code{Flush}  & V(I)              & Flush buffered output \\
    \code{FNum}   & I(I)              & Get file descriptor from Fortran unit number \\
    \code{FStat}  & V(I,I13,I)        & Get file information \\
    \code{GError} & V(C*N)            & Get error message for last error \\
    \code{GetArg} & V(I,C*N)          & Obtain command-line argument \\
    \code{GetCWD} & V(C*N,I)          & Get current working directory \\
    \code{GetEnv} & V(C*N,C*N)        & Get environment variable \\
    \code{GetGId} & I()               & Get process group ID \\
    \code{GetPId} & I()               & Get process ID \\
    \code{GetUId} & I()               & Get process user ID \\
    \code{GetLog} & V(C*N)            & Get login name \\
    \code{HostNm} & V(C*N,I)          & Get host name \\
    \code{IArgC}  & I()               & Obtain count of command-line arguments \\
    \code{IDate}  & V(I3)             & Get local date info \\
    \code{IErrNo} & I()               & Get error number for last error \\
    \code{ITime}  & V(I3)             & Get local time of day \\
    \code{LStat}  & V(C*N,I13,I)      & Get file information \\
    \code{PError} & V(C*N)            & Print error message for last error \\
    \code{Rename} & V(C*N,C*N,I)      & Rename file \\
    \code{Sleep}  & V(I)              & Sleep for a specified time \\
    \code{System} & V(C*N,I)          & Invoke shell (system) command \\
  \end{tabular}
  \end{center}
\Hrule
\end{table}


\begin{itemize}
  \item[\tt Abort]{\tt ()}
    Prints a message and potentially causes a core dump.
  \item[\tt Access]{\tt (Name, Mode)}
    Checks file \code{Name} for accessibility in the mode specified by
    \code{Mode}.  Returns $0$ if the file is accessible in that mode,
    otherwise an error code.  \code{Name} must be a \code{NULL}-terminated
    string of \code{CHARACTER} (i.e. a C-style string).  Trailing blanks
    in \code{Name} are ignored.
    \code{Mode} must be a concatenation of any of the following characters:
    {\bf r} meaning test for read permission, {\bf w} meaning test for write
    permission, {\bf x} meaning test for execute/search permission, or
    a space meaning test for existence of the file.
  \item[\tt DTime]{\tt (TArray, Result)}
    When called for the first time, returns the number of seconds of runtime
    since the start of the program in \code{Result}, the user component of
    this runtime in \code{TArray(1)}, and the system time in \code{TArray(2)}.
    Subsequent invocations values based on accumulations since the previous
    invocation.
  \item[\tt ETime]{\tt (TArray, Result)}
    Returns the number of seconds of runtime since the start of the program
    in \code{Result}, the user component of
    this runtime in \code{TArray(1)}, and the system time in \code{TArray(2)}.
    Subsequent invocations values based on accumulations since the previous
    invocation.
  \item[\tt Flush]{\tt (Unit)}
    Flushes the Fortran I/O unit with ID \code{Unit}.  The unit must be open
    for output. If the optional \code{Unit} argument is omitted, all open
    units are flushed.
  \item[\tt FNum]{\tt (Unit)}
    Returns the UNIX(tm) file descriptor number corresponding to the Fortran
    I/O unit \code{Unit}.  The unit must be open.
  \item[\tt FStat]{\tt (Unit, SArray, Status)}
    Obtains data about the file open on Fortran I/O unit \code{Unit} and
    places it in the array \code{SArray}. The values in this array are
    as follows:
    \begin{enumerate}
      \item Device ID 
      \item Inode number 
      \item File mode 
      \item Number of links 
      \item Owner's UID 
      \item Owner's GID 
      \item ID of device containing directory entry for file
      \item File size (bytes) 
      \item Last access time 
      \item Last modification time 
      \item Last file status change time 
      \item Preferred I/O block size (-1 if not available) 
      \item Number of blocks allocated (-1 if not available)
    \end{enumerate}
    If an element is not available, or not relevant on the host system,
    it is returned as 0 except when indicated otherwise in the above list.
    If the optional \code{Status} argument is supplied, it contains 0 on
    success or a nonzero error code upon return.
  \item[\tt Gerror]{\tt (Message)}
    Returns the system error message corresponding to the last system error
    (errno in C). The message is returned in \code{Message}.
    If \code{Message} is longer than the error message, it is padded with
    blanks after the message.  If \code{Message} is not long enough to hold
    the error message, the error message is truncated to the length of
    \code{Message}.
  \item[\tt GetArg]{\tt (Pos, Value)}
    Returns in \code{Value} the command-line argument in position \code{Pos}.
    If there are fever than \code{Pos} command-line arguments, \code{Value}
    is filled with blanks.  If \code{Pos} is 0, the name of the program is
    returned.
    If \code{Value} is longer than the command-line argument, it is padded
    with blanks after the argument.  If \code{Value} is not long enough to
    hold the command-line argument, the argument is truncated to the length
    of \code{Value}.
  \item[\tt GetCWD]{\tt (Name, Status)}
    Returns in \code{Name} the current working directory. If the optional
    \code{Status} argument is supplied, it contains 0 on success or a
    nonzero error code upon return.
  \item[\tt GetEnv]{\tt (Name, Value)}
    Returns in \code{Value} the environment variable identified with
    \code{Name}.  If \code{Name} has not been set, \code{Value} is filled
    with blanks. A \code{null} character marks the end of the name in
    \code{Name}. Trailing blanks in \code{Name} are ignored.
    If \code{Value} is longer than the environment variable, it is padded
    with blanks after the variable.  If \code{Value} is not long enough to
    hold the environment variable, the variable is truncated to the length
    of \code{Value}.
  \item[\tt GetGId]{\tt ()}
    Returns the group ID for the current process.
  \item[\tt GetPId]{\tt ()}
    Returns the process ID for the current process.
  \item[\tt GetUId]{\tt ()}
    Returns the user ID for the current process.
  \item[\tt GetLog]{\tt (Login)}
    Returns the login name for the process in \code{Login}, or a blank
    string if the host system does not support \code{getlogin(3)}.
    If \code{Login} is longer than the login name, it is padded
    with blanks after the login name.  If \code{Login} is not long enough
    to hold the login name, the login name is truncated to the length of
    of \code{Login}.
  \item[\tt HotNm]{\tt (Name, Status)}
    Returns in \code{Name} system's host name. If the optional \code{Status}
    argument is supplied, it contains 0 on success or a nonzero error code
    upon return.
    If \code{Name} is longer than the host name, it is padded
    with blanks after the host name.  If \code{Name} is not long enough
    to hold the host name, the host name is truncated to the length of
    of \code{Name}.
  \item[\tt IArgC]{\tt ()}
    Returns the number of command-line arguments.  The program name
    itself is not included in this number.
  \item[\tt IDate]{\tt (TArray)}
    Returns the current local date day, month, year in
    elements 1, 2, and 3 of \code{Tarray}, respectively.
    The year has four significant digits. 
  \item[\tt IErrno]{\tt ()}
    Returns the last system error number (\code{errno} in C).
  \item[\tt ITime]{\tt (TArray)}
    Returns the current local time hour, minutes, and seconds in
    elements 1, 2, and 3 of \code{TArray}, respectively.
  \item[\tt LStat]{\tt (File, SArray, Status)}
    Obtains data about a file named \code{File} and places places it in
    the array \code{SArray}. The values in this array are as follows:
    \begin{enumerate}
      \item Device ID 
      \item Inode number 
      \item File mode 
      \item Number of links 
      \item Owner's UID 
      \item Owner's GID 
      \item ID of device containing directory entry for file
      \item File size (bytes) 
      \item Last access time 
      \item Last modification time 
      \item Last file status change time 
      \item Preferred I/O block size (-1 if not available) 
      \item Number of blocks allocated (-1 if not available)
    \end{enumerate}
    If an element is not available, or not relevant on the host system,
    it is returned as 0 except when indicated otherwise in the above list.
    If the optional \code{Status} argument is supplied, it contains 0 on
    success or a nonzero error code upon return.
  \item[\tt PError]{\tt (MsgPrefix)}
    Prints a newline-terminated error message corresponding to the last
    system error. This is prefixed by the string \code{MsgPrefix},
    a colon and a space.  The error message is printed on the C
    {\tt stderr} stream.
  \item[\tt Rename]{\tt (Path1, Path2, Status)}
    Renames the file named \code{Path1} to \code{Path2}. A \code{null}
    character marks the end of the names.  Trailing blanks are ignored.
    If the optional \code{Status} argument is supplied, it contains 0
    on success or a nonzero error code upon return.
  \item[\tt Sleep]{\tt (Seconds)}
    Causes the program to pause for \code{Seconds} seconds.
  \item[\tt System]{\tt (Command, Status)}
    Passes the string in \code{Command} to a shell though {\tt system(3)}.
    If the optional argument \code{Status} is present, it contains the
    value returned by {\tt system(3)}.
\end{itemize}

 


\appendix
\chapter{Linux Conventions}

This chapter describes some details that are only relevant to 
GNU/Linux systems and the Linux kernel.

\section{Execution of 32-bit Programs}

% Let's follow Sparc64 and MIPS64

The \xARCH processors are able to execute 64-bit \xARCH and also
32-bit ia32 programs.  Libraries conforming to the \intelabi will live
in the normal places like \path{/lib}, \path{/usr/lib} and
\path{/usr/bin}.  Libraries following the \xARCH, will use
\path{lib64} subdirectories for the libraries, e.g \path{/lib64} and
\path{/usr/lib64}.  Programs conforming to \intelabi and to the \xARCH
ABI will share directories like \path{/usr/bin}.  In particular, there
will be no \path{/bin64} directory.

\section{AMD64 Linux Kernel Conventions}

The section is informative only.

\subsection{Calling Conventions}

The Linux \xARCH kernel uses internally the same calling conventions as user-level
applications (see section \ref{sec-calling-conventions} for details).
User-level applications that like to call system calls should use the
functions from the C library.  The interface between the C library and
the Linux kernel is the same as for the user-level applications with
the following differences:
\begin{enumerate}
\item User-level applications use as integer registers for passing the
  sequence \RDI, \RSI, \RDX, \RCX, \reg{r8} and \reg{r9}.  The kernel
  interface uses \RDI, \RSI, \RDX, \reg{r10}, \reg{r8} and \reg{r9}.
\item A system-call is done via the \code{syscall} instruction.  The
  kernel destroys registers \RCX and \reg{r11}.
\item The number of the syscall has to be passed in register \RAX.
\item System-calls are limited to six arguments, no argument is passed
  directly on the stack.
\item Returning from the \code{syscall}, register \RAX contains the
  result of the system-call.  A value in the range between -4095 and
  -1 indicates an error, it is \code{-errno}.
\item Only values of class INTEGER or class MEMORY are passed to the
  kernel.
\end{enumerate}

\subsection{Stack Layout}

The Linux kernel does not honor the \textindex{red zone} (see
\ref{sec-stack-frame} and therefore this area is not allowed to be
used by kernel code.  Kernel code should be compiled by GCC with the
option \code{-mno-red-zone}.


\subsection{Required Processor Features}

Any program or kernel can expect that a \xARCH processor implements
the features mentioned in table~\ref{features}.  In general a program
has to check itself whether those features are available but for
\xARCH systems, these should always be available.
Table~\ref{features} uses the names for the processor features as
documented in the processor manual.

\begin{table}
\Hrule
\caption{Required Processor Features}\label{features}
  \begin{center}
\begin{tabular}{l|l}
\hline\noalign{\smallskip}
Feature & Comment\\
\hline
\multicolumn{2}{c}{Features need for programs}\\
\hline
fpu & Necessary for \code{long double}, MMX\\
tsc & User-visible\\
cx8 & User-visible\\
cmov& User-visible\\
mmx & User-visible\\
sse & User-visible, required for \code{float}\\
sse2& User-visible, required for \code{double}\\
fxsr& Required for SSE/SSE2 \\
syscall& For calling the kernel\\
\hline\noalign{\smallskip}
\multicolumn{2}{c}{Features need in the kernel}\\
\hline
pae& This kind of page tables is used \\
pse& PAE needs PSE.\\
msr & At least needed to enter long mode\\
pge & Kernel optimization\\
pat & Kernel optimization\\
clflush& Kernel optimization\\

  \end{tabular}
\end{center}
\Hrule
\end{table}

\subsection{Miscellaneous Remarks}

Linux Kernel code is not allowed to change the x87 and SSE units.  If
those are changed by kernel code, they have to be restored properly
before sleeping or leaving the kernel.  On preemptive kernels also
more precautions may be needed.


%%% Local Variables:
%%% mode: latex
%%% TeX-master: "abi"
%%% End:


\chapter{Index}

\printindex


%%% Local Variables:
%%% mode: latex
%%% TeX-master: "abi"
%%% End:



\end{document}
