\chapter{Low Level System Information}

\section{Machine Interface}

\subsection{Processor Architecture}

The \xARCH architecture (see
\url{http://www.x86-64.org/documentation}) defines extensions to
the x86 architecture.  The \xARCH architecture allows the execution of
32-bit programs in 32-bit mode.  Such programs will continue to make
use of the Intel386 psABI; this document applies only to programs
running in ``long'' mode.

\subsection{Data Representation}

Within this specification, the term \emph{halfword} refers to a 16-bit
object, the term \emph{word} refers to a 32-bit object, the term
\emph{doubleword} refers to a 64-bit object, and the term
\emph{quadword} refers to a 128-bit object.

\subsubsection{Fundamental Types}

Figure~\ref{basic-types} shows the correspondence between ISO C's
scalar types and the processor's.

\begin{figure}
  \caption{Scalar Types}\label{basic-types}
{ % Use small here - the table is still too large
  % Has anybody an idea how to shrink the table so that it fits the page?
  \small
  \begin{tabular}{l|l|c|c|l}
    \hline\noalign{\smallskip}
     & &  & \multicolumn{1}{c|}{Alignment} & \multicolumn{1}{c|}{x86-64} \\
    \multicolumn{1}{c|}{Type} & \multicolumn{1}{c|}{C}
     &  \texttt{sizeof} & (bytes)   
     & \multicolumn{1}{c|}{Architecture}  \\
    \hline
    & \texttt{char}        & 1 & 1 & signed byte \\
    & \texttt{signed char} & & \\
    \cline{2-5}
    & \texttt{unsigned char} & 1 & 1 & unsigned byte \\
    \cline{2-5}
    & \texttt{short} & 2 & 2 & signed halfword \\
    & \texttt{signed short} & & \\
    \cline{2-5}
    & \texttt{unsigned short} & 2 & 2 & unsigned halfword \\
    \cline{2-5}
    & \texttt{int} & 4 & 4 & signed word \\
    Integral & \texttt{signed int} & & \\
    & \texttt{enum} & & \\
    \cline{2-5}
    & \texttt{unsigned int} & 4 & 4 & unsigned word \\
    \cline{2-5}
    & \texttt{long} & 8 & 8 & signed doubleword \\
    & \texttt{signed long} & & \\
    & \texttt{long long} & & \\
    & \texttt{signed long long} & & \\
    \cline{2-5}
    & \texttt{unsigned long} & 8 & 8 & unsigned doubleword \\
    & \texttt{unsigned long long} & 8 & 8 & unsigned doubleword \\
    \hline
    Pointer & \texttt{\textit{any-type} *} & 8 & 8 & unsigned doubleword \\
    & \texttt{\textit{any-type} (*)()} & & \\
    \hline
    Floating-& \texttt{float} & 4 & 4 & single (IEEE) \\
    point & \texttt{double} & 8 & 8 & double (IEEE) \\
    & \texttt{long double} & 16 & 16 & 80-bit extended (IEEE) \\
    & \texttt{__float128} & 16 & 16 & 128-bit extended (IEEE) \\
    \hline
    Packed & \texttt{__m64} & 8 & 8 & \MMX{} and \threednow \\
    & \texttt{__m128} & 16 & 16 & SSE and SSE-2 \\
  \end{tabular}
}
\end{figure}

The \codeindex{__float128} type uses a 15-bit
exponent, a 113-bit mantissa (the high order significant bit is
implicit) and an exponent bias of 16383.%
\footnote{Initial implementations of the \xARCH architecture are
  expected to support operations on the \texttt{__float128} type only
  via software emulation.}

The \code{long double} type uses a 15 bit exponent, a 64-bit mantissa
with an explicit high order significand bit and an exponent bias of
16383\footnote{This is the x87 double extented precision data type.}.
The contents of the 6 padding bytes is undefined.

A null pointer (for all types) has value zero.

Like the Intel386 architecture, the \xARCH architecture does not
require all data access to be properly aligned.  Accessing misaligned
data will be slower than accessing properly aligned data, but
otherwise there is no difference.

\subsubsection{Aggregates and Unions}

An array uses the same alignment as its elements, except that an array
that requires at least 16 bytes always has alignment of at least 16
bytes.\footnote{This alignment requirement allows the use of SSE
  operations.}

No other changes required.

\subsubsection{Bit-Fields}

Amend the description of bit-field ranges as follows:

\begin{figure}[h]
\Hrule
  \caption{Bit-Field Ranges}
  \begin{center}
    \leavevmode
    \begin{tabular}{l|l|l}
      \multicolumn{1}{c}{Bit-field Type} 
         & \multicolumn{1}{c}{Width $w$} 
         & \multicolumn{1}{c}{Range} \\
      \hline
      \texttt{signed long} & & $-2^{w - 1}$ to $2^{w-1}-1$ \\
      \texttt{long} & 1 to 64 & 0 to $2^{w}-1$ \\
      \texttt{unsigned long} & & 0 to $2^{w}-1$ \\
    \end{tabular}
  \end{center}
\Hrule
\end{figure}

Properties of bitfields having the type \texttt{__m64} or
\texttt{__m128} are not defined by the ABI.  Although compilers are
free to support bitfields with these types, programs making use of
these features will not conform to the \xARCH ABI.

No other changes required.

\section{Function Calling Sequence}

This section describes the standard function calling sequence,
including stack frame layout, register usage, parameter passing and so
on.

The standard calling sequence requirements apply only to global
functions.  Local functions that are not reachable from other
compilation units may use different conventions.  Nevertheless, it is
recommended that all functions use the standard calling sequence when
possible.

\subsection{Registers}
\label{subsec-registers}

The \xARCH architecture provides 16 general purpose 64-bit registers.
In addition the architecture provides 16 SSE registers, each 128 bits
wide and 8 x87 floating point registers, each 80 bits wide.  Each of
the x87 floating point registers may be referred to in \MMX/\threednow
mode as a 64-bit register.  There are several additional
special-purpose registers.

This subsection discusses usage of each register.  The registers called
\emph{global registers} ``belong'' to the calling function and the called function
is required to preserve their values, while the \emph{temporary registers} may
be freely clobbered by the called function.
\editornote{The split to global/temporaries is still subject of change
  and depends on the different proposals for passing.}

\subsubsection {General Purpose Registers}

\begin{description}
  \item [\RAX]
    Temporary register used to pass first integer argument and to return
    the first doubleword of the return value.
  \item [\RDX]
    Temporary register used to pass second integer argument
    and to return the second doubleword of the return value.
  \item [\RCX] Temporary register used to pass third integer argument.
  \item [\RBX] Temporary register used to pass fourth integer argument.
  \item [\RBP] Global register optionally used as frame pointer.
  \item [\RSP] The stack pointer.
  \item [\RSI]
    Temporary register used to pass fifth doubleword of argument area.
  \item [\RDI] 
    Temporary register used to pass sixth doubleword of argument area.
  \item [\reg{R8}, \reg{R9}] Temporary registers.
  \item [\reg{R10} -- \reg{R15}] Global registers.
\end{description}

\subsubsection {SSE registers}

\begin{description}
\item [\reg{XMM0}, \reg{XMM1}] Temporary registers used to pass first
  two SSE arguments. \reg{XMM0} is also used as a return register for
  two doublewords.
  \item [\reg{XMM2} -- \reg{XMM6}] Temporary registers used to pass 2nd--6th SSE arguments
  \item [\reg{XMM6} -- \reg{XMM13}] Global registers
  \item [\reg{XMM13} -- \reg{XMM15}] Temporary registers
\end{description}

\subsubsection {\MMX/\threednow and x87 registers}

Since the usage of SSE instruction set if prefered over the x87, the CPU is in
``\MMX'' mode across function calls. Every function that uses x87 register
stack is required to convert the registers properly using \op{emms} or \op{femms}
instructions and thus x87 registers cannot be used to hold values across a
function call.  To make usage of x87 easier, all \MMX{} registers are temporaries,
so the FEMMS instruction may be used.
\begin{description}
\item [\reg{MM0}, \reg{MM1}] Temporary registers used to return first
  two doublewords of the return value.
  \item [\reg{MM2} -- \reg{MM8}] Temporary registers.
  \item [\reg{st(0)} -- \reg{st(8)}] Temporary registers not available across function calls
\end{description}


\subsubsection {Special purpose registers}
All special purpose registers should be handled as global registers
with the exception of the direction flag in EFLAGS register.  The
value of the direction flag must be clear at the entry of a function.
\editornote{Do we want to define values of fpu and SSE control words?}

\subsection{The Stack Frame}
In addition to registers, each function has a frame on the run-time stack.  This
stack grows downwards from high addresses, figure \ref{fig-stack-frame} shows the
stack organization.


\begin{figure}
\Hrule
  \caption{Stack Frame}
  \label{fig-stack-frame}
  \begin{center}
    \begin{tabular}{r|cl|l}
      \noalign{\smallskip}
      \multicolumn{1}{l}{\bf Frame} &
      \multicolumn{1}{c}{\bf Contents} &
      \multicolumn{1}{l}{\bf Size} &
      \multicolumn{1}{l}{\bf Address} \\
      \noalign{\smallskip}  \cline{2-3}
      & incoming arguments: doubleword $n$ & 8 bytes & $\RBP+16+8n$ \\
      & \dots & & \\
      Previous& incoming arguments: doubleword $0$ & 8 bytes & $\RBP+16$ \\ \cline{1-4} 
      Current & return address & 8 bytes & $\RBP+8$\\ \cline{2-3}
      & previous \RBP{} value & 8 bytes & $\RBP$ \\ \cline{2-3}
      & local save area & unspecified & $\RBP-8$\\ 
      & \dots & & \\ \cline {2-3}
      & outgoing arguments: doubleword $0$ & 8 bytes &\\
      & \dots & & \\
      & outgoing arguments: doubleword $n$ & 8 bytes & $\RSP$ \\ \cline{2-3}
    \end{tabular}
  \end{center}
\Hrule
\end{figure}


Functions may expect the end of the input argument area to be aligned
to a 16 byte boundary --- this means that value $(\RSP - 8)$ is always a
multiple of $16$ at the entry point. This alignment must be preserved by
all called functions that use the standard calling convention. The
frames of functions with local calling conventions may or may not obey
this rule so algorithms for stack unwinding cannot rely on it.

At any point of program execution, the stack pointer, \RSP, shall
point to the end of the latest allocated stack frame.

The organization of the \emph{local save area} is implementation
specific. Usual practice is to save clobbered global registers in the
prologue at the beginning or end of the save area and use the rest for saving
local object and spilled variables.  Saving registers at the beginning
is harder to implement, since the offset from \RBP{} is not known
until after the register allocation, but overall results in shorter
and faster prologues and epilogues allowing to allocate and deallocate
the outgoing arguments area by the same instruction used to
(de)allocate the \emph{local save area}.

The conventional use of \RBP{} as a frame pointer (base pointer) for a stack
frame may be avoided by using \RSP{} (the stack pointer) to index the
stack frame in cases where the frame is fixed length.%
\footnote{This corresponds to the \code{-fomit-frame-pointer} option of GCC}
This saves two instructions in the prologue and epilogue and
gives another global general purpose register.  See details about
stack traceback algorithm for conditions when this optimization is
allowed.  \editornote{The stack traceback algorithm has not been
  specified yet and we are not sure if we need one at all.}

\subsection{Classes}

The \xARCH CPU has three register classes available for parameter and
return value passing --- the integer (general purpose) registers, SSE
registers and \MMX{} registers.  All integral types and pointer belong
to the class \emph{Int}, all floating point values (including the
128bit type) and type \verb|__m128| belongs to the class \emph{SSE}
and finally the type \verb|__mm64| to class \emph{MMX}.

Aggregate and union types are split to doublewords and each doubleword has
assigned its own class. To assign the class of doubleword \emph{X} the
first matching rule of the following is used.
\begin{enumerate}
  \item (Aggregates only) In case the doubleword \emph{X} contains upper half field of
    \verb|__m128| type, the class is \emph{SSE}.
    
  \item In case all fields of an aggregate (or all subtypes of a
    union) crossing the word \emph{X} have the same class \emph{Y},
    the class is \emph{Y}.
    
  \item In case all fields of an aggregate (or all subtypes of a
    union) crossing the word \emph{X} have the same class \emph{SSE}
    or \emph{SSE2}, the class is \emph{SSE}.

  \item If nothing matches, use class is \emph{Int}.
\end{enumerate}

Doubleword \emph{X} of an aggregate gets assigned class \emph{Y} if and only
if all the fields crossing doubleword \emph{X} belongs to same class \emph{Y}.  If
no field crosses the given doubleword, its type is \emph{None}. When no class fits
the rules above, class \emph{Int} is used.  Exception is also 
\editornote{\textbf{Jan}, something is missing here --- please fill in the rest
  of the sentence.}

Similarly doubleword \emph{X} of a union gets assigned class \emph{Y}
if and only if each type in the union either does not cross \emph{X}
or its class for doubleword \emph{X} is \emph{Y}.

\subsection{Parameter Passing Conventions}

It is generally more efficient to pass arguments to called functions
in registers than to push them onto the stack. Even with good support
for memory operands and \op{push} instruction, passing arguments in
registers reduces memory bandwidth as well as simplifies the cleanup
after function call.  The number of registers implemented in the
processor architecture naturally limits the number of arguments passed
in this manner.

For \xARCH up to six general purpose registers and six SSE registers
are used for parameter passing.  If fewer (or no) arguments are
passed, the unneeded registers will contain undefined value at the
entry of function.  
\editornote{The number of registers used for passing is subject to
  change.}

\MMX{} registers are not used to pass \verb|__mm64| type, since \MMX{}
usage is depreached and doing so simplifies functions with variable
number of arguments.  Instead all arguments of type \verb|__mm64| as
well as arguments that do not fit into the available registers are
passed on the runtime stack.

The aggregates and unions shorter than 16~bytes are passed directly
using registers, while others by reference.  The caller is expected to
make a copy of the object so the callee is allowed to freely modify
the contents.  The caller may not copy the object in case it is known
to be unused after a function call, it does have proper alignment and
no aliases. The aggregate or union, when passed in registers is
represented equally as if it were passed at the stack slot.

The following algorithm is used to determine where data is passed for
C language.  For this purpose consider arguments as ordered from left
(first argument) to right (last arguments), although the order of
evaluation is unspecified. In this algorithm \verb|sr| contains the
first available SSE register and \verb|gr| contains the next available
general purpose register and \verb|ap| is used as a pointer to the
``output argument area'' on the stack.  As written, the algorithm is
not directly applicable, since it fills the stack in the opposite
direction.  In reality a two pass algorithm is required.  The first
pass computes positions for output arguments and the second pass
actually pushes the arguments onto the stack and/or loads them into
registers.


\subsubsection {Proposal 1: ``Smart'' mapping of structures}

\begin{description}
  \item[Initialize:]
    Set $\verb|sr|=1$, $\verb|gr|=1$, $\verb|ap|=0$


  \item[Scan:]
    If there are no more arguments, go to ``terminate''. Otherwise select one of the
    following rules depending on the type of the next argument:
      \begin{description}
        \item[Simple integer argument:]
          A simple integer argument is one of the following:
          \begin{itemize}
            \item One of the simple integer types, maximal 64~bits long.
            \item A pointer.
            \item Aggregates and unions greater than 16 bytes should
              be treated as pointer to object, or pointer to a copy of
              the objects if necessary to enforce call-by-value
              semantics.
            \item Aggregates and unions maximal 8 bytes long with
              class \emph{Int}.
          \end{itemize}
        If $\verb|gr| >= 6$ go to other, otherwise pick the \verb|gr|-th register from
        the sequence: \RAX, \RDX, \RCX, \RBX, \RSI, \RDI{} and assign it to the current
        argument. Set \verb|gr| to $\verb|gr|+1$.


    \item[Simple SSE argument:]
      A simple SSE argument is one of the following:
      \begin{itemize}
        \item One of the simple floating point types.
        \item An argument of type \verb|__m128|.
        \item Aggregates and unions maximal 8 bytes long with class \emph{SSE}.
        \item Aggregates and unions greater than 8 bytes and smaller
          than 17 bytes with class  \emph{SSE} of the first doubleword
          and class \emph{SSE} of the second doubleword.
      \end{itemize}
      If $\verb|sr| >= 6$ go to other, otherwise pick the \verb|sr|-th
      register from the sequence: \reg{XMM0}, \reg{XMM1}, \reg{XMM2},
      \reg{XMM3}, \reg{XMM4}, \reg{XMM5}, \reg{XMM6} and assign it to
      the current argument.  Set \verb|sr| to $\verb|sr|+1$.

    \item[Double integer argument]
      A double integer argument is one of the following:
      \begin{itemize}
        \item An argument of type \verb|__int128|.
        \item Aggregates and unions greater than 8 bytes and smaller
          than 17 bytes with class \emph{Int} for both doublewords.
      \end{itemize}
      If $\verb|gr| >= 5$ go to other, otherwise pick the \verb|gr|-th register from
      the sequence: \RAX, \RDX, \RCX, \RBX, \RSI, \RDI{} and assign it together with the
      next register to the current argument.
      Set \verb|gr| to $\verb|gr|+2$.

    \item[Double SSE argument]
      A double SSE argument is one of the following:
      \begin{itemize}
        \item Aggregates and unions greater than 8 bytes and smaller
          than 17 bytes with class \emph{SSE} for both doublewords.
      \end{itemize}
      If $\verb|sr| >= 5$ go to other, otherwise pick the \verb|sr|-th
      register from the sequence: \reg{XMM0}, \reg{XMM1}, \reg{XMM2},
      \reg{XMM3}, \reg{XMM4}, \reg{XMM5}, \reg{XMM6} and assign it
      together with the next register to the current argument.  Set
      \verb|sr| to $\verb|sr|+2$.

    \item[Integer-SSE type argument]
      An Integer-SSE type argument is one of the following:
      \begin{itemize}
        \item Aggregates and unions greater than 8 bytes and smaller
          than 17 bytes with class \emph{Int} for the first doubleword
          and \emph{SSE} for the second doubleword.
      \end{itemize}
      If $\verb|sr| >= 6$ or $\verb|gr| >= 6$ go to other, otherwise
      pick the \verb|gr|-th register from the sequence: \RAX, \RDX,
      \RCX, \RBX, \RSI, \RDI{} and assign it the first doubleword of
      the current argument and pick the \verb|sr|-th register from the
      sequence: \reg{XMM0}, \reg{XMM1}, \reg{XMM2}, \reg{XMM3},
      \reg{XMM4}, \reg{XMM5}, \reg{XMM6} and assign it the second
      doubleword of the current argument.  Set \verb|gr| to
      $\verb|gr|+1$ and set \verb|sr| to $\verb|sr|+1$.

    \item[SSE-Integer type argument]
      An SSE-Integer type argument is one of the following:
      \begin{itemize}
      \item Aggregates and unions greater than 8 bytes and smaller
        than 17 bytes with class \emph{SSe} for the first doubleword
        and \emph{Int} for the second doubleword.
      \end{itemize}
      If $\verb|sr| >= 6$ or $\verb|gr| >= 6$ go to other, otherwise
      pick the \verb|sr|-th register from the sequence: \reg{XMM0},
      \reg{XMM1}, \reg{XMM2}, \reg{XMM3}, \reg{XMM4}, \reg{XMM5},
      \reg{XMM6} and assign it the first doubleword of the current
      argument and pick the \verb|gr|-th register from the sequence:
      \RAX, \RDX, \RCX, \RBX, \RSI, \RDI{} and assign it the second
      doubleword of the current argument.  Set \verb|gr| to
      $\verb|gr|+1$ and set \verb|sr| to $\verb|sr|+1$.

   \end{description}
   Go to ``Scan''.

  \item[Other:]

    Arguments not handled otherwise are passed at stack frame. \RSP{} is the
    pointer to the end of the outgoing argument area and we do expect \RSP{} to be
    aligned to a 16 byte boundary and the outgoing area large enough to fit all
    the arguments (this means the value \verb|ap| bytes for the final value of
    \verb|ap| once algorithm terminates).  To compute the address of a given argument,
    round \verb|ap| to the nearest multiple of the operand's alignment in bytes
    (minimally to 8 bytes) and
    assign location $\verb|ap|+\RSP$ to the argument and increase \verb|ap| by
    the size of the argument in bytes.
    Go to ``Scan''.

  \item[Terminate:]
    Round \verb|ap| up to a multiple of 16. Terminate the algorithm.
\end{description}

As an example assume the function call as shown below:

\begin{verbatim}
int a,b,c,d,e;
float A,B;
double C,D;
long double E;
__mm64 mm;
struct small {long a; double b;} small;
struct large {int a,b,c,d,e,f;} small;
func (small, large, mm, a, b, c, d, e, A, B, C, D, E);
\end{verbatim}

The results of register allocation for parameter passing is shown in table \ref{tab-passing}.

\begin{table}
  \caption{Register Allocation For Parameter Passing.}\label{tab-passing}
  \begin{tabular}{rlrlrl}
    \hline\noalign{\smallskip}
    integer register & contents & SSE register & contents & stack slot & contents\\
    \noalign{\smallskip}\hline\noalign{\smallskip}
    \RAX & \verb|small.a| & \reg{XMM0} & \verb|small.b| & $\RSP$ & \verb|mm|\\
    \RDX & \verb|&large| & \reg{XMM1} & \verb|A| & $\RSP+8$ & \verb|e|\\
    \RCX & \verb|a| & \reg{XMM2} & \verb|B| &  & \\
    \RBX & \verb|b| & \reg{XMM3} & \verb|C| &  & \\
    \RSI & \verb|c| & \reg{XMM4} & \verb|D| &  & \\
    \RDI & \verb|d| & \reg{XMM5} & \verb|E| &  & \\
    \hline
  \end{tabular}
\end{table}

\subsubsection {Proposal 2: Structures always mapped to integers}

\begin{description}
  \item[Initialize:]
    Set $\verb|sr|=1$, $\verb|gr|=1$, $\verb|ap|=0$


  \item[Scan:]
    If there are no more arguments, go to ``terminate''. Otherwise select one of the
    following rules depending on the type of the next argument:
      \begin{description}
        \item[Simple integer argument:]
          A simple integer argument is one of the following:
          \begin{itemize}
            \item One of the simple integer types, maximal 64 bits long.
            \item A pointer.
            \item Aggregates and unions greater than 16 bytes should be treated as
              pointer to object, or pointer to a copy of the objects if necessary to enforce
              call-by-value semantics.
            \item Aggregates and unions maximal 8 bytes long with class \emph{Int}.
          \end{itemize}
        If $\verb|gr| >= 6$ go to other, otherwise pick the \verb|gr|-th register from
        the sequence: \RAX, \RDX, \RCX, \RBX, \RSI, \RDI{} and assign it to the current
        argument. Set \verb|gr| to $\verb|gr|+1$.


    \item[Simple SSE argument:]
      A simple SSE argument is one of the following:
      \begin{itemize}
        \item One of the simple floating point types.
        \item An argument of type \verb|__m128|.
        \item Aggregates and unions maximal 8 bytes long with class \emph{SSE}.
        \item Aggregates and unions greater than 8 bytes and smaller
          than 17 bytes with class \emph{SSE} for the first doubleword
          and class \emph{SSE2} for the second doubleword.
      \end{itemize}
      If $\verb|sr| >= 6$ go to other, otherwise pick the \verb|sr|-th
      register from the sequence: \reg{XMM0}, \reg{XMM1}, \reg{XMM2},
      \reg{XMM3}, \reg{XMM4}, \reg{XMM5}, \reg{XMM6} and assign it to
      the current argument.
      Set \verb|sr| to $\verb|sr|+1$.

    \item[Double integer argument]
      A double integer argument is one of the following:
      \begin{itemize}
        \item An argument of type \verb|__int128|.
        \item Aggregates and unions greater than 8 bytes and smaller
          than 17 bytes.
      \end{itemize}
      If $\verb|gr| >= 5$ go to other, otherwise pick the \verb|gr|-th register from
      the sequence: \RAX, \RDX, \RCX, \RBX, \RSI, \RDI{} and assign it together with the
      next register to the current argument.
      Set \verb|gr| to $\verb|gr|+2$.
   \end{description}
   Go to ``Scan''.

  \item[Other:]

    Arguments not handled otherwise are passed at stack frame. \RSP{} is the
    pointer to the end of outgoing argument area and we do expect \RSP{} to be
    aligned to a 16 byte boundary and the outgoing area space large enough to fit all
    the arguments (this means the value \verb|ap| bytes for the final value of
    \verb|ap| once algorithm terminates).  To compute the address of a given argument,
    round \verb|ap| to the nearest multiple of the operand's alignment in bytes
    (minimal to 8 bytes) and
    assign location $\verb|ap|+\RSP$ to the argument and increase \verb|ap| by
    the size of the argument in bytes.
    Go to ``Scan''.

  \item[Terminate:]
    Round \verb|ap| up to a multiple of 16. Terminate the algorithm.
\end{description}

As an example assume the function call as shown below:

\begin{verbatim}
int a,b,c,d,e;
float A,B;
double C,D;
long double E;
__mm64 mm;
struct small {long a; double b;} small;
struct large {int a,b,c,d,e,f;} small;
func (small, large, mm, a, b, c, d, e, A, B, C, D, E);
\end{verbatim}

The results of register allocation for parameter passing is shown in table \ref{tab-passing2}.

\begin{table}
  \caption{Register Allocation For Parameter Passing.}\label{tab-passing2}
  \begin{tabular}{rlrlrl}
    \hline\noalign{\smallskip}
    integer register & contents & SSE register & contents & stack slot & contents\\
    \noalign{\smallskip}\hline\noalign{\smallskip}
    \RAX & \verb|small.a| & \reg{XMM0} & \verb|A| & $\RSP$ & \verb|mm|\\
    \RDX & \verb|small.b| & \reg{XMM1} & \verb|B| & $\RSP+8$ & \verb|e|\\
    \RCX & \verb|&large| & \reg{XMM2} & \verb|C| &$\RSP+16$& \verb|d| \\
    \RBX & \verb|a| & \reg{XMM3} & \verb|D| &  & \\
    \RSI & \verb|b| & \reg{XMM4} & \verb|E| &  & \\
    \RDI & \verb|c| & \reg{XMM5} & undefined &  & \\
    \hline
  \end{tabular}
\end{table}


\subsection{Variable Argument Lists}

Some otherwise portable C programs depend on the argument passing scheme,
implicitly assuming that 1) all arguments are passed on the stack, and 2)
arguments appear in increasing order on the stack. Programs that make these
assumptions never have been portable, but they have worked on many
implementations. However, they do not work on the \xARCH architecture because
some arguments are passed in registers. Portable C programs use the header
files \verb|stdarg.h| or \verb|varargs.h| to deal with variable argument lists
on x86-64 and other machines as well.

\editornote {Use flag to signalize SSE registers?}

\subsection{Return Values}

\subsubsection {Proposal 1: ``Smart'' mapping of structures}

Scalars, complex numbers, aggregates and unions not exceeding 16 bytes
or no value are returned directly.  First one or two classes are
assigned to a given value and then the relevant registers are used as
discussed in subsection \ref{subsec-registers} leaving the rest of
registers undefined.

Scalar values are returned in a single register. One of the registers
\RAX, \reg{XMM0} and \reg{MM0} is chosen according to its class.
\verb|__int128| is returned in \RDX:\RAX.

The first doubleword of unions, aggregates and real part of complex
values not exceeding 8 bytes is returned the same way as simple scalar
registers.  The second doubleword (if available) is returned in one of
the following registers: \RAX, \reg{XMM0} and \reg{MM1} depending on
their class.  Class \emph{SSE} is returned in the upper half of
register \reg{XMM0}.

Complex numbers, aggregates and unions exceeding 16 bytes are stored
in an area prepared by the caller and passed to the function as
``hidden'' first parameter.  The contents of all return registers is
left undefined.
\editornote{i386 abi returns the pointer to the area to simplify
  caller.  Is that a good idea?}

\subsubsection {Proposal 2: Integer-only mapping of structures}

The \verb|__mm64| type is returned using \reg{MM0}.  Complex numbers,
aggregates and unions exceeding 16 bytes are stored in an area
prepared by the caller and passed to the function as ``hidden'' first
parameter.  The contents of all return registers is left undefined.

Other values are returned as if they were passed as first argument to
the function.

\section{Operating System Interface}

\subsection{Virtual Address Space}

Although the \xARCH architecture uses 64-bit pointers, implementations
are only required to handle 48-bit addresses.  Therefore, conforming
processes may only use addresses from \texttt{0x0000000000000000} to
\texttt{0x00007fffffffffff}\footnote{0x0000ffffffffffff is not a
  canoncial address and cannot be used.}.

No other changes required.

\subsection{Page Size}

Systems are permitted to use any page size between 4KB and 64KB,
inclusive.

No other changes required.

\subsection{Virtual Address Assignments}

Conceptually processes have the full address space available.
In practice, however, several factors limit the size of a process.
\begin{itemize}
  \item The system reserves a configuration dependent amount of virtual space.
  \item The system reserves a configuration dependent amount of space per
    process.
  \item
    A process whose size exceeds the system's available combined physical
    memory and secondary storage cannot run. Although some physical memory
    must be present to run any process, the system can execute processes that
    are bigger than physical memory, paging them to and from secondary storage.
    Nonetheless, both physical memory and secondary storage are
    shared resources. System load, which can vary from one program execution
    to the next, affects the available amount.
\end{itemize}

\begin{figure}[H]
\Hrule
  \caption{Virtual Address Configuration}
  \label{fig-address}
  \begin{center}
    \begin{tabular}{r|c|l}
      \noalign{\smallskip}  \cline{2-2}
      \verb|0xffffffffffffffff| & Reserved system area & End of memory\\ 
      & \dots & \\ \cline{2-2}
      & \dots & \\
      \verb|0x80000000000| & Dynamic segments & \\ \cline{2-2}
      & \dots & \\
      \verb|0| & Process segments & Beginning of memory\\ \cline{2-2}
    \end{tabular}
  \end{center}
\Hrule
\end{figure}

Although applications may control their memory assignments, the typical
arrangement appears in figure \ref{fig-cfg}.

\begin{figure}[H]
\Hrule
  \caption{Conventional Segment Arrangements}
  \label{fig-cfg}
  \begin{center}
    \begin{tabular}{r|c|l}
      \cline{2-2}
      & \dots & \\
      \verb|0x80000000000| & Dynamic segments & \\ \cline{2-2}
      & Stack segment & \\ 
      & \dots & \\ \cline{2-2}
      & \dots & \\
      & Data segments & \\ \cline{2-2}
      & \dots & \\
      \verb|0x10000| & Text segments & \\ \cline{2-2}
      \verb|0| & Unmapped & \\ \cline{2-2}
    \end{tabular}
  \end{center}
\Hrule
\end{figure}

\section{Coding Examples}

The following sections show only the difference to the i386 ABI.

\subsection{Position-Independend Function Prologue}

\xARCH does not need any function prologue for calculating the global
offset table address since it does not have an explicit GOT pointer.

\subsection{Data Objects}

Not done yet.

\subsection{Function Calls}

\begin{figure}[H]
\Hrule
\caption{Position-Independent Direct Function Call}
\begin{center}
\begin{tabular}{|l|c|l|}
\cline{1-1}\cline{3-3}
extern void function ();  &&.globl function\\
function ();              &&call function@PLT\\
\cline{1-1}\cline{3-3}
\end{tabular}
\end{center}
\Hrule
\end{figure}

\begin{figure}[H]
\Hrule
\caption{Position-Independent Indirect Function Call}
\begin{center}
\begin{tabular}{|l|c|l|}
\cline{1-1}\cline{3-3}
extern void (*ptr) ();    &&.globl ptr, name\\
extern void name ();      && \\
ptr = name;               &&movl ptr@GOTPCREL(\%rip), \%rax  \\
                          &&movl name@GOTPCREL(\%rip), \%rdx  \\
                          &&movl \%rdx, (\%rax)  \\
                          &&  \\
(*ptr)();                 &&movl ptr@GOTPCREL(\%rip), \%rax  \\
                          &&call *(\%rax)  \\
\cline{1-1}\cline{3-3}
\end{tabular}
\end{center}
\Hrule
\end{figure}


%%% Local Variables: 
%%% mode: latex
%%% TeX-master: "abi"
%%% End: 
