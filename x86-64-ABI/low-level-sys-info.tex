\chapter{Low Level System Information}

\section{Machine Interface}

\subsection{Processor Architecture}

\subsection{Data Representation}

Within this specification, the term \emph{\textindex{\byte{}}} refers to
a 8-bit object, the term \emph{\textindex{\twobyte{}}} refers to a 16-bit
object, the term \emph{\textindex{\fourbyte{}}} refers to a 32-bit
object, the term \emph{\textindex{\eightbyte{}}} refers to a 64-bit
object, and the term \emph{\textindex{\sixteenbyte{}}} refers to a
128-bit object.%
\footnote{The \intelabi uses the term \emph{\textindex{halfword}} for
  a 16-bit object, the term \emph{\textindex{word}} for a 32-bit
  object, the term \emph{\textindex{doubleword}} for a 64-bit object.  But
  most ia32 processor specific documentation define a
  \emph{\textindex{word}} as a 16-bit object, a
  \emph{\textindex{doubleword}} as a 32-bit object, a
  \emph{\textindex{quardword}} as a 64-bit object and a
  \emph{\textindex{double quadword}} as a 128-bit object.}

\subsubsection{Fundamental Types}

Figure~\ref{basic-types} shows the correspondence between ISO C's
scalar types and the processor's. The \code{__int128},
\code{__float128}, \code{__m64} and \code{__m128} types are optional.

\begin{figure}
  \caption{Scalar Types}\label{basic-types}
{ % Use small here - the table is still too large
  % Has anybody an idea how to shrink the table so that it fits the page?
  \small
  \begin{tabular}{l|l|c|c|l}
    \hline\noalign{\smallskip}
     & &  & \multicolumn{1}{c|}{Alignment} & \multicolumn{1}{c|}{x86-64} \\
    \multicolumn{1}{c|}{Type} & \multicolumn{1}{c|}{C}
     &  \texttt{sizeof} & (bytes)
     & \multicolumn{1}{c|}{Architecture}  \\
    \hline
    & \texttt{char}        & 1 & 1 & signed byte \\
    & \texttt{signed char} & & \\
    \cline{2-5}
    & \texttt{unsigned char} & 1 & 1 & unsigned byte \\
    \cline{2-5}
    & \texttt{short} & 2 & 2 & signed \twobyte \\
    & \texttt{signed short} & & \\
    \cline{2-5}
    & \texttt{unsigned short} & 2 & 2 & unsigned \twobyte \\
    \cline{2-5}
    & \texttt{int} & 4 & 4 & signed \fourbyte \\
    Integral & \texttt{signed int} & & \\
    & \texttt{enum} & & \\
    \cline{2-5}
    & \texttt{unsigned int} & 4 & 4 & unsigned \fourbyte \\
    \cline{2-5}
    & \texttt{long} & 8 & 8 & signed \eightbyte \\
    & \texttt{signed long} & & \\
    & \texttt{long long} & & \\
    & \texttt{signed long long} & & \\
    \cline{2-5}
    & \texttt{unsigned long} & 8 & 8 & unsigned \eightbyte \\
    & \texttt{unsigned long long} & 8 & 8 & unsigned \eightbyte \\
    \hline
    & \texttt{__int128}$^\dagger$ & 16 & 16 & signed \sixteenbyte \\
    & \texttt{signed __int128}$^\dagger$ & 16 & 16 & signed \sixteenbyte \\
    \hline
    & \texttt{unsigned __int128}$^\dagger$ & 16 & 16 & unsigned \sixteenbyte \\
    \hline
    Pointer & \texttt{\textit{any-type} *} & 8 & 8 & unsigned \eightbyte \\
    & \texttt{\textit{any-type} (*)()} & & \\
    \hline
    Floating-& \texttt{float} & 4 & 4 & single (IEEE) \\
    point & \texttt{double} & 8 & 8 & double (IEEE) \\
    & \texttt{long double} & 16 & 16 & 80-bit extended (IEEE) \\
    & \texttt{__float128}$^\dagger$ & 16 & 16 & 128-bit extended (IEEE) \\
    \hline
    Packed & \texttt{__m64}$^\dagger$ & 8 & 8 & \MMX{} and \threednow \\
    & \texttt{__m128}$^\dagger$ & 16 & 16 & SSE and SSE-2 \\
\noalign{\smallskip}
\cline{1-2}\multicolumn{3}{l}{\small $^\dagger$ These types are optional.}\\
  \end{tabular}
}
\end{figure}

The \codeindex{__float128} type uses a 15-bit exponent, a 113-bit
mantissa (the high order significant bit is implicit) and an exponent
bias of 16383.\footnote{Initial implementations of the \xARCH
  architecture are expected to support operations on the
  \texttt{__float128} type only via software emulation.}

The \code{long double} type uses a 15 bit exponent, a 64-bit mantissa
with an explicit high order significant bit and an exponent bias of
16383.\footnote{This type is the x87 double extended precision data
  type.}  Although a \code{long double} requires 16 bytes of storage,
only the first 10 bytes are significant.  The remaining six bytes are
tail padding, and the contents of these bytes are undefined.

The \code{__int128} type is stored in little-endian order in memory,
i.e., the 64 low-order bits are stored at a a lower address than the
64 high-order bits.

A null pointer (for all types) has the value zero.

Like the Intel386 architecture, the \xARCH architecture does not
require all data access to be properly aligned.  Accessing misaligned
data will be slower than accessing properly aligned data, but
otherwise there is no difference.

\subsubsection{Aggregates and Unions}

An array uses the same alignment as its elements, except that a local
or global array variable that requires at least 16 bytes, or a C99
local or global variable-length array variable, always has alignment
of at least 16 bytes.\footnote{The alignment requirement allows the
  use of SSE instructions when operating on the array.  The compiler
  cannot in general calculate the size of a variable-length array (VLA), but
  it is expected that most VLAs will require at least 16 bytes, so it
  is logical to mandate that VLAs have at least a 16-byte alignment.}

No other changes required.

\subsubsection{Bit-Fields}

Amend the description of bit-field ranges as follows:

\begin{figure}[h]
\Hrule
  \caption{Bit-Field Ranges}
  \begin{center}
    \leavevmode
    \begin{tabular}{l|l|l}
      \multicolumn{1}{c}{Bit-field Type}
         & \multicolumn{1}{c}{Width $w$}
         & \multicolumn{1}{c}{Range} \\
      \hline
      \texttt{signed long} & & $-2^{w - 1}$ to $2^{w-1}-1$ \\
      \texttt{long} & 1 to 64 & 0 to $2^{w}-1$ \\
      \texttt{unsigned long} & & 0 to $2^{w}-1$ \\
    \end{tabular}
  \end{center}
\Hrule
\end{figure}

The ABI does not permit bitfields having the type \texttt{__m64} or
\texttt{__m128}.  Programs using bitfields of these types are not
portable.

No other changes required.

\section{Function Calling Sequence}

This section describes the standard function calling sequence,
including stack frame layout, register usage, parameter passing and so
on.

The standard calling sequence requirements apply only to global
functions.  Local functions that are not reachable from other
compilation units may use different conventions.  Nevertheless, it is
recommended that all functions use the standard calling sequence when
possible.

\subsection{Registers and the Stack Frame}
\label{subsec-registers}

The \xARCH architecture provides 16 general purpose 64-bit registers.
In addition the architecture provides 16 SSE registers, each 128 bits
wide and 8 x87 floating point registers, each 80 bits wide.  Each of
the x87 floating point registers may be referred to in \MMX/\threednow
mode as a 64-bit register.  All of these registers are global to all
procedures in a running program.

This subsection discusses usage of each register.  Registers \RBP, \RBX and
\reg{r12} through \reg{r15} ``belong'' to the calling function and the
called function is required to preserve their values.  In other words,
a called function must preserve these registers' values for its
caller.  Remaining registers ``belong'' to the called
function.\footnote{Note that in contrast to the \intelabi, \RDI,
  and \RSI belong to the called function, not the caller.}  If a
calling function wants to preserve such a register value across a
function call, it must save the value in its local stack frame.

The CPU shall be in x87 mode upon entry to a function.  Therefore,
every function that uses the \MMX registers is required to issue
an \op{emms} or \op{femms} instruction before accessing the \MMX
registers.\footnote{All x87 registers are caller-saved, so
  callees that make use of the \MMX registers may use the faster
  \op{femms} instruction.}  The direction flag in the \reg{eflags}
register must be clear on function entry, and on function return.


\subsection{The Stack Frame}

In addition to registers, each function has a frame on the run-time
stack.  This stack grows downwards from high addresses.  Figure
\ref{fig-stack-frame} shows the stack organization.

\begin{figure}
\Hrule
  \caption{Stack Frame with Base Pointer}
  \label{fig-stack-frame}
  \begin{center}
    \begin{tabular}{r|c|l}
      \noalign{\smallskip}
      \multicolumn{1}{l}{Position} &
      \multicolumn{1}{c}{Contents} &
      \multicolumn{1}{l}{Frame} \\
      \noalign{\smallskip}  \cline{1-3}
      \code{8n+16(\RBP)} & argument \eightbyte $n$ \\
      & \dots & Previous \\
      \code{16(\RBP)} & argument \eightbyte $0$ \\
      \cline{1-3}
      \code{8(\RBP)} & return address \\ \cline{2-2}
      \code{0(\RBP)} & previous \RBP value \\
      \cline{2-2}
      \code{-8(\RBP)} & unspecified & Current \\
      & \dots & \\
      \code{0(\RSP)} & variable size \\
      \cline{2-2}
      \code{128(\RSP)} & red zone\\
    \end{tabular}
  \end{center}
\Hrule
\end{figure}

The end of the input argument area shall be aligned on a 16 byte
boundary.  In other words, the value $(\RSP - 8)$ is always a multiple
of $16$ when control is transferred to the function entry point.  The
stack pointer, \RSP, always points to the end of the latest allocated
stack frame.  \footnote{The conventional use of \RBP{} as a frame
  pointer for the stack frame may be avoided by using \RSP (the stack
  pointer) to index into the stack frame.  This technique saves two
  instructions in the prologue and epilogue and makes one additional
  general-purpose register (\RBP) available.}

The 128-byte area beyond the location pointed to by \RSP is considered
to be reserved and shall not be modified by signal or interrupt
handlers.\footnote{Locations within 128 bytes can be addressed using
  one-byte displacements.}  Therefore, functions may use this area for
temporary data that is not needed across function calls.  In
particular, leaf functions may use this area for their entire stack
frame, rather than adjusting the stack pointer in the prologue and
epilogue.

\subsection{Parameter Passing}
\label{sec-calling-conventions}

After the argument values have been computed, they are placed in
registers, or pushed on the stack.  The way how values are passed is
described in the following sections.

\paragraph{Definitions}
We first define a number of classes to classify arguments.  The
classes are corresponding to x86-64 register classes and defined as:

\begin{description}
\item[INTEGER] This class consists of integral types that fit into one of
  the general purpose registers (\RAX--\reg{r15}).
\item[SSE] The class consists of types that fits into a SSE register.
\item[SSEUP] The class consists of types that fit into a SSE register and
  can be passed in the most significant half of it.
\item[X87, X87UP] These classes consists of types that will be passed via
  the x87 FPU.
\item[NO\_CLASS] This class is used as initializer in the algorithms.  It
  will be used for padding and empty structures and unions.
\item[MEMORY] This class consists of types that will be passed in memory
  via the stack.
\end{description}


\paragraph{Classification}
The size of each argument gets rounded up to
\eightbytes.\footnote{Therefore the stack will always be \eightbyte aligned.}

The basic types are assigned their natural classes:
\begin{itemize}
\item Arguments of types (signed and unsigned) \code{char},
  \code{short}, \code{int}, \code{long}, \code{long long}, and
  pointers are in the INTEGER class.
\item Arguments of types \code{float}, \code{double} and \code{__m64}
  are in class SSE.
\item Arguments of types \code{__float128} and \code{__m128} are split
  into two halves.  The least significant ones belong to class SSE, the
  most significant one to class SSEUP.
\item The 64-bit mantissa of arguments of type \code{long double}
  belongs to class X87, the 16-bit exponent plus 6 bytes of padding
  belongs to class X87UP.
\item Arguments of type \code{__int128} are in the MEMORY class.
\end{itemize}


The classification of aggregate (structures and arrays) and union
types works as follows:

\begin{enumerate}
\item If the size of an object is larger than two \eightbytes, or
    in C++, is a non-\textindex{POD}%
\footnote{The term POD is from the ANSI/ISO C++ Standard, and
    stands for Plain Ol' Data.  Although the exact definition is
    technical, a POD is essentially a structure or union that could
    could have been written in C; there cannot be any member
    functions, or base classes, or similar C++ extensions.}
  structure or union type, or contains unaligned fields, it has class
  MEMORY.%
   \footnote{A non-POD object cannot be passed in registers
    because such objects must have well defined addresses; the address
    at which an object is constructed (by the caller) and the address
    at which the object is destroyed (by the callee) must be the same.
    Similar issues apply when returning a non-POD object from a
    function.}

\item Both \eightbytes get initialized to class NO_CLASS.

\item Each field of an object is classified recursively so that always
   two fields are considered.  The resulting class is calculated
   according to the classes of the fields in the \eightbyte:
   \begin{enumerate}
   \item
      If both classes are equal, this is the resulting class.
   \item If one of the classes is NO_CLASS, the resulting class is the other class.
   \item If one of the classes is MEMORY, the result is the MEMORY class.
   \item If one of the classes is INTEGER, the result is the INTEGER.
   \item If one of the classes is X87 or X87UP class, MEMORY is used as
      class.
   \item Otherwise class SSE is used.
   \end{enumerate}
\item Then a post merger cleanup is done:
  \begin{enumerate}
  \item If one of the classes is MEMORY, the whole argument is passed in memory.
  \item If SSEUP is not preceeded by SSE, it is converted to SSE.
  \end{enumerate}
\end{enumerate}

\paragraph{Passing}
Once arguments are classified, the registers get assigned (in
left-to-right order) for passing as follows:

\begin{enumerate}
\item If the class is MEMORY, pass the argument on the stack.

\item If the class is INTEGER, the next available register of the
  sequence \RDI, \RSI, \RDX, \RCX, \reg{r8} and \reg{r9} is
  used\footnote{Note that \reg{r11} is neither required to be
    preserved, nor is it used to pass arguments.  Making this register
    available as scratch register means that code in the PLT need not
    spill any registers when computing the address to which control
    needs to be transferred.  \RAX is used to indicate the number of
    SSE arguments passed to a function requiring a variable number of
    arguments. \reg{r10} is used for passing a function's static chain
    pointer.}.

\item If the class is SSE, the next available SSE register is used, the
   registers are taken in the order from \reg{xmm0} to \reg{xmm15}.

item If the class is SSEUP, the \eightbyte is passed in the upper
   half of the least used SSE register.

\item If the class is X87 or X87UP, it is passed in memory.
\end{enumerate}

\begin{figure}
\Hrule
  \caption{Register Usage}
  \label{fig-reg-usage}
  \begin{center}
    \begin{tabular}{l|p{6.2cm}|l}
      \noalign{\smallskip}
      \multicolumn{1}{c}{} &
      \multicolumn{1}{c}{}&
      \multicolumn{1}{l}{Preserved across}\\
      \multicolumn{1}{c}{Register} &
      \multicolumn{1}{c}{Usage}&
      \multicolumn{1}{l}{function calls}\\
      \hline
\RAX & temporary register; with variable arguments passes
information about the number of SSE registers used; 1$^{\rm st}$
return register & No \\
\RBX & callee-saved register; optionally used as base pointer & Yes \\
\RCX & used to pass 4$^{\rm th}$ integer argument to functions & No \\
\RDX & used to pass 3$^{\rm rd}$ argument to functions ; 2$^{\rm nd}$ return register & No \\
\RSP & stack pointer & Yes \\
\RBP & callee-saved register & Yes \\
\RSI & used to pass 2$^{\rm nd}$  argument to functions & No \\
\RDI & used to pass 1$^{\rm st}$  argument to functions & No \\
\reg{r8} & used to pass 5$^{\rm th}$  argument to functions & No \\
\reg{r9} & used to pass 6$^{\rm th}$  argument to functions & No \\
\reg{r10} & temporary register, used for passing a function's static
chain pointer & No \\
\reg{r11} & temporary register & No\\
\reg{r12--r15} & callee-saved registers & Yes \\
\reg{xmm0}--\reg{xmm1} & used to pass and return floating point
arguments & No\\
\reg{xmm2}--\reg{xmm15} & used to pass floating point arguments & No\\
\reg{mmx0}--\reg{mmx7}& temporary registers & No\\
\reg{st0} & temporary register; used to return \code{long double} arguments & No \\
\reg{st1}--\reg{st7} & temporary registers & No \\
    \end{tabular}

  \end{center}
\Hrule
\end{figure}

If there is no register available anymore for any \eightbyte of an
argument, the whole argument is passed on the stack. If registers have
already been assigned for some \eightbytes of this argument, those
assignments get reverted.

Once registers are assigned, the arguments passed in memory are pushed
on the stack in reversed (right-to-left\footnote{Right-to-left order
  on the stack makes the handling of functions that take a variable
  number of arguments simpler.  The location of the first argument can
  always be computed statically, based on the type of that argument.
  It would be difficult to compute the address of the first argument
  if the arguments were pushed in left-to-right order.}) order.

For calls that may call functions that use varargs or stdargs
(prototype-less calls or calls to functions containing ellipsis
(\dots) in the declaration) \RAX is used as hidden argument to specify
the number of SSE registers used. The contents of \RAX do not need to
match exactly the number of registers, but must be an upper bound on
the number of SSE registers used and is in the range 0--15 inclusive.

\paragraph{Returning of Values}
The returning of values is done according to the following algorithm:
\begin{enumerate}
\item Classify the return value with the classification algorithm.

\item If the value has class MEMORY, then the caller provides space for
   the return value and passes the address of this storage in \RDI as
   if it were the first argument to the function.  In effect, this
   address becomes a ``hidden'' first argument.

\item If the class is INTEGER, the next available register of the
   sequence \RAX, \RDX is used.

\item If the class is SSE, the next available SSE register of the
   sequence \reg{xmm0}, \reg{xmm1} is used.

\item If the class is SSEUP, the \eightbyte is passed in the upper half of the
   least used SSE register.

\item If the class is X87, the value is returned on the X87 stack in
   \reg{st0} as 80-bit x87 number.

\item If the class is X87UP, the value is returned together with the
   previous X87 value in \reg{st0}.
\end{enumerate}

As an example of the register passing conventions, consider the
declarations and the function call shown in
Figure \ref{fig_passing_example}.  The corresponding register
allocation is given in Figure \ref{fig_allocation_example}, the stack
frame offset given shows the frame before calling the function.

\begin{figure}[H]
\Hrule
\caption{Parameter Passing Example}
\label{fig_passing_example}
\begin{center}
\code{
\begin{tabular}{|l|}
\cline{1-1}
typedef struct \{ \\
\ \ int a, b;\\
\ \ double d;\\
\} structparm;\\
structparm s;\\
int e, f, g, h, i, j, k;\\
long double ld;\\
double m, n;\\
\\
extern void func (int e, int f,\\
\phantom{extern void func (}structparm s, int g, int h,\\
\phantom{extern void func (}long double ld, double m,\\
\phantom{extern void func (}double n, int i, int j, int k);\\
);\\
\\
func (e, f, s, g, h, ld, l, m, n, i, j, k);\\
\cline{1-1}
\end{tabular}
}
\end{center}
\Hrule
\end{figure}

\begin{figure}[H]
\Hrule
\caption{Register Allocation Example}
\label{fig_allocation_example}
\begin{center}
\begin{tabular}{ll|ll|ll}
\multicolumn{2}{c}{General Purpose Registers} &
\multicolumn{2}{c}{Floating Point Registers} &
\multicolumn{2}{c}{Stack Frame Offset}\\
\hline
\RDI: &\code{e} & \reg{xmm0}: &\code{s.d} &\code{0:}& \code{ld} \\
\RSI: &\code{f} & \reg{xmm1}: &\code{m}& \code{16:}& \code{j} \\
\RDX: &\code{s.a,s.b} & \reg{xmm2}: &\code{n}&\code{24:}& \code{k} \\
\RCX: &\code{g} & & & \\
\reg{r8}:&\code{h} & && & \\
\reg{r9}:&\code{i} & && & \\
\end{tabular}

\end{center}
\Hrule
\end{figure}

\section{Operating System Interface}

\subsection{Exception Interface}

As the \xARCH manuals describe, the processor changes mode to handle
\emph{\textindex{exceptions},} which may be synchronous,
floating-point/coprocessor or asynchronous.
Synchronous and floating-point/coprocessor exceptions,
being caused by instruction execution, can be explicitly generated
by a process. This section, therefore, specifies those exception types
with defined behavior. The \xARCH architecture classifies exceptions as
\emph{faults}, \emph{traps}, and \emph{aborts}. See the \intelabi
for more information about their differences.

\subsubsection{Hardware Exception Types}
The operating system defines the correspondence between hardware
exceptions and the signals specified by \codeindex{signal}(BA\_OS)
as shown in table \ref{tab-hw-exceptions}. Contrary to the i386
architecture, the \xARCH does not define any instructions that
generate a bounds check fault in long mode.

\begin{table}
\Hrule
  \caption{Hardware Exceptions and Signals}
  \label{tab-hw-exceptions}
  \begin{center}
    \begin{tabular}[t]{c|l|l}
      \multicolumn{1}{c}{Number} & \multicolumn{1}{c}{Exception name}
         & \multicolumn{1}{c}{Signal}\\
      \hline
      \texttt{0} & \texttt{divide error fault} & \texttt{SIGFPE} \\
      \texttt{1} & \texttt{single step trap/fault} & \texttt{SIGTRAP} \\
      \texttt{2} & \texttt{nonmaskable interrupt} & \texttt{none} \\
      \texttt{3} & \texttt{breakpoint trap} & \texttt{SIGTRAP} \\
      \texttt{4} & \texttt{overflow trap} & \texttt{SIGSEGV} \\
      \texttt{5} & \texttt{(reserved)} &\\
      \texttt{6} & \texttt{invalid opcode fault} & \texttt{SIGILL} \\
      \texttt{7} & \texttt{no coprocessor fault} & \texttt{SIGFPE} \\
      \texttt{8} & \texttt{double fault abort} & \texttt{none} \\
      \texttt{9} & \texttt{coprocessor overrun abort} & \texttt{SIGSEGV} \\
      \texttt{10} & \texttt{invalid TSS fault} & \texttt{none} \\
      \texttt{11} & \texttt{segment no present fault} & \texttt{none} \\
      \texttt{12} & \texttt{stack exception fault} & \texttt{SIGSEGV} \\
      \texttt{13} & \texttt{general protection fault/abort}&\texttt{SIGSEGV} \\
      \texttt{14} & \texttt{page fault} & \texttt{SIGSEGV} \\
      \texttt{15} & \texttt{(reserved)} &\\
      \texttt{16} & \texttt{coprocessor error fault} & \texttt{SIGFPE} \\
      \texttt{other} & \texttt{(unspecified)} & \texttt{SIGILL}
    \end{tabular}
  \end{center}
\Hrule
\end{table}

\begin{table}
\Hrule
  \caption{Floating-Point Exceptions}
  \begin{center}
    \begin{tabular}[t]{l|l}
      \multicolumn{1}{c}{Code} & \multicolumn{1}{c}{Reason} \\
      \hline
      \texttt{FPE\_FLTDIV} & \texttt{floating-point divide by zero} \\
      \texttt{FPE\_FLTOVF} & \texttt{floating-point overflow} \\
      \texttt{FPE\_FLTUND} & \texttt{floating-point underflow} \\
      \texttt{FPE\_FLTRES} & \texttt{floating-point inexact result} \\
      \texttt{FPE\_FLTINV} & \texttt{invalid floating-point operation}
    \end{tabular}
  \end{center}
\Hrule
\end{table}

\subsection{Special Registers}

The \xARCH architecture defines floating point instructions.  At
process startup the two floating point units, SSE2 and x87, both have
all floating-point exception status flags cleared.  The status of the
control words is as defined in tables \ref{x87-fpucw} and
\ref{mxcsr-status}.


\begin{table}
\Hrule
  \caption{x87 Floating-Point Control Word}
  \label{x87-fpucw}
  \begin{center}
    \begin{tabular}[t]{l|l|l}
      \multicolumn{1}{c}{Field} & \multicolumn{1}{c}{Value}& \multicolumn{1}{c}{Note} \\
      \hline
      \texttt{RC} & 0 & Round to nearest\\
      \texttt{PC} & 11& Double extended precision\\
      \texttt{PM} & 1& Precision masked\\
      \texttt{UM} & 1& Underflow masked\\
      \texttt{OM} & 1& Overflow masked\\
      \texttt{ZM} & 1& Zero divide masked\\
      \texttt{DM} & 1& Denormal operand masked\\
      \texttt{IM} & 1& Invalid operation masked\\
    \end{tabular}
  \end{center}
\Hrule
\end{table}

\begin{table}
\Hrule
  \caption{MXCSR Status Bits}
  \label{mxcsr-status}
  \begin{center}
    \begin{tabular}[t]{l|l|l}
      \multicolumn{1}{c}{Field} & \multicolumn{1}{c}{Value}& \multicolumn{1}{c}{Note} \\
      \hline
      \texttt{FZ} & 0 & Do not flush to zero\\
      \texttt{RC} & 0 & Round to nearest\\
      \texttt{PM} & 1& Precision masked\\
      \texttt{UM} & 1& Underflow masked\\
      \texttt{OM} & 1& Overflow masked\\
      \texttt{ZM} & 1& Zero divide masked\\
      \texttt{DM} & 1& Denormal operand masked\\
      \texttt{IM} & 1& Invalid operation masked\\
      \texttt{DAZ} & 0 & Denormals are not zero\\
    \end{tabular}
  \end{center}
\Hrule
\end{table}

\subsection{Virtual Address Space}

Although the \xARCH architecture uses 64-bit pointers, implementations
are only required to handle 48-bit addresses.  Therefore, conforming
processes may only use addresses from \texttt{0x00000000\,00000000} to
\texttt{0x00007fff\,ffffffff}\footnote{0x0000ffff\,ffffffff is not a
  canonical address and cannot be used.}.

No other changes required.

\subsection{Page Size}

Systems are permitted to use any power-of-two page size between 4KB
and 64KB, inclusive.

No other changes required.

\subsection{Virtual Address Assignments}

Conceptually processes have the full address space available.
In practice, however, several factors limit the size of a process.
\begin{itemize}
  \item The system reserves a configuration dependent amount of virtual space.
  \item The system reserves a configuration dependent amount of space per
    process.
  \item
    A process whose size exceeds the system's available combined physical
    memory and secondary storage cannot run. Although some physical memory
    must be present to run any process, the system can execute processes that
    are bigger than physical memory, paging them to and from secondary storage.
    Nonetheless, both physical memory and secondary storage are
    shared resources. System load, which can vary from one program execution
    to the next, affects the available amount.
\end{itemize}

\begin{figure}[H]
\Hrule
  \caption{Virtual Address Configuration}
  \label{fig-address}
  \begin{center}
    \begin{tabular}{r|c|l}
      \noalign{\smallskip}  \cline{2-2}
      \verb|0xffffffffffffffff| & Reserved system area & End of memory\\
      & \dots & \\ \cline{2-2}
      & \dots & \\
      \verb|0x80000000000| & Dynamic segments & \\ \cline{2-2}
      & \dots & \\
      \verb|0| & Process segments & Beginning of memory\\ \cline{2-2}
    \end{tabular}
  \end{center}
\Hrule
\end{figure}

Although applications may control their memory assignments, the typical
arrangement appears in figure \ref{fig-cfg}.

\begin{figure}[H]
\Hrule
  \caption{Conventional Segment Arrangements}
  \label{fig-cfg}
  \begin{center}
    \begin{tabular}{r|c|l}
      \cline{2-2}
      & \dots & \\
      \verb|0x80000000000| & Dynamic segments & \\ \cline{2-2}
      & Stack segment & \\
      & \dots & \\ \cline{2-2}
      & \dots & \\
      & Data segments & \\ \cline{2-2}
      & \dots & \\
      \verb|0x10000| & Text segments & \\ \cline{2-2}
      \verb|0| & Unmapped & \\ \cline{2-2}
    \end{tabular}
  \end{center}
\Hrule
\end{figure}

\section{Process Initialization}

\subsection{Auxiliary Vector}

The \xARCH ABI uses the following auxiliary vector types.

\begin{figure}[H]
\Hrule
\caption{Auxiliary Vector Types}
\label{aux-vec}
\begin{center}
\begin{tabular}{l|r|l}
  \multicolumn{1}{c}{Name}
         & \multicolumn{1}{c}{Value}
         & \multicolumn{1}{c}{a_un} \\
      \hline
\texttt{AT_NULL}& 0 & ignored\\
\texttt{AT_IGNORE}& 1& ignored\\
\texttt{AT_EXECFD}& 2& \texttt{a_val}\\
\texttt{AT_PHDR}& 3& \texttt{a_ptr}\\
\texttt{AT_PHENT}& 4& \texttt{a_val}\\
\texttt{AT_PHNUM}& 5& \texttt{a_val}\\
\texttt{AT_PAGESZ}& 6& \texttt{a_val}\\
\texttt{AT_BASE}& 7& \texttt{a_ptr}\\
\texttt{AT_FLAGS}& 8& \texttt{a_val}\\
\texttt{AT_ENTRY}& 9& \texttt{a_ptr}\\
\texttt{AT_NOTELF}& 10& \texttt{a_val}\\
\texttt{AT_UID}& 11& \texttt{a_val}\\
\texttt{AT_EUID}& 12& \texttt{a_val}\\
\texttt{AT_GID}& 13& \texttt{a_val}\\
\texttt{AT_EGID}& 14& \texttt{a_val}\\
\hline
    \end{tabular}
  \end{center}
\Hrule
\end{figure}

The entries that are different than in the \intelabi are
specified as follows:

\begin{description}
\item[AT_NOTELF] The \code{a_val} member of this entry is non-zero if
  the program is in another format than ELF.
\item[AT_UID] The \code{a_val} member of this entry holds the real
  user id of the process.
\item[AT_EUID] The \code{a_val} member of this entry holds the
  effective user id of the process.
\item[AT_GID] The \code{a_val} member of this entry holds the
  real group id of the process.
\item[AT_EGID] The \code{a_val} member of this entry holds the
  effective group id of the process.
\item[AT_EUID] The \code{a_val} member of this entry holds the
  effective user id of the process.
\end{description}


\section{Coding Examples}

The following sections show only the difference to the i386 ABI.

\subsection{Architectural Constraints}

The \xARCH architecture usually does not allow to encode arbitrary
64-bit constants as immediate operand of the instruction.  Most
instructions accept 32-bit immediates that are sign extended to the
64-bit ones.  Additionally the 32-bit operations with register
destinations implicitly perform zero extension making loads of 64-bit
immediates with upper half set to 0 even cheaper.

Additionally the branch instructions accept 32-bit immediate operands
that are sign extended and used to adjust instruction pointer.
Similarly an instruction pointer relative addressing mode exists for
data accesses with equivalent limitations.

In order to improve performance and reduce code size, it is desirable
to use different \textindex{code models} depending on the
requirements.

Code models define constraints for symbolic values that allow the
compiler to generate better code.  Basically code models differ in
addressing (absolute versus position independent), code size, data
size and address range.  We define only a small number of code models
that are of general interest:

\begin{description}
\item[\textindex{Small code model}]
  The virtual address of code executed is known at link time.
  Additionally all symbols are known to be located in the virtual
  addresses in the range from $0$ to $2^{31}-2^{10} - 1$.

  This allows the compiler to encode symbolic references with offsets
  in the range from $-2^{31}$ to $2^{10}$ directly in the sign
  extended immediate operands, with offsets in the range from $0$ to
  $2^{31}+2^{10}$ in the zero extended immediate operands and use
  instruction pointer relative addressing for the symbols with offsets
  in the range $-2^{10}$ to $2^{10}$.

  This is the fastest code model and we expect it to be suitable for
  the vast majority of programs.

\item[\textindex{Kernel code model}]

  The kernel of an operating system is usually rather small but runs
  in the negative half of the address space.  So we define all symbols
  to be in the range from $2^{64}-2^{31}$ to $2^{64}-2^{10}$.

  This code model has advantages similar to those of the small model,
  but allows encoding of zero extended symbolic references only for
  offsets from $2^{31}$ to $2^{31}+2^{10}$. The range offsets for
  sign extended reference changes to $0$--$2^{31}+2^{10}$.

\item[\textindex{Medium code model}]

  The medium code model does not make any assumptions about the range
  of symbolic references to data sections. Size and address of the
  text section have the same limits as the small code model.

  This model requires the compiler to use \code{movabs} instructions
  to access static data and to load addresses into register, but keeps
  the advantages of the small code model for manipulation of addresses
  to the text section (specially needed for branches).

\item[\textindex{Large code model}]

  The large code model makes no assumptions about addresses and sizes
  of sections.

  The compiler is required to use the \code{movabs} instruction, as in
  the medium code model, even for dealing with addresses inside the
  text section.  Additionally, indirect branches are needed when
  branching to addresses whose offset from the current instruction
  pointer is unknown.

  It is possible to avoid the limitation for the text section by
  breaking up the program into multiple shared libraries, so we do not
  expect this model to be needed in the foreseeable future.

\item[\textindex{Small position independent code model} (\textindex{PIC})]

  Unlike the previous models, the virtual addresses of instructions
  and data are not known until dynamic link time.  So all addresses
  have to be relative to the instruction pointer.

  Additionally the maximum distance between a symbol and the end of an
  instruction is limited to $2^{31}-2^{10}-1$, allowing the compiler
  to use instruction pointer relative branches and addressing modes
  supported by the hardware for every symbol with an offset in the
  range $-2^{10}$ to $2^{10}$.

\item[\textindex{Medium position independent code model}
  (\textindex{PIC})]

  This model is like the previous model, but makes no assumptions
  about the distance of symbols to the data section.

  In the medium PIC model, the instruction pointer relative addressing
  can not be used directly for accessing static data, since the offset
  can exceed the limitations on the size of the displacement field in
  the instruction.  Instead an unwind sequence consisting of
  \code{movabs}, \code{lea} and \code{add} needs to be used.

\item[\textindex{Large position independent code model}
  (\textindex{PIC})]

  This model is like the previous model, but makes no assumptions
  about the distance of symbols.

  The large PIC model implies the same limitation as the medium
  PIC model regarding addressing of static data.  Additionally,
  references to the global offset table and to the procedure linkage
  table and branch destinations need to be calculated in a similar
  way.

\end{description}

\subsection{Position-Independent Function Prologue}

\xARCH does not need any function prologue for calculating the global
offset table address since it does not have an explicit GOT pointer.

\subsection{Data Objects}

Not done yet.

\subsection{Function Calls}

\begin{figure}[H]
\Hrule
\caption{Position-Independent Direct Function Call}
\begin{center}
\begin{tabular}{|l|c|l|}
\cline{1-1}\cline{3-3}
extern void function ();  &&.globl function\\
function ();              &&call function@PLT\\
\cline{1-1}\cline{3-3}
\end{tabular}
\end{center}
\Hrule
\end{figure}

\begin{figure}[H]
\Hrule
\caption{Position-Independent Indirect Function Call}
\begin{center}
\begin{tabular}{|l|c|l|}
\cline{1-1}\cline{3-3}
extern void (*ptr) ();    &&.globl ptr, name\\
extern void name ();      && \\
ptr = name;               &&movl ptr@GOTPCREL(\%rip), \%rax  \\
                          &&movl name@GOTPCREL(\%rip), \%rdx  \\
                          &&movl \%rdx, (\%rax)  \\
                          &&  \\
(*ptr)();                 &&movl ptr@GOTPCREL(\%rip), \%rax  \\
                          &&call *(\%rax)  \\
\cline{1-1}\cline{3-3}
\end{tabular}
\end{center}
\Hrule
\end{figure}

\subsection{Variable Argument Lists}

Some otherwise portable C programs depend on the argument passing
scheme, implicitly assuming that 1) all arguments are passed on the
stack, and 2) arguments appear in increasing order on the stack.
Programs that make these assumptions never have been portable, but
they have worked on many implementations. However, they do not work on
the \xARCH architecture because some arguments are passed in
registers.  Portable C programs must use the header files
\code{<stdarg.h>} or \code{<varargs.h>} in order to handle variable
argument lists.

When a function taking variable-arguments is called, \reg{rax} must be
set to eight times the number of floating point parameters passed to
the function in SSE registers.

\subsubsection{The Register Save Area\index{register save area}}

The prologue of a function taking a variable argument list and known
to call the macro \code{va_start} is expected to save the argument
registers to the \emph{register save area}.  Each argument register
has a fixed offset in the register save area as defined in the figure
\ref{fig-reg-save}.

Only registers that might be used to pass arguments need to be saved.
Other registers are not accessed and can be used for other purposes.  If a
function is known to never accept arguments passed in
registers\footnote{This fact may be determined either by exploring
  types used by the \code{va_arg}
macro, or by the fact that the named
  arguments already are exhausted the argument registers entirely.},
the register save area may be omitted entirely.

The prologue should use \RAX to avoid unnecessarily saving XMM
registers.  This is especially important for integer only programs to
prevent the initialization of the XMM unit.


\begin{figure}[H]
\Hrule
\caption{Register Save Area}
\label{fig-reg-save}
\begin{center}
\begin{tabular}{l|r}
\multicolumn{1}{c}{Register}&\multicolumn{1}{c}{Offset}\\
\hline
\RDI & $0$ \\
\RSI & $8$ \\
\RDX & $16$ \\
\RCX & $24$ \\
\reg{r8} & $32$ \\
\reg{r9} & $40$ \\
\reg{xmm0} & $48$ \\
\reg{xmm1} & $64$ \\
\dots &  \\
\reg{xmm15} & $288$ \\
\end{tabular}
\end{center}
\Hrule
\end{figure}

\subsubsection{The \codeindex{va_list} Type}

The \code{va_list} type is an array containing a single element of one
structure
containing the necessary information to implement the \code{va_arg} macro. The C
definition of \code{va_list} type is given in figure \ref{fig-va_list}.

\begin{figure}[H]
\Hrule
\caption{\code{va_list} Type Declaration}
\label{fig-va_list}
\begin{center}
\code{
\begin{tabular}{|l|}
\cline{1-1}
typedef struct \{\\
\ \ \ unsigned int gp_offset;\\
\ \ \ unsigned int fp_offset;\\
\ \ \ void *overflow_arg_area;\\
\ \ \ void *reg_save_area;\\
  \} va_list[1];\\
\cline{1-1}
\end{tabular}
}
\end{center}
\Hrule
\end{figure}


\subsubsection{The \codeindex{va_start} Macro}

The \code{va_start} macro initializes the structure as follows:

\begin{description}
\item [reg_save_area]
The element points to the start of the register save area.
\item [overflow_arg_area] This pointer is used to fetch arguments
  passed on the stack.  It is initialized with the address of the
  first argument passed on the stack, if any, and then always updated
  to point to the start of the next argument on the stack.
\item [gp_offset] The element holds the offset in bytes from
  \code{reg_save_area} to the place where the next available general
  purpose argument register is saved.  In case all argument registers
  have been exhausted, it is set to the value 48 ($6*8$).
\item [fp_offset]
The element holds the offset in bytes from \code{reg_save_area} to the
place where the next available floating point
argument register is saved.  In case all argument registers have been exhausted,
it is set to the value 304 ($6*8+16*16$).
\end{description}

\subsubsection{The \codeindex{va_arg} Macro}

The algorithm for the generic \code{va_arg(l, type)} implementation is
defined as follows:

\begin{enumerate}
\item
Determine whether \code{type} may be passed in the registers.  If not go to step \ref{stack}.
\item
Compute \code{num_gp} to hold the number of general purpose registers needed to pass \code{type}
and \code{num_fp} to hold the number of floating point registers needed.
\item
Verify whether arguments fit into registers.
In the case:
$$\code{l->gp_offset} > 48 - \code{num_gp} * 8$$ or $$\code{l->fp_offset} > 304 - \code{num_fp} * 16$$
go to step \ref{stack}.
\item
  Fetch \code{type} from \code{l->reg_save_area} with an offset of
  \code{l->gp_offset} and/or \code{l->fp_offset}.  This may require
  copying to a temporary location in case the parameter is passed in
  different register classes or requires an alignment greater than 8 for
  general purpose registers and 16 for XMM registers.
\item
Set:
$$\code{l->gp_offset} = \code{l->gp_offset} + \code{num_gp} * 8$$
$$\code{l->fp_offset} = \code{l->fp_offset} + \code{num_fp} * 16.$$
\item
Return the fetched \code{type}.
\item
\label{stack}
Align \code{l->overflow_arg_area} upwards to a 16 byte boundary if
alignment needed by \code{type} exceeds 8 byte boundary.
\item
Fetch \code{type} from \code{l->overflow_arg_area}.
\item
Set \code{l->overflow_arg_area} to:
$$\code{l->overflow_arg_area} + \code{sizeof} (\code{type})$$
\item
Align \code{l->overflow_arg_area} upwards to an 8 byte boundary.
\item
Return the fetched \code{type}.
\end{enumerate}

The \code{va_arg} macro is usually implemented as a compiler builtin and expanded in
simplified forms for each particular type.  Figure \ref{fig-va_arg} is a sample
implementation of the \code{va_arg} macro.
\begin{figure}[H]
\Hrule
\caption{Sample Implementation of \code{va_arg(l, int)}}
\label{fig-va_arg}
\begin{center}
\begin{tabular}{|llll|}
\cline{1-4}
&movl&\code{l->gp_offset}, \reg{eax}&\\
&cmpl&\$48, \reg{eax}&Is register available?\\
&jae&stack&If not, use stack\\
&leal&\$8(\RAX), \reg{edx}&Next available register\\
&addq&\code{l->reg_save_area}, \RAX&Address of saved register\\
&movl&\reg{edx}, \code{l->gp_offset}&Update \code{gp_offset}\\
&jmp&fetch&\\
stack:&movq&\code{l->overflow_arg_area}, \RAX&Address of stack slot\\
&leaq&8(\RAX), \RDX&Next available stack slot\\
&movq&\RDX,\code{l->overflow_arg_area}&Update\\
fetch:&movl&(\RAX), \reg{eax}&Load argument\\
\cline{1-4}
\end{tabular}
\end{center}
\Hrule
\end{figure}

\section{DWARF Definition}

This section\footnote{This section is structured in a way similar to the psABI for PowerPC}
defines the Debug With Arbitrary Record Format (DWARF) debugging
format for the \xARCH processor family. The \xARCH ABI does not define
a debug format.  However, all systems that do implement DWARF shall use
the following definitions.

DWARF is a specification developed for symbolic, source-level debugging.
The debugging information format does not favor the design of any
compiler or debugger.  For more information on DWARF,
see \emph{DWARF Debugging Information Format},
revision: Version 2.0.0, July 27, 1993, UNIX International,
Program Languages SIG.

\subsection{DWARF Release Number}

The DWARF definition requires some machine-specific definitions.
The register number mapping needs to be specified for the \xARCH
registers. In addition, the DWARF Version 2 specification
requires processor-specific address class codes to be defined.

\subsection{DWARF Register Number Mapping}

Table \ref{tbl-reg-num-map} outlines the register number mapping
for the \xARCH processor family.%
\footnote{This document does not define mappings for privileged registers.}%
\footnote{The table defines Return Address to have a register number, even
though the address is stored in 0(\RSP) and not in a physical register.}

\begin{figure}
\caption{DWARF Register Number Mapping} \label{tbl-reg-num-map}
\begin{center}
\begin{tabular}{l|r|l}
\multicolumn{1}{c}{Register Name}&\multicolumn{1}{c}{Number}&\multicolumn{1}{c}{Abbreviation}\\
\hline
General Purpose Register RAX & 0 &\RAX\\
General Purpose Register RBX & 1 &\RBX\\
General Purpose Register RCX & 2 &\RCX\\
General Purpose Register RDX & 3 &\RDX\\
Frame Pointer Register   RBP & 4 &\RBP\\
General Purpose Register RSI & 5 &\RSI\\
General Purpose Register RDI & 6 &\RDI\\
Stack Pointer Register   RSP & 7 &\RSP\\
Extended Integer Registers 8-15 & 8-15 &\reg{r8}--\reg{r15}\\
Return Address RA
& 16&\\
SSE Registers 0--7              & 17-24 & \reg{xmm0}--\reg{xmm7} \\
Extended SSE Registers 8--15    & 25-32 & \reg{xmm8}--\reg{xmm15} \\
Floating Point Registers 0--7   & 33-40 & \reg{st0}--\reg{st7} \\
MMX Registers 0--7              & 41-48 & \reg{mm0}--\reg{mm7} \\
\end{tabular}
\end{center}
\end{figure}

%%% Local Variables:
%%% mode: latex
%%% TeX-master: "abi"
%%% End:
