
\chapter{Low Level System Information}

\section{Machine Interface}

\subsection{Processor Architecture}

The x86-64 architecture defines a new mode, referred to as 32/64-bit
mode. See x86-64 Specification details. All information in this
document is relevant only to 32/64-bit mode, unless stated otherwise.

\editornote{Is this correct?}

\subsection{Data Representation}

In accordance with x86 tradition and x86-64 register names the terms
byte, word, dword, qword and tword refer to the corresponding data
objects of 8, 16, 32, 64 and 128 bits in size.

\editornote{Is tword correct for 128 bits?  Elsewhere oword seems to
  be used instead}

\subsubsection{Fundamental Types}

The basic types are specified in Table \ref{tab-basic-types}.  The
\verb|__int128| and \verb|__mm64| types are optional.

\begin{table}
\caption{Basic Types}\label{tab-basic-types}
\begin{tabular}{lrllr}
\hline\noalign{\smallskip}
% multicolumn is used to shrink the table
&sizeof & type			&\multicolumn{2}{l}{C type\hfill alignment} \\
\noalign{\smallskip}\hline\noalign{\smallskip}
Integral:
&1	& signed byte		&        char, signed char		& 1 \\
&1	& unsigned byte		&        unsigned char, bool		& 1 \\
&2	& signed word		&        short, singed short		& 2 \\
&2	& unsigned word		&        unsigned short			& 2 \\
&4	& signed dword		&        int, signed int, wchar		& 4 \\
&4	& unsigned dword	&        unsigned int,	& 4 \\
&       &                       &        unsigned wchar & \\
&8	& signed qword		&        long, signed long, 	& 8 \\
&        &                       &       long long, 		&  \\
&        &                       &       signed long long		&  \\
&8	& unsigned qword	&                unsigned long,		&	8 \\
&        &                       &          unsigned long long		& \\
&16	& signed tword		&        \verb|__int128|, & 16 \\
&	& 		        &        signed \verb|__int128|	&  \\
&16	& unsigned oword	&                unsigned \verb|__int128|	&	16 \\
\hline
Packed:
&8	& MMX and 3DNOW		&	\verb|__mm64|				& 8 \\
&16	& SSE and SSE-2		&	\verb|__m128|				& 16 \\
\hline
Pointer:
&8	& unsigned qword	&		any type *		& 8 \\
\hline
Floating-
&4	& single-precision (IEEE) &	float				& 4 \\
point:&8	& double-precision (IEEE) &	double				& 8 \\
&16	& extended-precision (IEEE) & 	long double			& 16 \\
\hline
\end{tabular}
\end{table}

\subsubsection{Aggregates and Unions}

Structure members are "`naturally"' aligned.  Structures, arrays and
unions assume the alignment of the most strictly aligned component
(compare with Table \ref{tab-basic-types}). The size of aggregates is
a multiple of the alignment.


\subsubsection{Bit-Fields}

Bit-fields are non-negative values with the range from 1 to 64 bits
depending on the bit-field type. Bit ordering is equivalent to
Intel386 specs.

\editornote{This needs a bit (pun intended) more specificity.}

\section{Function Calling Sequence}

Function may expect at the entry and must preserve in the calls:
\begin{enumerate}
\item CPU in i387 mode (we generally want to replace MMX with SSE, on the other
      hand, we want the same for i387.  i387 is still having advantage of 80bit
      long double hardware support, so it is probably more usefull at the moment)
\item direction flag to be clear (this will allow us to eliminate most of cli
      instructions).  In gcc this will bring an complications with assembly
      statements.  I suggest to define direction flag to be clear at the
      statements exit.
\item rsp to be stack pointer, rbp to be base pointer, rbx to be GOT pointer
\item define FP control word value?
\item define values of other cpu flags?
\end{enumerate}
Return value:
\begin{enumerate}
\item Integers come in rax, 128bits in rdx:rax
\item Floats come in xmm0 (including the long double, that will require an
      load from i387 stack, but would allow to forget about the stack in future)
\item Structures up to 128bits are mapped to the rax/rdx
      ??? We may define fields to be mapped to the corresponding register,
      (so fp fields will be return in xmm register)
      Other structures are returned in the area preallocated by the caller
      obtained as the hidden first argument.
\end{enumerate}

Register usage:
\begin{enumerate}
\item rsp, rbp, rbx, xmm8-xmm15, r10b-r15b are preserved across
      the function calls
\item rax, rdx and xmm0 are return registers
\item xmm0-5, rax, rdx, rsi, rdi, rcx and r8 are used for register passing
    conventions.  Uppermost
    bytes of the argument area are not pushed to the stack and mapped to the
    registers instead. Xth oword from the top comes in the Xth register.
    When prototype is present, Integer comes in the integers, fp and
    packed types comes in floats.
    For the ellipsis, everyting comes in the integer (and to in the floats).
    For prototypeless calling convention, the floats and packed comes in
    both.
    Usually only 64bytes of each xmm register is used, long double comes in
    single xmm register and the value of next xmm register is undefined.
\end{enumerate}

 Argument area:
\begin{enumerate}
\item Size of every value is rounded to 64bits, padded by caller, extended as
    necesary by callee.
    (??? Zero extend by caller?)
\item 128bit values are padded to be aligned expecting the byte after
      uppermost byte of the argument area is 16 byte aligned.
\end{enumerate}

\begin{table}
\caption{General Registers}
\begin{center}
\begin{tabular}{ll}
\hline\noalign{\smallskip}
Register & Use \\
\noalign{\smallskip}\hline\noalign{\smallskip}
RAX & \\
RBX & \\
RCX & \\
RDX & \\
RDI & \\
RSI & \\
RBP & \\
RSP & \\
R8 & \\
R9 & \\
R10 & \\
R11 & \\
R12 & \\
R13 & \\
R14 & \\
R15 & \\
\hline
\end{tabular}
\end{center}
\end{table}

\begin{table}
\caption{Floating Point Registers}
\begin{center}
\begin{tabular}{ll}
\hline\noalign{\smallskip}
Register & Use \\
\noalign{\smallskip}\hline\noalign{\smallskip}
XMM0& \\
XMM1 & \\
XMM2 & \\
XMM3 & \\
XMM4 & \\
XMM5 & \\
XMM6 & \\
XMM7 & \\
XMM8 & \\
XMM9 & \\
XMM10 & \\
XMM11 & \\
XMM12 & \\
XMM13 & \\
XMM14 & \\
XMM15 & \\
\hline
\end{tabular}
\end{center}
\end{table}

%%% Local Variables: 
%%% mode: latex
%%% TeX-master: "abi"
%%% End: 
